\documentclass{article}
\usepackage{titling}
\usepackage{hyperref}

\setlength{\droptitle}{-10em}


\title{Biographical Sketch}
\author{Benjamin Fuller}
\date{}

\begin{document}
\maketitle


\noindent \textbf{Professional Preparation}\\
Rensselaer Polytechnic Institute, Troy, New York, 12180, USA. \\Mathematics and Computer Science B.S., 2006

\noindent
Boston University, Boston, Massachusetts, 02215, USA.\\Computer Science M.A., 2011

\noindent
Boston University, Boston, Massachusetts, 02215, USA.\\Computer Science Ph.D., 2015

\vspace{.1in}
\noindent \textbf{Appointments}


\noindent
Assistant Professor, \\
Department of Computer Science and Engineering\\
University of Connecticut, Storrs, Connecticut (2016 -- Present)

\vspace{.1in}\noindent
Research Scientist,\\
MIT Lincoln Laboratory, Lexington, Massachusetts (2007 -- 2016)

\vspace{.1in}
\noindent
\textbf{Most Relevant Products} 

\begin{enumerate}
\item Benjamin Fuller, Mayank Varia, Arkady Yerukhimovich, Emily Shen, Ariel Hamlin, Vijay Gadepally, Richard Shay, John Darby Mitchell, Robert Cunningham. \emph{SoK: Cryptographically Protected Database Search.}  IEEE Security \& Privacy, May 2017.
\item Benjamin Fuller, Leonid Reyzin, Adam Smith.  \emph{When are Fuzzy Extractors Possible?} Advances in Cryptology -- Asiacrypt 2016.  \href{http://eprint.iacr.org/2014/961}{http://eprint.iacr.org/2014/961}.
\item Ran Canetti, Benjamin Fuller, Omer Paneth, Leonid Reyzin, Adam Smith.  \emph{Reusable fuzzy extractors for low-entropy distributions}  Advances in Cryptology -- Eurocrypt, 117-146, 2016.  \href{http://eprint.iacr.org/2014/243}{http://eprint.iacr.org/2014/243}.
\item Benjamin Fuller, Leonid Reyzin, Xianrui Meng.  \emph{Computational Fuzzy Extractors}.  Advances in Cryptology -- Asiacrypt, 174-193, December 2013.  \href{http://eprint.iacr.org/2013/416}{http://eprint.iacr.org/2013/416}.

\item Benjamin Fuller, Adam O'Neill, Leonid Reyzin. \emph{A Unified Approach to Deterministic Encryption: New Constructions and a Connection to Computational Entropy}. Journal of Cryptology.  Vol. 28, No. 3, 671-717.  \href{http://eprint.iacr.org/2012/005}{http://eprint.iacr.org/2012/005}.
\end{enumerate}

\noindent
\textbf{Other Products}
\begin{enumerate}

\item Benjamin Fuller, Ariel Hamlin.  \emph{Unifying Leakage Classes: Simulatable Leakage and Pseudoentropy}.  International Conference on Information Theoretic Security, 69-86, 2015. \href{http://eprint.iacr.org/2015/857}{http://eprint.iacr.org/2015/857}.

\item Merrielle Spain, Benjamin Fuller, Kyle Ingols, Robert Cunningham.  \emph{Robust Keys from Physical Uncloneable Functions}.  IEEE Symposium on Hardware Oriented Security and Trust, 88-92, 2014. \href{http://ieeexplore.ieee.org/xpls/abs_all.jsp?arnumber=6855575}{Available here}.

\item Robert Cunningham, Benjamin Fuller, Sophia Yakoubov.  \emph{Catching MPC Cheaters: Identification and Openability}.  To appear at International Conference on Information Theoretic Security  2017.  \href{http://eprint.iacr.org/2016/611}{http://eprint.iacr.org/2016/611}.

\item Joseph Cooley, Roger Khazan, Benjamin Fuller, Galen Pickard.  \emph{GROK: A Practical System for Securing Group Communications}.  IEEE Network Computing and Applications, 100-107, 2010. \href{http://ieeexplore.ieee.org/xpls/abs_all.jsp?arnumber=5598226}{Available here}.

\item Jeremy Blackthorne, Benjamin Kaiser, Benjamin Fuller, Bulent Yener.  \emph{Environmental Authentication in Malware}.  Latincrypt 2017.  \href{https://eprint.iacr.org/2017/928}.
\end{enumerate}


\noindent
\textbf{Synergistic Activities}
\begin{enumerate}
\item Dr. Fuller is the advisor of \href{https://uconntact.uconn.edu/organization/UCSC}{Cyber Security club} at the University of Connecticut.  This club engages students at all levels, teaching how computers work and how to secure computing resources.

\item Program Committee Member for ICITS 2016, 2017, TCC 2017.
\item Dr. Fuller is developing cybersecurity concentration at the University of Connecticut.  Dr. Fuller has developed new courses on network security and cryptography.

\item Dr. Fuller served as panel member for NSF SaTC program.
\end{enumerate}
%\textbf{Collaborators, advisors, advisees}
%\\
%\underline{Collaborators (past 48 months)} (29): 
%Jeremy Blackthorne (RPI), Joseph Campbell (MIT/LL), Ran Canetti (BU), Venkat Chandar (D.E. Shaw and Co.), Robert Cunningham (MIT/LL), Srinivas Devadas (MIT), Vijay Gadepally (MIT/LL), Ariel Hamlin (MIT/LL and NEU), Charles Herder (MIT), Kyle Ingols (MIT/LL), Gene Itkis (MIT/LL), Chenglu Jin (UConn), Benjamin Kaiser (MIT/LL), Xianrui Meng (Apple), John Darby Mitchell (MIT/LL), Phuong Ha Nyugen (UConn), Adam O'Neill (Georgetown), Omer Paneth (MIT), Leonid Reyzin (BU), Richard Shay (MIT/LL), Emily Shen (MIT/LL), Adam Smith (BU), Merrielle Spain (Unaffiliated), Ling Ren (MIT), Marten van Dijk (UConn), Mayank Varia (BU), Sophia Yakoubov (MIT/LL and BU), Bulent Yener (RPI), Arkady Yerukhimovich (MIT/LL).\\
%
%\vspace{.05in} \noindent
%\underline{Graduate Advisor} (1):
%Leonid Reyzin, Boston University.\\
%
%\vspace{.05in} \noindent
%\underline{Thesis Advisor and Postgraduate-Scholar Sponsor} (3): Trevor Phillips (UConn), Sailesh Simhadri (UConn), Shreya Varshney (Uconn).

%\reversemarginpar % Move the margin to the left of the page 
%
%\newcommand{\MarginText}[1]{\marginpar{\raggedleft\itshape\small#1}} % New command defining the margin text style
%
%\usepackage[nochapters]{classicthesis} % Use the classicthesis style for the style of the document
%\usepackage[LabelsAligned,NoDate]{currvita} % Use the currvita style for the layout of the document
%\usepackage{hyperref}
%
%\renewcommand{\cvheadingfont}{\LARGE\color{Maroon}} % Font color of your name at the top
%
%%\usepackage{hyperref} % Required for adding links	and customizing them
%%\hypersetup{colorlinks, breaklinks, urlcolor=Maroon, linkcolor=Maroon} % Set link colors
%
%\newlength{\datebox}\settowidth{\datebox}{Spring 2011} % Set the width of the date box in each block
%
%\newcommand{\NewEntry}[3]{\noindent\hangindent=2em\hangafter=0 \parbox{\datebox}{\small \textit{#1}}\hspace{1.5em} #2 #3 % Define a command for each new block - change spacing and font sizes here: #1 is the left margin, #2 is the italic date field and #3 is the position/employer/location field
%\vspace{0.5em}} % Add some white space after each new entry
%
%\newcommand{\Description}[1]{\hangindent=2em\hangafter=0\noindent\raggedright\footnotesize{#1}\par\normalsize\vspace{1em}} % Define a command for descriptions of each entry - change spacing and font sizes here
%\newcounter{publicationcounter}
%\cvplace{Updated November 7, 2016}
%%----------------------------------------------------------------------------------------
%
%\begin{document}
%
%%\thispagestyle{empty} % Stop the page count at the bottom of the first page
%
%%----------------------------------------------------------------------------------------
%%	NAME AND CONTACT INFORMATION SECTION
%%----------------------------------------------------------------------------------------
%
%\begin{cv}{
%%\spacedallcaps{Benjamin W. Fuller}}\vspace{1.5em} % Your name
%
%\noindent\spacedlowsmallcaps{Personal Information}\vspace{0.5em} % Personal information heading
%
%%\NewEntry{}{\textit{Born in Canada,}}{20 November 1987} % Birthplace and date
%
%\NewEntry{email}{\href{mailto:benjamin.fuller@uconn.edu}{benjamin.fuller@uconn.edu}} % Email address
%
%\NewEntry{website}{\url{http://www.engr.uconn.edu/~bfuller}} % Personal website
%
%\NewEntry{phone}{(O) +1 (860) 486 2122}\ \ %$\cdotp$\ \ (M) +1 (000) 111 1112} % Phone number(s)
%
%\NewEntry{address}{371 Fairfield Way, Unit 4155, Storrs, CT 06269}
%
%\vspace{1em} % Extra white space between the personal information section and goal
%
%\noindent\spacedlowsmallcaps{Goal}\vspace{1em} % Goal heading, could be used for a quotation or short profile instead
%
%\Description{Advance security and cryptography research using techniques from information-theory and complexity.  Emphasize practical schemes that can be transitioned to use.  Educate scientists and responsible citizens in computer science and engineering.}\vspace{2em} % Goal text
%
%%\vspace{1em} % Extra space between major sections
%
%%----------------------------------------------------------------------------------------
%%	EDUCATION
%%----------------------------------------------------------------------------------------
%
%\spacedlowsmallcaps{Education}\vspace{1em}
%
%\NewEntry{2012-2014}{Boston University}
%
%\Description{\MarginText{Doctor of Philosophy}GPA: 3.96/4.00\ \ $\cdotp$\ \ Department: Computer Science \newline 
%%%Thesis: \textit{Money Is The Root Of All Evil -- Or Is It?}\newline
%%%Description: This thesis explored the idea that money has been the cause of untold anguish and suffering in the world. I found that it has, in fact, not.\newline
%Dissertation: Strong Key Derivation from Noisy Sources
%\newline
%Advisor: Leonid \textsc{Reyzin}
%\newline
%\textbf{Awards}: Computer Science Research Excellence Award}
%
%
%\NewEntry{2009-2011}{Boston University}
%
%\Description{\MarginText{Masters of Arts}GPA: 3.93/4.00\ \ $\cdotp$\ \ Department: Computer Science \newline 
%Thesis: \textit{Computational Entropy and Information Leakage}
%\newline
%Advisor: Leonid \textsc{Reyzin}}
%
%%------------------------------------------------
%
%\NewEntry{2003-2006}{Rensselaer Polytechnic Institute}
%
%\Description{\MarginText{Bachelor of Science}GPA: 4.00/4.00\ \ $\cdotp$\ \ Department: Mathematics and Computer Science\newline
%\textbf{Awards}: Rensselaer Medal Winner\ \ $\cdotp$ \ \ Computer Science Scholar's Award}
%%----------------------------------------------------------------------------------------
%%	WORK EXPERIENCE
%%----------------------------------------------------------------------------------------
%
%%------------------------------------------------
%
%
%
%
%
%%----------------------------------------------------------------------------------------
%%	PUBLICATIONS
%%----------------------------------------------------------------------------------------
%\spacedlowsmallcaps{Publications}\footnote{This list contains works in both the theory and systems communities.  In the theory community, authors are listed alphabetically, in the systems community, authors are listed by contribution.}
%\vspace{1.5em}
%
%\spacedlowsmallcaps{Conference Papers} \vspace{1em}
%\Description{
%\begin{enumerate}
%\item Benjamin \MarginText{Asiacrypt 2016} \textsc{Fuller}, Leonid \textsc{Reyzin}, and Adam \textsc{Smith}. \emph{When are Fuzzy Extractors Possible?} To appear at Asiacrypt 2016.
%\item Ran \MarginText{Eurocrypt 2016} \textsc{Canetti}, Benjamin \textsc{Fuller}, Omer \textsc{Paneth}, Leonid \textsc{Reyzin}, and Adam \textsc{Smith}.  \emph{Reusable Fuzzy Extractors via Digital Lockers.} Eurocrypt 2016.  Also presented without proceedings at Allerton 2014.
%\item Benjamin \MarginText{ICITS 2015} \textsc{Fuller} and Ariel \textsc{Hamlin}. \emph{Unifying Leakage Classes: Simulatable Leakage and Pseudoentropy.}  ICITS 2015.
%\item Merrielle \MarginText{HOST 2014} \textsc{Spain}, Benjamin \textsc{Fuller}, Kyle \textsc{Ingols}, and Robert \textsc{Cunningham}. \emph{Robust  Keys from Physical Unclonable Functions.} IEEE Symposium on Hardware Oriented Security and Trust, 2014. 
%\item Benjamin \MarginText{Asiacrypt 2013} \textsc{Fuller}, Leonid \textsc{Reyzin}, and Xianrui \textsc{Meng}. \emph{Computational Fuzzy Extractors.}
%Advances in Cryptology -- Asiacrypt, December 2013.
%\item Galen \MarginText{SICK 2012} \textsc{Pickard}, Roger \textsc{Khazan}, Benjamin \textsc{Fuller}, and Joseph \textsc{Cooley}. \emph{DSKE: Dynamic Set Key Encryption.} LCN Workshop on Security in Communication Networks, 2012.
%\item Benjamin \MarginText{TCC 2012} \textsc{Fuller}, Adam \textsc{O'Neil}, and Leonid \textsc{Reyzin}.
%\emph{A Unified Approach to Deterministic Encryption -- New Constructions and a Connection to Computational Entropy.}
%Theory of Cryptography, 2012. Also presented without proceedings at ICITS 2012.
%
%\item Benjamin \MarginText{NCA 2010} \textsc{Fuller}, Roger \textsc{Khazan}, Joseph \textsc{Cooley}, and Galen \textsc{Pickard}. \emph{ASE: Authenticated Statement Exchange.} IEEE Network Computing and Applications, 2010. \textbf{Award:} Best Paper.
%\item Joseph\MarginText{NCA 2010} \textsc{Cooley}, Roger \textsc{Khazan}, Benjamin \textsc{Fuller}, and Galen \textsc{Pickard}. \emph{GROK: A Practical System for Securing Group Communications.} IEEE Network Computing and Applications, 2010. \textbf{Award:} Best Paper Nominee.
%\item Roger \MarginText{MILCOM 2008} \textsc{Khazan}, Joseph \textsc{Cooley}, Galen \textsc{Pickard}, and Benjamin \textsc{Fuller}. \emph{GROK Secure Multi-User Chat at Red Flag 2007-03.} Military Communications Conference, 2008.
%\item Tamara \textsc{Yu}, \MarginText{Vizsec 2007}Benjamin \textsc{Fuller}, John \textsc{Bannick}, Lee \textsc{Rossey}, and Robert \textsc{Cunningham}. \emph{Integrated Environment Management for Informaiton Operations Testbeds.} Workshop on Visualization for Computer Security, 2007.
%\setcounter{publicationcounter}{\theenumi}
%\end{enumerate}}
%%\Description{\MarginText{ICITS 2015}Benjamin \textsc{Fuller} and Ariel \textsc{Hamlin}. \emph{Unifying Leakage Classes: Simulatable Leakage and Pseudoentropy.}  ICITS 2015.}
%%
%%\Description{\MarginText{HOST 2014}Robust Keys from Physical Unclonable Functions. \emph{IEEE Symposium on Hardware Oriented Security and Trust}. Merrielle \textsc{Spain}, Benjamin \textsc{Fuller}, Kyle \textsc{Ingols}, and Robert \textsc{Cunningham}}
%%
%%\Description{\MarginText{Asiacrypt 2013}Benjamin \textsc{Fuller}, Leonid \textsc{Reyzin}, and Xianrui \textsc{Meng}. \emph{Computational Fuzzy Extractors.}
%%Advances in Cryptology -- Asiacrypt.  December 2013.}
%%
%%\Description{\MarginText{Theory of Cryptography 2012}Benjamin \textsc{Fuller}, Adam \textsc{O'Neil}, and Leonid \textsc{Reyzin}.
%%\emph{A Unified Approach to Deterministic Encryption -- New Constructions and a Connection to Computational Entropy.}
%%Theory of Cryptography 2012. Also presented without proceedings at ICITS 2012.}
%%
%%\Description{\MarginText{SICK 2012}Galen \textsc{Pickard}, Roger \textsc{Khazan}, Benjamin \textsc{Fuller}, and Joseph \textsc{Cooley}. \emph{DSKE: Dynamic Set Key Encryption.} LCN Workshop on Security in Communication Networks, 2012.}
%%
%%\Description{\MarginText{NCA 2010}Benjamin \textsc{Fuller}, Roger \textsc{Khazan}, Joseph \textsc{Cooley}, and Galen \textsc{Pickard}. \emph{ASE: Authenticated Statement Exchange.} IEEE Network Computing and Applications, 2010. \textbf{Best Paper Award.}}
%%
%%\Description{\MarginText{NCA 2010}Joseph \textsc{Cooley}, Roger \textsc{Khazan}, Benjamin \textsc{Fuller}, and Galen \textsc{Pickard}. \emph{GROK: A Practical System for Securing Group Communications.} IEEE Network Computing and Applications, 2010. \textbf{Best Paper Nominee.}}
%%
%%\Description{\MarginText{MILCOM 2008}Roger \textsc{Khazan}, Joseph \textsc{Cooley}, Galen \textsc{Pickard}, and Benjamin \textsc{Fuller}. \emph{GROK Secure Multi-User Chat at Red Flag 2007-03.} Military Communications Conference, 2008.}
%%
%%\Description{\MarginText{Vizsec 2007}Tamara \textsc{Yu}, Benjamin \textsc{Fuller}, John \textsc{Bannick}, Lee \textsc{Rossey}, and Robert \textsc{Cunningham}. \emph{Integrated Environment Management for Informaiton Operations Testbeds.} Workshop on Visualization for Computer Security, 2007.}
%
%\spacedlowsmallcaps{Journal Papers}\vspace{1em}
%\Description{
%\begin{enumerate}
%\setcounter{enumi}{\thepublicationcounter}
%\item Benjamin\MarginText{Journal of Cryptology 2013} \textsc{Fuller}, Adam \textsc{O'Neil}, and Leonid \textsc{Reyzin}.
%\emph{A Unified Approach to Deterministic Encryption -- New Constructions and a Connection to Computational Entropy.}
%Journal of Cryptology 2013. %This paper contains significant new material that did not appear in the Theory of Cryptography version.
%\setcounter{publicationcounter}{\theenumi}
%\end{enumerate}
%}
%
%\spacedlowsmallcaps{Magazine Articles}\vspace{1em}
%\Description{
%\begin{enumerate}
%\setcounter{enumi}{\thepublicationcounter}
%\item Gene\MarginText{IEEE Signal Processing Magazine 2015} \textsc{Itkis}, Venkat \textsc{Chandar}, Benjamin \textsc{Fuller}, Joseph \textsc{Campbell}, Robert \textsc{Cunningham}. \emph{Iris Biometric Security Challenges and Possible Solutions: For your eyes only? Using the iris as a key}. IEEE Signal Processing Magazine, 2015.
%\setcounter{publicationcounter}{\theenumi}
%\end{enumerate}}
%
%\spacedlowsmallcaps{Papers in Submission}\vspace{1em}
%%\Description{%\MarginText{2014} 
%
%\Description{Robert \textsc{Cunningham}, Benjamin \textsc{Fuller}, and Sophia \textsc{Yakoubov}. \emph{Catching MPC Cheaters: Identification and Openability.} 2016.}
%
%\spacedlowsmallcaps{Papers in Preparation}\vspace{1em}
%%\Description{%\MarginText{2014} 
%
%\Description{Benjamin \textsc{Fuller}, Mayank \textsc{Varia}, Arkady \textsc{Yerukhimovich}, Emily \textsc{Shen}, Ariel \textsc{Hamlin}, Vijay \textsc{Gadepally}, Richard \textsc{Shay}, John Darby \textsc{Mitchell}, and Robert \textsc{Cunningham}. \emph{SoK: Cryptographically Protected Database Search.} 2016.}
%
%\Description{Chenglu \textsc{Jin}, Phuong Ha \textsc{Nguyen}, Marten \textsc{van Dijk}, and Benjamin \textsc{Fuller}. \emph{Strong PUFs via LPN.} 2016.}
%
%\spacedlowsmallcaps{Theses}\vspace{1em}
%
%\Description{\MarginText{Ph.D. Dissertation}\emph{Strong Key Derivation from Noisy Sources}, 
%\newline
%%Synopsis:  A shared cryptographic key enables strong authentication.  Candidate sources for creating such a shared key include biometrics and physically unclonable functions.  However, these sources come with a substantial problem: noise in repeated readings.  A fuzzy extractor produces a stable key from a noisy source.  For many sources of practical importance, traditional fuzzy extractors provide no meaningful security guarantee.  This dissertation improves fuzzy extractors.  First, we show how to incorporate structural information about the physical source to facilitate key derivation.  Second, most fuzzy extractors work by first recovering the initial reading from the noisy reading.  We improve key derivation by producing a consistent key without recovering the original reading.  Third, traditional fuzzy extractors provide information-theoretic security.  We build fuzzy extractors achieving new properties by only providing security against computational bounded adversaries.\newline
%Readers: Leonid \textsc{Reyzin}, Ran \textsc{Canetti} \& Daniel \textsc{Wichs}
%\newline Boston University Computer Science Department, 2014.
%}
% \Description{\MarginText{Master's Thesis}\emph{Computational Entropy and Information Leakage}, 
%%Description: We investigate how information leakage reduces computational entropy of a random variable. We prove an intuitively natural result: conditioning on an event of probability p reduces the quality of computational  entropy by a factor of p and the quantity of metric entropy by log 1/p.  Our result improves previous bounds of Dziembowski and Pietrzak (FOCS 2008), where the loss in the quantity of entropy was related to its original quality. Our result also simplifies the result of Reingold et. al. (FOCS 2008).
%\newline
%Readers: Leonid \textsc{Reyzin} \& Prof.~Peter \textsc{Gacs}
%\newline Boston University Computer Science Department, 2011.}
%
%
%\spacedlowsmallcaps{Invited Talks and Presentations}\vspace{1em}
%
%\Description{\emph{Strong Key Derivation from Noisy Sources.} \newline Privacy Enhancing Technologies for Biometrics, Haifa, January 2015 \newline MIT Computer and Information Security Seminar, November 2014.}
%\Description{\emph{When are Fuzzy Extractors Possible?}\newline Brown University Crypto Reading Group, October 2014.}
%\Description{\emph{Key Derivation from Noisy Sources with More Errors than Entropy.}\newline Georgetown University, May 2014 \newline MITRE, April 2014.}
%\Description{\emph{A Unified Approach to Deterministic Encryption.} \newline NYC Cryptoday, March 2012.}
%
%\spacedlowsmallcaps{Posters}\vspace{1em}
%
%\Description{\emph{Key Derivation from Noisy Sources with More Errors than Entropy.}  \newline Boston University Computer Science Research Open House, 2014.}
%\Description{\emph{A Unified Approach to Deterministic Encryption.}  \newline Boston University Computer Science Research Open House, 2012.}
%
%
%
%%----------------------------------------------------------------------------------------
%%	Presentations
%%----------------------------------------------------------------------------------------
%
%%\spacedlowsmallcaps{Presentations}\vspace{1em}
%
%%\Description{\MarginText{Basic}\textsc{java}, Adobe Illustrator}
%%
%%\Description{\MarginText{Intermediate}\textsc{python}, \textsc{html}, \LaTeX, OpenOffice, Linux, Microsoft Windows}
%%
%%\Description{\MarginText{Advanced}Computer Hardware and Support}
%
%%------------------------------------------------
%
%\vspace{1em} % Extra space between major sections
%
%%----------------------------------------------------------------------------------------
%%	Teaching
%%----------------------------------------------------------------------------------------
%
%\spacedlowsmallcaps{Teaching}\vspace{1em}
%
%\NewEntry{Spring 2017}{Introduction to Network Security}
%
%\NewEntry{Fall 2016}{UConn CSE 5852: Modern Cryptography: Foundations}
%
%\Description{Class Homepage: \url{http://benjamin-fuller.uconn.edu/teaching/modern-cryptography-foundations/}}
%%\Description{Instructor.}
%%\Description{\MarginText{Awards}2011\ \ $\cdotp$\ \ School of Business Postgraduate Scholarship}
%
%\spacedlowsmallcaps{Teaching Assistant}\vspace{1em}
%
%\NewEntry{Fall 2013}{Introduction to Network Security}
%
%\Description{Professor: Ran \textsc{Canetti}.}
%
%\NewEntry{Fall 2012}{Introduction to Cryptography}
%
%\Description{Professor: Leonid \textsc{Reyzin}.}
%
%\NewEntry{Fall 2006}{Computer Architecture}
%
%\Description{Professor: Franklin \textsc{Luk}.}
%
%\NewEntry{Fall 2006}{Calculus}
%
%\Description{Professor: Bruce \textsc{Piper}.}
%
%\NewEntry{Spring 2006}{Computer Organization}
%
%\Description{Professor: Franklin \textsc{Luk}.}
%
%
%\noindent\spacedlowsmallcaps{Professional Experience}\vspace{1em}
%
%\NewEntry{2015-2016}{Principal Investigator, MIT Lincoln Laboratory}
%
%\Description{\MarginText{Security and Privacy Assurance}\textbf{Contribution:} Served as principal investigator leading research and software development teams, managing between 5-10 staff and 3 research companies.  Primary responsibilities include project development and management, developing new cryptographic approaches, gathering and communicating requirements, specifying test procedures, integration and deployment, and evaluating user experience and technology utility.  Led the adaption, integration, and pilot deployment of privacy-preserving database prototypes in a real use case.\newline \textbf{Background:} Privacy-preserving databases balance the need for individuals' privacy and the need to perform data analytics.  Systems are approaching practical levels of performance for moderate size database systems.}
%
%\NewEntry{2007--2014}{Research Scientist, MIT Lincoln Laboratory}
%
%\Description{Performed research at the intersection of theoretic cryptography and
%secure systems. Major research projects sorted by recency.}
%
%\Description{\MarginText{Secure and Resilient Cloud}\textbf{Contribution:} Evaluated the applicability of multi-party computation to the cloud environment.  Built multi-party computation techniques using a sparse communication network.\newline \textbf{Background:} Computations increasingly occur in a cloud environment.  It is imprudent to assume that all cloud resources operate honestly.   \newline \textbf{Principal Investigator:} Nabil Schear.}
%
%\Description{\MarginText{Secure Cloud Authentication}\textbf{Contribution:} Researched image processing techniques and key derivation techniques to improve iris authentication.  \newline \textbf{Background:} User's data is increasingly pushed to resources they do not control.  Strong authentication is even more important in the cloud environment.  The human iris is a potential authentication source.  \newline \textbf{Principal Investigator:} Gene Itkis.}
%
%\Description{\MarginText{Physical Unclonable Functions}\textbf{Contribution:} Developed an optical physical unclonable function, focus on algorithms for image processing and key derivation.  \newline
%\textbf{Background:} A strong root-of-trust is critical to securing hardware devices.  Physical unclonable functions are one source for a root-of-trust.\newline
%\textbf{Principal Investigator:} Kyle Ingols.}
%
%\Description{\MarginText{Dynamic Group Key Management}\textbf{Contribution:} Developed and deployed new approaches for dynamic key management. \newline \textbf{Background:} Key management is a challenge in real-world cryptographic applications.  Standard approaches use static keys and assume a fixed set of participants.   \newline \textbf{Principal Investigator:} Roger Khazan.}
%
%\Description{\MarginText{Large Scale User Emulation}\textbf{Contribution:} Developed user models and advanced visualizations, enabling repeatable, realistic and scalable evaluations of network technology.  Focused on scalable visualizations. \newline \textbf{Background:} Computer systems are vast and interconnected with many sources of nondeterminism.  This complexity makes it difficult to evaluate new technologies in a repeatable and realistic environment.  \newline
%\textbf{Principal Investigator:} Lee Rossey.}
%
%%------------------------------------------------
%
%\NewEntry{2006}{Intern, National Security Agency}
%
%\Description{Applied mathematical principles to real-life cryptographic problems and protocols.}
%
%%------------------------------------------------
%
%\NewEntry{2005}{Intern, International Business Machines}
%
%\Description{Collected worldwide inventory aging information, and automated process to make business recommendations about assets and reserves.}
%%Worked in the Nerd Herd and helped to solve computer problems by asking customers to turn their computers off and on again. \\ Reference: Big \textsc{Mike}\ \ +1 (000) 111 1111\ \ $\cdotp$\ \ \href{mailto:mike@buymore.com}{mike@buymore.com}}
%
%
%
%\end{cv}

\end{document}