\section{Component C: Improving the Leakage}

Partial preserving encryption, adding noise

\paragraph*{Modularity.}  One mitigation of the attacks we propose to look at is \emph{modularity} in the sense of modular order-preserving encryption (M-OPE)~\cite{C:BolCheOne11}.  The idea of M-OPE is to apply a secret random offset modulo the largest possible message to a message before encrypting it (the secret random offset is chosen once and fixed in the secret key), so that everything gets ``shifted.''  Although similar attacks seem to apply to M-OPE, we propose to investigate more fine-grained modularity as a defense.  In the specific case of order-preserving encryption, we propose \emph{digit-modular} OPE (DM-OPE) where there is a secret modular offset applied to each digit.  The base in which the data is written could even itself be secret.  It becomes more complicated to make range queries with DM-OPE, as it requires an exponential number of queries in the number of ``wrap around'' digits in the query.  However, we propose to investigate approximating the queries efficiently (with some false positives that the client can filter out).