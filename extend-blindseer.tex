%!TEX root = dbproposal.tex

\subsection{BlindSeer in the Cloud Setting}
\label{sec:ext-blindseer}
\paragraph{Simplified threat model.}
When a database is outsourced, the database owner is often the client
itself, in which case we don't need to worry about whether the plaintext data
would be leaked to the client. Considering this setting will allow
numerous performance improvements to Blind Seer:

\begin{itemize}\setlength\itemsep{0em}
\item {\em Much simpler, more efficient setup.}
In this new simpler threat model, the client can be regarded as playing as both
$\sf S$ and $\sf C$ in the original BlindSeer architecture. Therefore, the
client can just hold one symmetric key in contrast to the original BlindSeer
system where a slow public-key encryption scheme and a random shuffling
must be introduced in the setup stage.
This also improves backward compatibility.

\item {\em MPC is not necessary.} In the original BlindSeer system, in order to
hide from $\sf C$ the BF data stored in each node of the search tree, $\sf
C$ and $\sf IS$ have to execute costly MPC computation. In our setting, the
    client can simply ask for the necessary encrypted BF data and decrypt them,
    since we don't need to hide anything from the client.  


\item {\em Dynamic record insertion.}
The original BlindSeer required all BF data to be known during the setup
stage in order to arrange for decryption keys.
Furthermore, data was encrypted with a very
simple mechanism of one-time pad so that MPC computation may be reasonably
efficient.
Both of these choices prevent adding or removing data after initialization.
In our setting, however, we don't need any key setup, nor MPC computation.
Therefore, we can add BF data directly to the search tree with much more
efficiency.  
\end{itemize}
%
For use cases where this new setting is appropriate, we
believe this simplified construction will be an {\em order of magnitude faster}
than the current BlindSeer system.

\paragraph{Proximity of high-dimensional data.}
BlindSeer already provides conjunctions and range queries. We will
use these queries in combination to answer proximity queries on a Euclidean space.
In particular, a query that searches for the points close to $(x, y)$ can be defined as a 2D rectangle defined by the top-left point $(x_1, y_1)$ and the
bottom-right point $(y_1, y_2)$ can be described with range queries and
conjunctions as follows: $$ x_1 \le x \le x_2 ~~{\sf AND}~~~ y_1 \le y \le y_2.
$$ This technique can be easily extended to data with multiple dimensions. 

\paragraph{Graph structure and the shortest path.}
Using Bloom filters we can encode a graph structure as follows:
\begin{itemize}
\item For each edge $(a, b)$ from vertex $a$ to vertex $b$, we insert an
  encryption $c = \Enc((a,b) \| w(a,b) )$, where $w(a,b)$ is the weight of the edge
    $(a,b)$, indexed by a BF keyword `$\sf edge$:a*'. 
\end{itemize}

\noindent
We can find all the neighbors of vertex $a$ along with the weight of the
associated edge. Therefore, we can run variants of Dijkstra's algorithm and
compute the shortest path, given two vertices $a$ and $b$ and a limit
on the maximum number of arcs. The key to this approach is the ``native''
ability of Bloom filters to handle conjunction queries.
% BlindSeer is based on
% the Bloom filters so it can support proximity and graph structure by adding
% relevant keyword index appropriately.
%
%Moreover, considering that BlindSeer allows disjunctions and conjunctions, the
%client can enjoy a much richer set of functionalities.  


\def\HHH{{\cal H}}
\paragraph{Enhancing security.}
In BlindSeer, the search query is executed by the client traversing the Bloom filter
search tree. In particular, in each node of the tree, the client and the server
execute the following:
\begin{enumerate}\setlength\itemsep{0em}
\item For each keyword $\alpha$ in the query, the client computes a hash on
  $\alpha$, based on which the look-up positions $Pos(\alpha) = (i_1, \ldots,
    i_\eta)$ for the BF contents are identified. Here, $\eta$ is a system
    parameter. 

\item The client sends $\{Pos(\alpha_1), \ldots, Pos(\alpha_q)\}$,
  where $\alpha_1,\ldots,\alpha_q$ are the keywords used in the query.

\item The client and the server execute a protocol using the look-up positions
  and the encrypted BF as input, so that the client knows whether the given
query is satisfied. 
\end{enumerate}

The look-up positions that are sent from the client to the server leak
$Pos(\alpha)$ to a significant degree, since $q$ is generally small.
Moreover, $Pos(\alpha)$ is a deterministic function on $\alpha$.
This implies that the server can infer with a reasonable probability whether
two queries contain the same keyword.
Given  auxiliary information about query statistics, the
server may be able to infer the plaintext keywords.

%\paragraph{Removing the leakage of BF indices.}

It is costly to reduce the leakage in the original BlindSeer system. In
particular, the above step 3 (i.e., the protocol execution to check whether the
given query is satisfied) is performed using secure two-party computation based
on Yao's garbled circuit. To reduce the leakage, it is natural to use more
sophisticated cryptographic mechanisms, but then the step 3 of secure
computation would be significantly slower. 

However, as noted above, we don't need secure computation in our simplified
system. In this case, we can actually remove the leakage by completely hiding
the look-up positions. As one promising direction, we can achieve this goal by
storing the encrypted BF using ORAM techniques. Note that the look-up positions
can be regarded as the metadata (i.e., ``addresses") for the encrypted data
(i.e., BF contents), which fits perfectly into the use case for ORAM.
%
We expect that the resulting system will be faster than the current one, since
two-party computation is known to be slower than ORAM. 
