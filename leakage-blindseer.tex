\def\HHH{{\cal H}}
\subsection{Component C-III: Enhancing security of BlindSeer}

\paragraph{Leakage of BF indices.}
In Blindseer, the search query is executed by the client traversing the BF
search tree. In particular, in each node of the tree, the client and the server
execute the following:

\begin{enumerate}
\item For each keyword $\alpha$ in the query, the client computes a hash on
$\alpha$, based on which BF indices $H(\alpha) =  (h_1, h_2, ..., h_\eta)$
are identified. Here, $\eta$ is a system parameter. 

\item Let $\alpha_1, \ldots, \alpha_q$ be the keywords used in the query.  The
  client sends $\HHH = \{H(\alpha_1), \ldots, H(\alpha_q)\}$ to the server. 

\item The client and the server execute a protocol using $\HHH$ and the
encrypted BF as input, so that the client knows whether the given query is
satisfied. 
\end{enumerate}

Note that the message $\HHH$ from the client to the server leaks $H(\alpha)$ to
a significantly degree, since $q$ is a small number most of the time. Moreover,
$H(\alpha)$ is deterministically dependent on $\alpha$. This implies that
although the server can infer with a reasonable probablity whether two queries
contain the same keyword, although it doesn't know what the keyword is.  If the
server is given the auxiliary information about the query statistics, the
server may guess what the queries are. 

\paragraph{Removing the leakage of BF indices.}

We note that it is difficult to reduce the leakag in the original
BlindSeer system. In particular, the above step 3 (i.e., the protocol execution
to check whether the given query is satisfied) is performed using secure
two-party computation based on Yao's garbled circuit. Thefore, it is critical
for the logic to be as simple as possible, in order to obtain efficiency. To
reduce the leakage, more sophisticaed cryptographic mechanisms need to be used,
but then the step 3 of secure computation will be significantly slower. 

However, recall that in the simplified construction suggested in
Section~\ref{sec:ext-blindseer} doesn't need to use secure computation due to
the simplication of the threat model. In this case, we can actually remove the
leakage by completely hiding $\HHH$. As one promising direction, we can achive
this goal by storing the encrypted BF with an ORAM technique. Note that the
$\HHH$ can be regarded as the metadata (i.e., ``addresses") for the encrypted
data (i.e., BF contents), which fits perfectly into the scneario that ORAM
tries to make secure. 
%
Even using the ORAM, we expect that the resulting system will be faster than
the system using two-party secure computation, which is known to be
significantly slower than ORAM. 


