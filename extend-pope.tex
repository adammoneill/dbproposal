%!TEX root = dbproposal.tex

\subsection{Extensions to Property-Preserving Encryption}
As stated above, the direct use of property-preserving encryption has a checked history with leakage attacks showing that deterministic and order-preserving encryption reveal the entire stored dataset for many applications (see work of the co-PI for an overview of leakage attacks~\cite{SP:FVYSHG17}).  However, PPE has many benefits that should not be overlooked by the research community.  PPE is easier to configure, use, deploy, does not require replacement of the entire software stack, and gives admins flexibility on the underlying database technology.  Given the momentum of PPE in the commercial sector, researchers should understand if a PPE solution can be used safely while retaining the benefits of the approach.  Our approach keeps the intact the design principal of PPE: the only place that the database should have to change is the comparison operator (equality, comparison, or distance).  The database should still be able use standard indexing mechanisms.  We note it is possible to override this operator to be interactive and require help from a client without altering the overall indexing structure.  Before describing our approach, we briefly describe prior work in the area.

\paragraph{Prior Work}
We are aware of two main works that follow the approach of PPE with added interactivity.  A co-PI recently introduced a new cryptographic approach, called POPE, to
supporting range queries over encrypted data, providing stronger security than
previous order-preserving encryptions~\cite{CCS:RACY16}.  In contrast to
existing OPE schemes, the server builds a novel indexing structure called a
POPE tree, in which each node has a {\it unsorted buffer} and a sorted list of
elements.  Thanks to this, the scheme can perform {\it lazy indexing, by
sorting values only when necessary}. In particular, on each range query the
scheme sorts the part of the data that is accessed during the search, leaving
much of the data untouched.

The other work is Arx due Poddar, Boelter, and Popa~\cite{EPRINT:PodBoePop16}.  The
Arx protocol builds an index for answering
range queries without revealing all order relationships to the
server. The index stores all encrypted values in a binary
tree so range queries can be answered by traversing this
tree for the end points. Using Yao's garbled circuits, the
server traverses the index without learning the values it is
comparing or the result of the comparison at each stage.  Since each garbled circuit can only be used once, as order operations are computed the client and server need to work together to prepare garbled circuits (and oblivious transfer).

\paragraph{POPE for Similarity of High-dimensional data}
Our main approach will be developing PPE that is just in time using advice from the client.  In the case, of POPE this took the form of asking the client to sort a small number of nodes to build out a tree and then comparing ciphertexts only with those nodes.  This limited leakage to comparison between the dataset and these nodes.  Roughly, this can be thought of as reducing leakage from a quadratic number of comparisons to a linear number.  Our first task will be extend the POPE paradigm to high dimensional data.  The approach uses random hyperplanes which have previously been used in the concept of locality-sensitive hashing~\cite{charikar2002similarity}.  The idea is as follows:

\begin{enumerate}
\item The server initially stores a unsorted buffer of the entire dataset.
\item The client initiates a search for items that are similar to $a$.
\item The servers asks the client to split the tree.
\item The client generates a random hyperplane $x$.  
\item The server asks the client to indicate $\sign(<x^T,b>)$ for each stored element $b$. In addition, the client indicates $\sign(<x^T,a>)$.
\item The client and server repeat the process with the relevant subtree.
\end{enumerate} 

This basic protocol above is not ready for deployment.  First, it assumes the creation of a binary tree rather than a B-tree with many children per tree node. The second problem is that random hyperplanes have a substantial error probability.  There will be many nodes $b$ such that $d(a, b)=close$ such that $a$ and $b$ lie on different sides of the selected hyperplane.  To deal with this problem, the client can select a collection of hyperplanes $x_1,..., x_k$ and split the tree based on the magnitude of $\sum_{i=1}^k \sign(<x_i^T, b>)$.  This technique allows us to control the probability that $a$ and $b$ will be denoted as far (lying in different subtrees) when they are close (using tail bounds for the binomial distribution).  Furthermore, it also creates a larger number of children for each node pushing from a binary to a B$^+$-tree.

\paragraph{POPE for substring search}
Our second task is to construct substring search with limited leakage.  Substring search has previously been addressed directly in work by Chase and Shen~\cite{chase2015substring} and Moataz and Blass~\cite{EPRINT:MoaBla15} and by augmented keyword equality~\cite{RSA:IKLO16} and Boolean formula search~\cite{CCS:JJKRS13}.  In this proposal, we focus on extending the approach of Chase and Shen.  Their approach was to build an encrypted suffix tree.  A suffix tree is a commonly used data structure for substring search~\cite{mccreight1976space}.  Our idea is to build the suffix tree just in time using ideas from the POPE protocol.  Essentially, the client will build a single level of the suffix tree.  When the client searches for a string $a$ they will traverse the partially constructed suffix tree with the client/server interactively building out the tree as necessary.  This approach will require augmentations to a traditional suffix tree as the original searchable string must be stored and queryable to build the tree on demand.  Balancing the speed and privacy of querying the original string represents an important tradeoff in this approach.