%!TEX root = dbproposal.tex

\subsection{Hybrid approache to improve accuracy in approximate DRE}

Even after our improvements above, the fully-interactive POPE and
BlindSeer schemes may still not be sufficiently scalable for the largest
databases and composable with existing softare. We not propose a novel,
\emph{hybrid approach} that uses interaction alongside the Approximate
Distance Revealing Encryption schemes we discussed in \Cref{sec:adre}.

The basic idea is as follows:
\begin{itemize}\setlength{\itemsep}{0em}
  \item User encrypts keys using a noisy Approximate DRE scheme and
  stores them on the server
  \item User queries for all points within distance $d$ of $a$ by
    sending an (approximate distance-revealing) encryption $a'$ of $a$ to the
    server, along with a scaling $d'$ of $d$ according to the scaling of
    the encryption scheme.
  \item The server finds a set $S$ of all ciphertexts within $d'$ of $a'$
    non-interactively
  \item The server enters into an interactive protocol with the user to
    discover and return
    the true set $S' \subseteq S$ of results within the true
    distance $d$ of the true query point $a$.
\end{itemize}

By using the approach of our enhanced POPE protocol within smaller
``bins'' of ciphertexts on the server side, this can effectively
improve on the performance and scalability of POPE by limiting the
interactive portion of the protocol to a subset $S$ of the entire
database. At the same time, it improves on the communication complexity
of Approximate DRE by limiting the final results to only those within
the \emph{actual} desired ball.

The privacy provided by this proposed approach is no worse than that
provided by POPE or by Approximate DRE separately, but the performance
is better than using either of those approaches alone. Alternatively,
the hybrid approach could allow a \emph{more noisy} approximate DRE
scheme to be used, as the results will always be trimmed down using an
efficient interactive protocol.

This idea is still preliminary and depends crucially on the success of
prior portions of the proposal, so we will be flexible to alter this
approach as necessary depending on the lessons learned in working on the
building blocks of the hybrid scheme.
