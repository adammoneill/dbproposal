\documentclass[11pt]{article}

\usepackage{fullpage,varwidth,url}
\usepackage{amssymb}
\usepackage{amsmath,amsthm}
\usepackage{relsize}
\usepackage{color}
%
%    \setlength{\evensidemargin}{-0.2in}
%    \setlength{\oddsidemargin}{-0.2in}
%    \setlength{\textwidth}{6.9in}
%    \setlength{\textheight}{9.5in}
%    \setlength{\topmargin}{-0.35in}
%    \setlength{\headheight}{0in}
%    \setlength{\headsep}{0in}
%    \setlength{\footskip}{0.5in}
\title{{{\Large{TWC: Small:  Next-Generation Secure Outsourced Databases}}}}
%\author{Sanjam Garg \\University of California, Berkeley \and 	Mohammad Mahmoody \\ University of Virginia\and Adam O'Neill \\Georgetown University}


%\usepackage{anysize} % to set up document margins.
%\marginsize{1.1in}{1.1in}{1.1in}{1.0in}
%\vspace{-1mm}
\usepackage{times}
\usepackage{etex}
\usepackage{framed}

\newcommand{\Att}{\textsc{Attack}}
\newcommand{\Hyb}{\textsc{Hybrid}}
\newcommand{\Idl}{\textsc{Ideal}}
\newcommand{\Lzy}{\textsc{Lazy}}


\newcommand{\OR}{\mathsf{OR}}
\newcommand{\AND}{\mathsf{AND}}


\newtheorem{thm}{Theorem}[section]
\newtheorem{lem}[thm]{Lemma}
\newtheorem{cor}[thm]{Corollary}
\newtheorem{propo}[thm]{Proposition}
\newtheorem{clm}[thm]{Claim}
\newtheorem{defn}[thm]{Definition}
\newtheorem{conj}[thm]{Conjecture}
\newtheorem{assm}[thm]{Assumption}
\newtheorem{rem}[thm]{Remark}
\newtheorem{obs}[thm]{Observation}
\newtheorem{egs}[thm]{Example}
\newtheorem{fct}[thm]{Fact}
\newtheorem{expr}{Experiment}
\newtheorem{cons}[thm]{Construction}
\newtheorem{nte}[thm]{Note}
\newtheorem{aim}[thm]{Aim}

% keys mathsf
%\newcommand{\sk}{\mathsf{sk}}
%\newcommand{\SK}{\mathsf{SK}}
%\newcommand{\pk}{\mathsf{pk}}
%\newcommand{\PK}{\mathsf{PK}}
%\newcommand{\dk}{\mathsf{dk}}
%\newcommand{\DK}{\mathsf{DK}}


%\newcommand{\att}{\cA}
%\newcommand{\prd}{\cP}


%%%%%%%%%%%%%%%%%%%%%%%%%%%%%%%%%%%%%%%%%%%%%%%%%%%55
\newcommand{\remove}[1]{}
\newcommand{\num}[1]{{\bf (#1)}}

\newcommand{\procfont}[1]{\mathsc{#1}}
\newcommand{\tablefont}[1]{\mathsf{#1}}

\newcommand{\mpk}{\pi}
\newcommand{\msk}{\varfont{msk}}
\newcommand{\usk}{\varfont{dk}}
\newcommand{\usknew}{\overline{\varfont{dk}}}
\newcommand{\ctxt}{\varfont{C}}
\newcommand{\ctxtnew}{\overline{\varfont{C}}}
\newcommand{\IBE}{\schemefont{IBE}}
\newcommand{\IBEnew}{\overline{\schemefont{IBE}}}
\newcommand{\mskspace}{{\cal S}}
%\newcommand{\kg}{{\cal K}}
\newcommand{\kg}{\mathsf{Kg}}
\newcommand{\kgnew}{\overline{{\cal K}}}
%\newcommand{\pg}{{\cal P}}
\newcommand{\pg}{\mathsf{Pg}}
\newcommand{\Enc}{\enc}
\newcommand{\Encnew}{\encnew}
\newcommand{\Dec}{\decc}
\newcommand{\Decnew}{\deccnew}

\def\vk{\mathit{vk}}
\def\pars{\pi}
\def\ek{\varfont{ek}}
\def\pk{\varfont{pk}}
\def\sk{\varfont{sk}}
\newcommand{\vecsk}{\mathbf{sk}}
\newcommand{\vecpk}{\mathbf{pk}}
\newcommand{\vecm}{\mathbf{m}}
\newcommand{\vecM}{\mathbf{M}}
\def\dk{\varfont{dk}}
\def\ak{\varfont{ak}}
\def\stateinfo{\varfont{st}}
\def\st{\varfont{state}}
\def\stt{\varfont{St}}
\newcommand{\RO}{H}
\newcommand{\keygen}{KG}
\newcommand{\pargen}{PG}
\newcommand{\trapdoor}{Td}
\newcommand{\test}{Test}
%\newcommand{\setup}{Setup}
\renewcommand{\AE}{\mathcal{AE}}
%\newcommand{\enc}{\mathcal{E}}
\newcommand{\enc}{\mathsf{Enc}}
%\newcommand{\decc}{\mathcal{D}}
\newcommand{\decc}{\mathsf{Dec}}
\newcommand{\eval}{\mathsf{Eval}}
\newcommand{\genc}{\algfont{\Enc}}
\newcommand{\dec}{\mathsf{Dec}}
\newcommand{\gdec}{\algfont{\Dec}}
%\newcommand{\com}{Com}
\newcommand{\ver}{Ver}
\newcommand{\open}{\ver}
\def\Hash{H}

\newcommand{\Funcs}{\mathsf{Funcs}}

\newcommand{\ol}{\overline}
\newcommand{\wt}[1]{\widetilde{#1}}
\newcommand{\wh}[1]{\widehat{#1}}
\newcommand{\nin}{\not \in}

\newcommand{\es}{\varnothing} % empty set
\newcommand{\se}{\subseteq}
\newcommand{\sm}{\setminus}
\newcommand{\nse}{\not \se}
\newcommand{\nf}[2]{\nicefrac{#1}{#2}}
\newcommand{\rv}[1]{\mathbf{#1}} % Random Variable
\newcommand{\mal}[1]{\widehat{#1}} % Malicious version of the alg

\newcommand{\schemefont}[1]{{\mathsf{#1}}}
\newcommand{\primfont}[1]{{\mathsf{#1}}}
\newcommand{\algfont}[1]{{\mathsf{#1}}}
\newcommand{\advfont}[1]{{\mathsf{#1}}}
\newcommand{\orfont}[1]{\mathsc{#1}}
\newcommand{\varfont}[1]{\mathit{#1}}
\newcommand{\randvarfont}[1]{\mathbf{#1}}
\newcommand{\constfont}[1]{\mathtt{#1}}
\newcommand{\notionfont}[1]{\mathrm{#1}}
\newcommand{\eventfont}[1]{{\mathsc{#1}}}
\newcommand{\vectorfont}[1]{\mathslbf{#1}}
\newcommand{\transformfont}[1]{\mathsf{#1}}


\newcommand{\tuple}[1]{\langle{#1}\rangle}
\def\from{\mbox{from}\ }
\def\From{\mbox{From}\ }
\def\bits{\{0,1\}}
\def\cross{\times}
\newcommand{\xor}{{\oplus}}
\newcommand{\Colon}{{:\;\;}}
\newcommand{\mystrut}{\rule{0em}{15pt}}
\newcommand{\namestrut}{\rule{0em}{20pt}}
\def\poly{\mathop{\rm poly}\nolimits}
\def\emptystring{\varepsilon}
\def\emptyid{()}
\newcommand{\Dom}{\mathsf{Dom}}
\newcommand{\DomR}{\mathsf{DomR}}
\newcommand{\Rng}{\mathsf{Rng}}
\newcommand{\RngR}{\mathsf{RngR}}
\newcommand{\Keys}{\mathsf{Keys}}
\newcommand{\calA}{{\cal A}}
\newcommand{\calB}{{\cal B}}
\newcommand{\calC}{{\cal C}}
\newcommand{\calF}{{\cal F}}
\newcommand{\calL}{{\cal L}}
\newcommand{\calM}{{\cal M}}
\newcommand{\calO}{{\cal O}}
\newcommand{\calR}{{\cal R}}
\newcommand{\calH}{{\cal H}}
\newcommand{\calG}{{\cal G}}
\newcommand{\calD}{{\cal D}}
\newcommand{\calE}{{\cal E}}
\newcommand{\calP}{{\cal P}}
\newcommand{\calS}{{\cal S}}
\newcommand{\calU}{{\cal U}}
\newcommand{\calX}{{\cal X}}
\newcommand{\calY}{{\cal Y}}
\newcommand{\calEE}{{\cal EE}}
\newcommand{\N}{{{\mathbb N}}}
\newcommand{\Z}{{{\mathbb Z}}}
\newcommand{\R}{{{\mathbb R}}}
\newcommand{\goesto}{{\rightarrow}}
\newcommand{\eqdef}{\;\stackrel{\rm def}{=}\;}
\newcommand{\then}{{\;;\;\;}}   % for [ ; ; ; : ] notation
\newcommand{\andthen}{{\;:\;\;}}
\def\union{\cup}
\def\bigunion{\bigcup}
\def\intersection{\cap}
\def\bigintersection{\bigcap}
\def\suchthatt{\: :\:}
\newcommand{\suchthat}{{\mbox{s.t.\ }}}
\def\next{\:;\:}
\def\nextt{\:;\:}
\newcommand{\sett}[1]{\{#1\}}
\newcommand{\set}[2]{\{\:#1 \suchthatt #2\:\}}
\newcommand{\setsize}[1]{\left|{#1}\right|}
\def\leqq{\;\leq\;}
\def\eqq{\;=\;}
\def\geqq{\;\geq\;}
\def\equivv{\;\equiv\;}
\def\prn#1{\left(#1\right)}
\newcommand{\getsr}{{\:{\leftarrow{\hspace*{-3pt}\raisebox{.75pt}{$\scriptscriptstyle\$$}}}\:}}
% \def\getsr{\stackrel{{\scriptscriptstyle\$}}{\leftarrow}}
% \renewcommand{\choose}[2]{{{#1}\atopwithdelims(){#2}}}

\newcommand{\Var}{{\mbox{\bf Var}}}
\newcommand{\E}{\mathbf{E}}
\newcommand{\EE}[1]{{\E\left[{#1}\right]}}
\newcommand{\EEE}[2]{{\E_{#1}\left[{#2}\right]}}
\newcommand{\Prob}[1]{{\Pr\left[\,{#1}\,\right]}}
\newcommand{\probb}[2]{{\Pr}_{#1}\left[\,{#2}\,\right]}
\newcommand{\probbs}[3]{{\Pr}^{#1}_{#2}\left[\,{#3}\,\right]}
\newcommand{\probbp}[2]{{\Pr}_{#1}'\left[\,{#2}\,\right]}
\newcommand{\Probb}[2]{{{\Pr}_{#1}\left[\,{#2}\,\right]}}
\newcommand{\Probbp}[2]{{{\Pr}_{#1}'\left[\,{#2}\,\right]}}
\newcommand{\condProb}[2]{{\Pr}\left[\,#1\,|\,#2\,\right]}
\newcommand{\CondProb}[2]{{\Pr}\left[\: #1\:\left|\right.\:#2\:\right]}
\newcommand{\CondProbb}[3]{{\Pr}_{#1}\left[\: #2\:\left|\right.\:#3\:\right]}
\newcommand{\CondProbbp}[3]{%
{\Pr}_{#1}^{'}\left[\: #2\:\left|\right.\:#3\:\right]}


% ===================================================================
\newcommand{\kwfont}[1]{{\ensuremath{\mathrm{#1}}}}
\newcommand{\kwfunction}{{\kwfont{function}\ }}
\newcommand{\kwlabel}{{\kwfont{label}\ }}
\newcommand{\kwfor}{{\kwfont{for}\ }}
\newcommand{\kwand}{{\kwfont{and}\ }}
\newcommand{\kwor}{{\kwfont{or}\ }}
\newcommand{\kwnot}{{\kwfont{not}}}
\newcommand{\kwdo}{{\kwfont{do}\ }}
\newcommand{\kwreturn}{{\kwfont{return}\ }}
\newcommand{\kwReturn}{{\kwfont{Return}\ }}
\newcommand{\kwalgorithm}{{\ensuremath{\mathbf{Algorithm}\ }}}
\newcommand{\kwprotocol}{{\kwfont{Protocol}\ }}
\newcommand{\kwexperiment}{{\kwfont{Experiment}\ }}
\newcommand{\kwadversary}{{\kwfont{Adversary}\ }}
\newcommand{\kworacle}{{\kwfont{Oracle}\ }}
\newcommand{\kwuntil}{{\kwfont{until}\ }}
\newcommand{\kwrepeat}{{\kwfont{repeat}\ }}
\newcommand{\kwif}{{\kwfont{If}\ }}
\newcommand{\kwthen}{{\kwfont{Then}\ }}
\newcommand{\kwelse}{{\kwfont{Else}\ }}
\newcommand{\kwabort}{{\kwfont{abort}\ }}
\newcommand{\kwgoto}{{\kwfont{goto}\ }}
\newcommand{\kwwhile}{{\kwfont{while}\ }}
\newcommand{\kwparse}{{\kwfont{parse}\ }}
\newcommand{\kwas}{{\kwfont{as}\ }}
\newcommand{\kwstatic}{{\kwfont{static}\ }}
\newcommand{\kwrun}{{\kwfont{run}\ }}
\newcommand{\kwbegin}{{\kwfont{begin}\ }}
\newcommand{\kwend}{{\kwfont{end}\ }}
\newcommand{\kwstart}{{\kwfont{start}\ }}
\newcommand{\kwcontinue}{{\kwfont{continue}\ }}
\newcommand{\kwdefine}{{\kwfont{define}\ }}
\newcommand{\kwflip}{{\kwfont{flip}\ }}
\newcommand{\kwlet}{{\kwfont{let}\ }}
\newcommand{\kwof}{{\kwfont{of}\ }}
\newcommand{\kwcase}{{\kwfont{case}\ }}
\newcommand{\kwswitch}{{\kwfont{switch}\ }}
\newcommand{\kwpick}{{\kwfont{pick}\ }}
\newcommand{\kwset}{{\kwfont{set}\ }}
\newcommand{\kwcompute}{{\kwfont{compute}\ }}
\newcommand{\comment}[1]{\hspace{15pt}{\small /$\!\!$/\ #1}}
\newcommand{\Comment}[1]{\hspace{5pt}{/$\!\!$/\ #1}}
\newcommand{\sComment}[1]{\hspace{2pt}{/$\!\!$/#1}}

\newcommand{\Coins}{\mathsf{Coins}}

\newcommand{\ProbExp}[2]{{\Pr}\left[\: #1\:\suchthatt\:#2\:\right]}

\newcommand{\PRG}{\schemefont{PRG}}
\newcommand{\PRGpg}{\schalg{PRG}{Ev}}
\newcommand{\PRGkg}{\schalg{PRG}{Kg}}

\newcommand{\FSE}{\schemefont{FSE}}

\newcommand{\PE}{\schemefont{PE}}
\newcommand{\FE}{\schemefont{FE}}
\newcommand{\setup}{\mathsf{Setup}}
\newcommand{\keyder}{\mathsf{KeyDer}}

\newcommand{\OT}{\schemefont{OT}}
\newcommand{\Send}{\procfont{Sender}}
\newcommand{\Rec}{\procfont{Receiver}}
\newcommand{\PKE}{\schemefont{PKE}}
\newcommand{\FPKE}{\schemefont{FPKE}}
\newcommand{\DPKE}{\schemefont{DPKE}}
\newcommand{\DE}{\schemefont{DE}}
\newcommand{\PKEpg}{\schalg{PKE}{Pg}}
\newcommand{\oPKEpg}{\overline{{\cal P}}}
\newcommand{\PKEEnc}{\schalg{PKE}{Enc}}
\newcommand{\oPKEEnc}{\schalg{\overline{PKE}}{Enc}}
\newcommand{\PKEDec}{\schalg{PKE}{Dec}}
\newcommand{\oPKEDec}{\overline{{\cal D}}}
\newcommand{\PKEkg}{\schalg{PKE}{Kg}}
\newcommand{\oPKEkg}{\overline{{\cal K}}}
\newcommand{\PKEpk}{\mathit{ek}}
\newcommand{\PKEPK}{\schalg{PKE}{PK}}
\newcommand{\PKESK}{\schalg{PKE}{SK}}
\newcommand{\oPKEpk}{\overline{\mathit{ek}}}
\newcommand{\PKEsk}{\mathit{dk}}
\newcommand{\oPKEsk}{\overline{\mathit{dk}}}
\newcommand{\PKEpars}{\pi}
\newcommand{\oPKEpars}{\overline{\pi}}

\newcommand{\SE}{\schemefont{SE}}
\newcommand{\SEpg}{{\cal P}}
\newcommand{\SEEnc}{{\cal E}}
\newcommand{\SEDec}{{\cal D}}
\newcommand{\SEkg}{{\cal K}}
\newcommand{\SEkey}{K}
\newcommand{\SEpars}{\pi}

%%%  English %%%%%%%%%%%%%%%%%%%%%%%%%%%%%%%%%%%%%%%%%%%%%%%%%%%%%%

\newcommand{\etal}{et~al.\ }
\newcommand{\aka}{also known as,\ }
\newcommand{\resp}{resp.,\ }
\newcommand{\ie}{i.e.,\ }
\newcommand{\wolog}{w.l.o.g.\ }
\newcommand{\Wolog}{W.l.o.g.\ }
\newcommand{\eg}{e.g.,\ }
\newcommand{\Eg}{E.g.,\ }
\newcommand{\wrt} {with respect to\ }
\newcommand{\cf}{{cf.,\ }}

%%%  math %%%%%%%%%%%%%%%%%%%%%%%%%%%%%%%%%%%%%%%%%%%%%%%%%%%%%%

\newcommand{\round}[1]{\lfloor #1 \rceil}
\newcommand{\ceil}[1]{\lceil #1 \rceil}
\newcommand{\floor}[1]{\lfloor #1 \rfloor}
\newcommand{\angles}[1]{\langle #1 \rangle}
\newcommand{\parens}[1]{( #1 )}
\newcommand{\bracks}[1]{[ #1 ]}
\newcommand{\bra}[1]{\langle#1\rvert}
\newcommand{\ket}[1]{\lvert#1\rangle}


\newcommand{\adjRound}[1]{\left\lfloor #1 \right\rceil} % Adjusted Round
\newcommand{\adjCeil}[1]{\left\lceil #1 \right\rceil}
\newcommand{\adjFloor}[1]{\left\lfloor #1 \right\rfloor}
\newcommand{\adjAngles}[1]{\left\langle #1 \right\rangle}
\newcommand{\adjParens}[1]{\left( #1 \right)}
\newcommand{\adjBracks}[1]{\left[ #1 \right]}
\newcommand{\adjBra}[1]{\left\langle#1\right\rvert}
\newcommand{\adjKet}[1]{\left\lvert#1\right\rangle}
\newcommand{\adjSet}[1]{\left\{ #1 \right\}}
\newcommand{\half}{\tfrac{1}{2}}
\newcommand{\third}{\tfrac{1}{3}}
\newcommand{\quarter}{\tfrac{1}{4}}
%\newcommand{\eqdef}{:=}
%\newcommand{\zo}{\{0,1\}}




\newcommand{\cA}{{\mathcal A}}
\newcommand{\cB}{{\mathcal B}}
\newcommand{\cC}{{\mathcal C}}
\newcommand{\cD}{{\mathcal D}}
\newcommand{\cE}{{\mathcal E}}
\newcommand{\cF}{{\mathcal F}}
\newcommand{\cG}{{\mathcal G}}
\newcommand{\cH}{{\mathcal H}}
\newcommand{\cI}{{\mathcal I}}
\newcommand{\cJ}{{\mathcal J}}
\newcommand{\cK}{{\mathcal K}}
\newcommand{\cL}{{\mathcal L}}
\newcommand{\cM}{{\mathcal M}}
\newcommand{\cN}{{\mathcal N}}
\newcommand{\cO}{{\mathcal O}}
\newcommand{\cP}{{\mathcal P}}
\newcommand{\cQ}{{\mathcal Q}}
\newcommand{\cR}{{\mathcal R}}
\newcommand{\cS}{{\mathcal S}}
\newcommand{\cT}{{\mathcal T}}
\newcommand{\cU}{{\mathcal U}}
\newcommand{\cV}{{\mathcal V}}
\newcommand{\cW}{{\mathcal W}}
\newcommand{\cX}{{\mathcal X}}
\newcommand{\cY}{{\mathcal Y}}
\newcommand{\cZ}{{\mathcal Z}}

\newcommand{\bfA}{\mathbf{A}}
\newcommand{\bfB}{\mathbf{B}}
\newcommand{\bfC}{\mathbf{C}}
\newcommand{\bfD}{\mathbf{D}}
\newcommand{\bfE}{\mathbf{E}}
\newcommand{\bfF}{\mathbf{F}}
\newcommand{\bfG}{\mathbf{G}}
\newcommand{\bfH}{\mathbf{H}}
\newcommand{\bfI}{\mathbf{I}}
\newcommand{\bfJ}{\mathbf{J}}
\newcommand{\bfK}{\mathbf{K}}
\newcommand{\bfL}{\mathbf{L}}
\newcommand{\bfM}{\mathbf{M}}
\newcommand{\bfN}{\mathbf{N}}
\newcommand{\bfO}{\mathbf{O}}
\newcommand{\bfP}{\mathbf{P}}
\newcommand{\bfQ}{\mathbf{Q}}
\newcommand{\bfR}{\mathbf{R}}
\newcommand{\bfS}{\mathbf{S}}
\newcommand{\bfT}{\mathbf{T}}
\newcommand{\bfU}{\mathbf{U}}
\newcommand{\bfV}{\mathbf{V}}
\newcommand{\bfW}{\mathbf{W}}
\newcommand{\bfX}{\mathbf{X}}
\newcommand{\bfY}{\mathbf{Y}}
\newcommand{\bfZ}{\mathbf{Z}}


\newcommand{\bfa}{\mathbf{a}}
\newcommand{\bfb}{\mathbf{b}}
\newcommand{\bfc}{\mathbf{c}}
\newcommand{\bfd}{\mathbf{d}}
\newcommand{\bfe}{\mathbf{e}}
\newcommand{\bff}{\mathbf{f}}
\newcommand{\bfg}{\mathbf{g}}
\newcommand{\bfh}{\mathbf{h}}
\newcommand{\bfi}{\mathbf{i}}
\newcommand{\bfj}{\mathbf{j}}
\newcommand{\bfk}{\mathbf{k}}
\newcommand{\bfl}{\mathbf{l}}
\newcommand{\bfm}{\mathbf{m}}
\newcommand{\bfn}{\mathbf{n}}
\newcommand{\bfo}{\mathbf{o}}
\newcommand{\bfp}{\mathbf{p}}
\newcommand{\bfq}{\mathbf{q}}
\newcommand{\bfr}{\mathbf{r}}
\newcommand{\bfs}{\mathbf{s}}
\newcommand{\bft}{\mathbf{t}}
\newcommand{\bfu}{\mathbf{u}}
\newcommand{\bfv}{\mathbf{v}}
\newcommand{\bfw}{\mathbf{w}}
\newcommand{\bfx}{\mathbf{x}}
\newcommand{\bfy}{\mathbf{y}}
\newcommand{\bfz}{\mathbf{z}}



\newcommand{\sfA}{\mathsf{A}}
\newcommand{\sfB}{\mathsf{B}}
\newcommand{\sfC}{\mathsf{C}}
\newcommand{\sfD}{\mathsf{D}}
\newcommand{\sfE}{\mathsf{E}}
\newcommand{\sfF}{\mathsf{F}}
\newcommand{\sfG}{\mathsf{G}}
\newcommand{\sfH}{\mathsf{H}}
\newcommand{\sfI}{\mathsf{I}}
\newcommand{\sfJ}{\mathsf{J}}
\newcommand{\sfK}{\mathsf{K}}
\newcommand{\sfL}{\mathsf{L}}
\newcommand{\sfM}{\mathsf{M}}
\newcommand{\sfN}{\mathsf{N}}
\newcommand{\sfO}{\mathsf{O}}
\newcommand{\sfP}{\mathsf{P}}
\newcommand{\sfQ}{\mathsf{Q}}
\newcommand{\sfR}{\mathsf{R}}
\newcommand{\sfS}{\mathsf{S}}
\newcommand{\sfT}{\mathsf{T}}
\newcommand{\sfU}{\mathsf{U}}
\newcommand{\sfV}{\mathsf{V}}
\newcommand{\sfW}{\mathsf{W}}
\newcommand{\sfX}{\mathsf{X}}
\newcommand{\sfY}{\mathsf{Y}}
\newcommand{\sfZ}{\mathsf{Z}}

\newcommand{\sfa}{\mathsf{a}}
\newcommand{\sfb}{\mathsf{b}}
\newcommand{\sfc}{\mathsf{c}}
\newcommand{\sfd}{\mathsf{d}}
\newcommand{\sfe}{\mathsf{e}}
\newcommand{\sff}{\mathsf{f}}
\newcommand{\sfg}{\mathsf{g}}
\newcommand{\sfh}{\mathsf{h}}
\newcommand{\sfi}{\mathsf{i}}
\newcommand{\sfj}{\mathsf{j}}
\newcommand{\sfk}{\mathsf{k}}
\newcommand{\sfl}{\mathsf{l}}
\newcommand{\sfm}{\mathsf{m}}
\newcommand{\sfn}{\mathsf{n}}
\newcommand{\sfo}{\mathsf{o}}
\newcommand{\sfp}{\mathsf{p}}
\newcommand{\sfq}{\mathsf{q}}
\newcommand{\sfr}{\mathsf{r}}
\newcommand{\sfs}{\mathsf{s}}
\newcommand{\sft}{\mathsf{t}}
\newcommand{\sfu}{\mathsf{u}}
\newcommand{\sfv}{\mathsf{v}}
\newcommand{\sfw}{\mathsf{w}}
\newcommand{\sfx}{\mathsf{x}}
\newcommand{\sfy}{\mathsf{y}}
\newcommand{\sfz}{\mathsf{z}}

\newcommand{\eps}{\epsilon}
%\newcommand{\e}{\epsilon}
\newcommand{\veps}{\varepsilon}
%\newcommand{\vare}{\varepsilon}
\newcommand{\vphi}{\varphi}
\newcommand{\vsigma}{\varsigma}
\newcommand{\vrho}{\varrho}
\newcommand{\vpi}{\varpi}
\newcommand{\tO}{\widetilde{O}}
\newcommand{\tOmega}{\widetilde{\Omega}}
\newcommand{\tTheta}{\widetilde{\Theta}}
\newcommand{\field}{\ensuremath \Bbbk}
\newcommand{\reals}{\ensuremath \mathbb{R}}

\newcommand{\client}{\mathsl{client}}
\newcommand{\server}{\mathsl{server}}
\newcommand{\TTP}{\procfont{TTP}}
\newcommand{\rec}{\mathsf{rec}}
\newcommand{\doc}{\mathsf{doc}}
\newcommand{\pred}{\mathsf{pred}}
\newcommand{\scrub}{\mathsf{scrub}}

\newcommand{\ODS}{\schemefont{ODS}}
\newcommand{\commit}{\mathsf{Commit}}
\newcommand{\query}{\mathsf{Query}}
\newcommand{\DB}{\mathsf{DB}}


\newcommand{\sanjam}[1]{\textcolor{red}{Sanjam: #1}}
\newcommand{\moh}[1]{}%{\textcolor{red}{Moh: #1}}
\newcommand{\adam}[1]{\textcolor{red}{Adam: #1}}

%---
\newtheorem{definition}{Definition}[section]
\newtheorem{question}{Question}[section]
\newtheorem{fact}{Fact}[section]
\newtheorem{lemma}{Lemma}[section]
\newtheorem{corollary}{Corollary}[section]
\newtheorem{theorem}{Theorem}[section]
\newtheorem{claim}{Claim}[section]
\newtheorem{assumption}{Assumption}[section]
\newtheorem{proposition}{Proposition}[section]
\newtheorem{hypothesis}{Hypothesis}[section]
\newtheorem{observation}{Observation}[section]
\theoremstyle{remark}
\newtheorem{construction}{Construction}[section]
\newtheorem{remark}{Remark}[section]
\newtheorem{example}{Example}[section]

\newcommand{\pone}{\mbox{$P_1$}}
\newcommand{\ptwo}{\mbox{$P_2$}}

\renewcommand{\S}{{\cal S}}
\newcommand{\F}{{\cal F}}
\newcommand{\G}{{\mathbb{G}}}

\newcommand{\vecx}{{\mathbf{x}}}
\newcommand{\vecy}{{\mathbf{y}}}

\newcommand{\numcorr}{{\mathsf{ncorr}}}
\newcommand{\numencq}{{\mathsf{nencq}}}
\newcommand{\numkeyq}{{\mathsf{nkeyq}}}


\renewenvironment{proof}{\noindent{\bf Proof:~~}}{\qed}
\newcommand{\BPF}{\begin{proof}} \newcommand {\EPF}{\end{proof}}
\newenvironment{proofsketch}{\noindent{\bf Proof Sketch:~~}}{\qed}
\newcommand{\BPFS}{\begin{proofsketch}} \newcommand {\EPFS}{\end{proofsketch}}

%\def\qed{\quad\blackslug\lower 8.5pt\null\par}


\newcommand{\BPR}{\begin{myprotocol}}   \newcommand{\EPR}{\end{myprotocol}}
\newcommand{\ourfigg}[5]{
{\begin{figure}[#4]
\begin{center}
\framebox[\width][c]{
    \small
    \hbox{\quad
    \begin{varwidth}[c]{0.9\textwidth}
    %\begin{center}
    \begin{myfigure}
    [#1]
    \label{#2}
    \end{myfigure}
    %\end{center}
    \vspace{-3ex}
    #5
    \end{varwidth}
    \quad}
    }
    \begin{center} #3 \end{center}
    \vspace{-6ex}
\end{center}
\end{figure}
} }


% USAGE: \ourfig{TITLE}{LABEL}{CAPTION}{BODY}
\newcommand{\ourfig}[4]{\ourfigg{#1}{#2}{#3}{htb}{#4}}

\newcommand{\prott}[5]{
{\begin{figure}[#4]
\begin{center}
\framebox[\width][c]{
    \small
    \hbox{\quad
    \begin{varwidth}[c]{0.9\textwidth}
    %\begin{center}
    \begin{myprotocol}
    [#1]
    \label{#2}
    \end{myprotocol}
    %\end{center}
    \vspace{-3ex}
    #5
    \end{varwidth}
    \quad}
    }
    \begin{center} #3 \end{center}
    \vspace{-6ex}
\end{center}
\end{figure}
} }

% USAGE: \prot{TITLE}{LABEL}{CAPTION}{BODY}
\newcommand{\prot}[4]{\prott{#1}{#2}{#3}{htb}{#4}}

\newcommand{\view}{{\sf view}}
\newcommand{\trans}{{\sf trans}}
\newcommand{\com}{Z}
\newcommand{\cecom}{{\sf CECom}}
\newcommand{\mxcom}{{\sf MXCom}}
\newcommand{\mxzk}{{\sf MXZK}}
\newcommand{\nmmxcom}{{\sf NMMXCom}}
\newcommand{\gen}{{\sf Gen}}


\newcommand{\crsgen}{{\sf CRSGen}}
\newcommand{\crs}{{\sf CRS}}


\newcommand{\Commit}{{\sf Com}}


\newcommand{\scheme} {{\mathcal{S}}}

\newcommand{\siggen} {{\sf SigKeyGen}}
\newcommand{\sig} {{\sf Sig}}

%\newcommand{\size} {{\sf SIZE}}
\newcommand{\state} {\mathrm{state}}

\newcommand{\idx} {\mathrm{index}}

\newcommand{\kdm}{{\scriptscriptstyle\mathrm{KDM}}}
\newcommand{\skdm}{{\scriptscriptstyle\mathrm{SKDM}}}
\newcommand{\cpa}{{\scriptscriptstyle\mathrm{CPA}}}
\newcommand{\new}{{\scriptscriptstyle\mathrm{NEW}}}


\newcommand{\rsetup}[1]{\R({\sf setup}, #1 ) }
\newcommand{\rquery}[1]{\R({\sf query}, #1 ) }
\newcommand{\rchall}[1]{\R({\sf challenge}, #1 ) }
\newcommand{\rfinal}[1]{\R({\sf final}, #1 ) }
\newcommand{\iO}{i\mathcal{O}}

\newcommand{\PRF}{{\mathsf{F}}}
\newcommand{\PRFGen}{\mathsf{PRF{.}Gen}}
\newcommand{\PRFPunc}{\mathsf{PRF{.}Punc}}
\newcommand{\seed}{\ensuremath{{K}}}
\newcommand{\secpar}{\secparam}
\newcommand{\Adv}{\mathcal{A}}
\newcommand{\rsample}{\gets}

\newcommand{\tf}{\mathrm{tf}}
\newcommand{\idf}{\mathrm{idf}}
\newcommand{\df}{\mathrm{df}}
\newcommand{\p}{\mathrm{P}}



\date{}
\begin{document}

\maketitle
%\vspace{-20mm}
%\begin{abstract}
%ent results have resolved functional encryption for all circuits. Does this solve most problems for functional encryption. We think not. In this paper we present the future vision for functional encryption. Our vision for functional encryption is to realize encryption systems which (at least asymptotically) approach the efficiency levels approached in insecure solutions.
%\end{abstract}
\vspace{-20mm}

\section{Introduction}

\subsection{Background and Motivation} 

In modern society, data is often stored at rest for later use than in transit.  Indeed, databases are pervasive and the task of organizing and maintaining data is often off-loaded to an external provider. However, this means the external provider must be trusted with confidentiality of the data.  To mitigate this, an attractive solution is for a client to encrypt its data, under a key only it knows, before sending it to the external provider.  However, standard encryption hides all partial information about the data and does not allow efficient query processing.  Recent work in the cryptographic community has tried to address this problem by providing specialized encryption schemes and other secure database outsourcing protocols that both provide confidentiality to the extent possible and allow efficient query processing.  \textbf{However, this work has so far been mainly limited to very basic SQL functionality, which is outdated}.  The goal of this proposal is to bring us beyond this basic functionality to ``next-generation'' functionality that  practitioners expect.  

\subsection{This Proposal} \label{subsec-prop}

The goal of this proposal is create new secure outsourced database protocols that are more in line with the real-world functionality practitioners expect.  Real-world databases are often NoSQL or NewSQL and store spatial, time-series, or graph data.  We focus on extending two key approaches to secure outsourced databases to these cases: Protocols based on \emph{property-preserving encryption} like CryptDB, and protocols based on \emph{secure multiparty computation} like BlindSeer.  Finally, we plan to extend recent leakage-abuse attacks to our new protocols and propose novel mitigations.

\paragraph*{Proposed contributions in brief.}
In sum, the main proposed contributions are:
\begin{itemize}
\item Design advanced types of property-preserving and partially-property preserving encryption schemes for use on spacial, time series, and graph data.
\item Design extensions of BlindSeer to handle such data.
\item Extend recent attacks on databases that use property-preserving encryption and propose novel mitigations of these attacks. 
\end{itemize}

\paragraph*{Intellectual merit and research team.}  

 \paragraph*{Broader Impacts.}

\subsection{Related Work}

We do not address approaches based on searchable symmetric encryption or functional encryption.  

\section{Component A:  Approaches Based On Property-Preserving Encryption}


\subsection{Component A-I:  Property-Preserving Encryption for Spatial Databases}

In spatial databases, nearest neighbor queries (\emph{e.g.}, finding the closest soldier in the field) and clustering queries are pervasive.  Algorithms for executing these query types need to perform the fundamental operation of \emph{distance comparison} between points in the database, \emph{i.e.}, determining which of two candidates points are closer to a target point.  We call encryption that supports distance comparison \emph{distance-comparison revealing} encryption (DCRE).  Following the literature on order-revealing vs.~order-preserving encryption, we call the special case where ciphertexts themselves are spatial points \emph{distance-comparison preserving enryption} (DCPE).  We propose to study DCRE and DCPE analogously to the on-going study of order-revealing and order-preserving encryption.  
This leads to the following questions:

\begin{question}
Can we design efficient DCPE?  What security can be achieved by such schemes?
\end{question}


\begin{question}
Can we design efficient DCRE with better security?
\end{question}

\begin{question}
Can we design ``partial'' DCPE with better security?
\end{question}

To answer the first question, in on-going work we have found that distance comparison preserving functions do not seem to have a nice ``recursive'' property as in the case of order-preserving functions, which was crucially exploited by~\cite{EC:BCLO09}.  However, based on computer experiments, we conjecture that \emph{distance-comparison preserving functions are approximately distance-preserving}.    So far, we have proven this conjecture in one dimension.

%On the one hand, this would be depressing because it would mean that DCPE may not be able to provide much better security than \emph{distance-preserving} encryption, which seeks weak. 
If this conjecture is true, then for the first question we could equivalently turn our attention to the design and analysis of  an \emph{approximately distance-preserving} encryption scheme. 
Distance-preserving functions are easy to characterize geometrically, in terms of a scaling factor plus flips, rotations and reflections.  To approximately preserve distance, we can also  ``perturb'' each image point within a ball of given radius.  
We can show that independent random such perturbations yields a function that, while not strictly DCP, is \emph{approximately} so, and that encrypting via an approximately DCP function still guarantees accuracy of nearest neighbor  search within the approximation.  Moreover, as such perturbations can easily be derandomized, this gives an efficient \emph{approximate} DCPE scheme from PRFs.  Finally, the we plansto conduct a separate analysis in the spirit of~\cite{C:BolCheOne11} to answer the question of what privacy such a scheme provides.  

\subsection{Component A-II:  Property-Preserving Encryption for Time Series Data} 

In time series data one is often interested in correlations and anomalies.  Accordingly, we propose to look at \emph{correlation revealing and preserving encryption}  (CRE and CPE) and \emph{anomaly revealing and preserving encryption} (ARE and APE).  In other words, in the ``preserving'' case we are interested in perturbing statistical data in a way that preserves statistics or the fact that a point is an anomaly.  

Regarding correlation revealing schemes
\begin{question}
Can we design efficient CPE for correlations of interest?  Which correlations should we target?  What security can be achieved by such schemes?
\end{question}


\begin{question}
Can we design efficient CRE with better security?
\end{question}

\begin{question}
Can we design ``partial'' CPE with better security?
\end{question}


Regarding anomaly  evealing schemes
\begin{question}
Can we design efficient APE?  How should anomaly thresholds be set?  What security can be achieved by such schemes?
\end{question}


\begin{question}
Can we design efficient ARE with better security?
\end{question}

\begin{question}
Can we design ``partial'' APE with better security?
\end{question}

\subsection{Component A-III:  Property-Preserving Encryption for Graph Data} 

For graph data, shortest path and disease propagation queries are common.  Previous work looks at supporting shortest path queries in the context of searchable symmetric encryption.  Accordingly, we propose to investigate \emph{shortest path revealing  and preserving encryption} (SPRE and SPPE) and \emph{random walk revealing and preserving encryption} (RWRE and RWPE).   We believe that the study of such encryption schemes will lead to interesting questions in graph theory.  We will particularly try to leverage work on differentially private graph sanitization. 

Regarding shortest path revealing schemes: 

\begin{question}
Can we design efficient SPPE?  What security can be achieved by such schemes?
\end{question}


\begin{question}
Can we design efficient SPRE with better security?
\end{question}

\begin{question}
Can we design ``partial'' SPPE with better security?
\end{question}

And then regarding random walk preserving schemes:

\begin{question}
Can we design efficient RWPE?  What security can be achieved by such schemes?
\end{question}


\begin{question}
Can we design efficient RWRE with better security?
\end{question}

\begin{question}
Can we design ``partial'' RWPE with better security?
\end{question}


 \section{Component B: Approaches Based on Secure Multiparty Computation}
 %We note that to the best of our knowledge, no prior protocols supporting search on encrypted data support (efficient) relevance ranking. 

\subsection{Component B-I:  }

\subsection{Component B-II:  }


\section{Component C: Leakage-Abuse Attacks and Defenses}

\subsection{Component C-I: Extensions to Recent Attacks}

\subsection{Component C-II:  Attack Mitigations}

\paragraph*{Modularity.}  One mitigation of the attacks we propose to look at is \emph{modularity} in the sense of modular order-preserving encryption (M-OPE)~\cite{C:BolCheOne11}.  The idea of M-OPE is to apply a secret random offset modulo the largest possible message to a message before encrypting it (the secret random offset is chosen once and fixed in the secret key), so that everything gets ``shifted.''  Although similar attacks seem to apply to M-OPE, we propose to investigate more fine-grained modularity as a defense.  In the specific case of order-preserving encryption, we propose \emph{digit-modular} OPE (DM-OPE) where there is a secret modular offset applied to each digit.  The base in which the data is written could even itself be secret.  It becomes more complicated to make range queries with DM-OPE, as it requires an exponential number of queries in the number of ``wrap around'' digits in the query.  However, we propose to investigate approximating the queries efficiently (with some false positives that the client can filter out).

\section{Prior Accomplishments and NSF Support}

\textbf{Adam O'Neill:}
In his Ph.D.~work, the PI  developed the notions of deterministic encryption~\cite{C:BelBolOne07,ABO07,C:BolFehOne08,C:BFOR08,TCC:FulNeiRey12} and order-preserving encryption~\cite{EC:BCLO09,C:BolCheOne11} to help enable search on encrypted data with processing time comparable to that for unencrypted data, while providing as-strong-as-possible security guarantees subject to this constraint.
The PI has also worked on  instantiating random oracles~\cite{C:KilOneSmi10,TCC:GoyONeRao11,EC:LewONeSmi13},   aggregate signatures~\cite{CCS:BGOY07,AC:GLOW12}, deniable encryption~\cite{C:OnePeiWat11},  chosen-ciphertext security~\cite{EC:KilMohOne10,PKC:DFMO14}, and  functional encryption~\cite{EPRINT:ONeill10b,C:DIJOPP13, CANS:BelONe13}.   %Furthermore, he collaborated with database faculty George Kollios on refinements to order-preserving encryption~\cite{}.
Since joining Georgetown, he has also been working on  applications of indistinguishability obfuscation~\cite{DGLOZ14} and on integrating cryptography with emerging applications, such as outsourced database systems using modular order-preserving encryption~\cite{SIGMOD:MCOKC15} and  privacy preserving network provenance using structured encryption~\cite{LOSZZ14}.

Prior support: ``Program Obfuscation: From Theory to Practice." NSF Research Experiences for Undergraduates Supplement (PI).
Supplement to Award \#IIP-1362046,   2014 - 2019, \$8,000.

\section{Schedule and Management Plan}

\bibliographystyle{alpha}
{
\bibliography{cryptobib/abbrev3,cryptobib/crypto,new}
}





\end{document}