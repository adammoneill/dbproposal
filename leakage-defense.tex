

\subsection{Component C-I:  Enhancing security of M-POPE}
One mitigation of the attacks we propose to look at is \emph{modularity} in the
sense of modular order-preserving encryption (M-OPE)~\cite{C:BolCheOne11}.  The
idea of M-OPE is to apply a secret random offset modulo the largest possible
message to a message before encrypting it (the secret random offset is chosen
once and fixed in the secret key), so that everything gets ``shifted.''
Although similar attacks seem to apply to M-OPE, we propose to investigate more
fine-grained modularity as a defense. 

\paragraph{Random offset for each digit.}
In particular, we propose \emph{digit-modular} OPE (DM-OPE) where there is a
secret modular offset applied to each digit.  The base in which the data is
written could even itself be secret.  It becomes more complicated to make range
queries with DM-OPE, as it requires an exponential number of queries in the
number of ``wrap around'' digits in the query.  However, we propose to
investigate approximating the queries efficiently (with some false positives
that the client can filter out).


\subsection{Component C-II:  Enhancing security of POPE}
A co-PI recently introduced a new cryptographic approach, called POPE, to
supporting range queries over encrypted data, providing stronger security than
previous order-preserving encryptions~\cite{CCS:RACY16}.  In contrast to
existing OPE schemes, the server builds a novel indexing structure called a
POPE tree, in which each node has a {\it unsorted buffer} and a sorted list of
elements.  Thanks to this, the scheme can perform {\it lazy indexing, by
sorting values only when necessary}. In particular, on each range query the
scheme sorts the part of the data that is accessed during the search, leaving
much of the data untouched.

\paragraph{Less leakage.}
In this proposal, we plan to investigate whether we can further reduce the
leakage of the POPE scheme. 

\begin{question}
Can we reduce the leakage for the POPE scheme?
\end{question}

We observe that in POPE, each search query gradually leaks the ordering
information mainly for the following reasons: 

\begin{itemize}
\item The tree is essentially a search tree. For example, the elements on the
  left sub-tree is is smaller than those on the right.
 
\item Once a ciphertext comes in a POPE tree, it never changes. This {\em
  deterministic} nature allows the attacker to trace when the ciphertext came
  in the tree, and how it was brought down to some leaf node.  
\end{itemize}

We plan to address those issue by taking the following approaches:

\begin{itemize}
\item {\it Shuffle child links in a tree node.} Since POPE is an interactive
  protocol, when a tree node is created, we can slightly modify the protocol so
    that the client additionally change the order of links to its children.
    Note that with this modification, still the search can be implemented
    correctly thanks to the client's correct guidance when navigating the tree
    nodes. 
    
 
\item  {\it Re-randomize the encryptions.} We observe that POPE can use a
  randomize encryption. Therefore, by re-randomizing encryptions, when they go
    down the tree (along with the shuffling idea above), we can erase the trace
    to a certain degree. 
\end{itemize}



\paragraph{Leakage over time.}
While POPE allows each search query to {\it gradually} leak the ordering
information of the underlying plaintexts, {\em most} of the ordering
information will be leaked over many queries. This begs the following question: 

\begin{question}
Can we still maintain a sufficient number of incomparable elements, even after
  many range queries have been performed?  
\end{question}


One promising approach is taking advantage of ORAM (Oblivious Random Acess
Memory). Although they are too slow to used throughout the entire system
containing a large amount of data, we hope that ORAMs can be used effectively
for achieving stronger privacy for the sensitive sub-part of the system (e.g.,
the bottom parts of the POPE tree).  In fact, ORAMs have been used to minimize
the leakage for SSE (symmetric searchable encryption) schemes which support
keyword search over encrypted data~\cite{NDSS:StePapShi14,C:GarMohPap16}. We
plan investigate whether some part POPE trees can be modified and incorporated
with ORAMs so that the resulting system enjoys stronger privacy while only
having marginal performance
degradation.  

\paragraph{Forward security.} todo: cite some searchable encryption and mention some techniques. 

\begin{question}
Can we modify the POPE system to achieve forward security?
\end{question}


todo: blind seer?
