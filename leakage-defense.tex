%!TEX root = dbproposal.tex


\subsection{Experimental performance and security evaluation}

\emph{Throughout all phases of the project}, the most promising of our
schemes will be implemented and tested on realistic scenarios.
Indeed, we plan to include
experimental components in many of our published artifacts, with
links to the open source implementations developed.

In order to evaluate the real-world performance of our schemes, we will
be careful to use realistic and relevant datasets and queries. Tools
such as the automatic test-suite generator developed in the IARPA SPAR
project \cite{HH14,varia2015automated} will be used to generate
repeatable and realistic experiments.

Our experimental evaluations will also cover the \emph{security} of our
schemes. Frequently the development of applied cryptographic tools
relies exclusively on security proofs in
discussing privacy. Our scalable approaches indeed contain security
proofs, but also usually entail some limited information leakage.
We will use empirical tools to highlight the practical
implications of this limited leakage, including leakage under known
attacks. For example, in \cite{CCS:RACY16}, two of the co-PIs not only
evaluated the number of incomparable ciphertext pairs in theory, but
also tested this leakage on a publicly-available salary database.
