
\subsection{Component C-I:  Attack Mitigations}

\iffalse
\paragraph*{Modularity.}  One mitigation of the attacks we propose to look at
is \emph{modularity} in the sense of modular order-preserving encryption
(M-OPE)~\cite{C:BolCheOne11}.  The idea of M-OPE is to apply a secret random
offset modulo the largest possible message to a message before encrypting it
(the secret random offset is chosen once and fixed in the secret key), so that
everything gets ``shifted.''  Although similar attacks seem to apply to M-OPE,
we propose to investigate more fine-grained modularity as a defense.  In the
specific case of order-preserving encryption, we propose \emph{digit-modular}
OPE (DM-OPE) where there is a secret modular offset applied to each digit.  The
base in which the data is written could even itself be secret.  It becomes more
complicated to make range queries with DM-OPE, as it requires an exponential
number of queries in the number of ``wrap around'' digits in the query.
However, we propose to investigate approximating the queries efficiently (with
some false positives that the client can filter out).
\fi


\paragraph*{POPE: Partial order preserving encoding.} A co-PI recently
introduced a new cryptographic mechanism, called POPE, to range queries over
encrypted data that is optimized to support insert-heavy
workload~\cite{CCS:RACY16}.  POPE provides much stronger security than previous
order-preserving encryptions for the following reasons: 

\begin{itemize}
\item The search 
