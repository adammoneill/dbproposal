%!TEX root = dbproposal.tex

In this section of the proposal, we outline our efforts to analyze,
implement, and compare the most promising schemes from the previous two
sections. These efforts will be \emph{concurrent} and \emph{ongoing}
with the theoretical developments, as we continually seek to understand
the pratical benefits and drawbacks of different approaches.

The main goals of this thrust of our work is to put our schemes in
context of existing approaches and provide clear comparisons for
practitioners and researchers alike.
The work of this section will also feature the most prominent engagement
with student researchers, particularly
undergraduate students at our respective institutions.

\subsection{Leakage analysis}

All of the schemes we have proposed make some compromise of leakage for
performance. Unfortunately, many prior works either focus primarily on security
proofs and (sometimes novel) security definitions, or make heuristic
arguments for security. This means it is not
straightforward, for example, to decide what is ``more secure'' between
different approaches.

Significant progress towards fair comparisons was made recently in the
Systemization of Knowledge paper by one of the co-PIs and others
\cite{SP:FVYSHG17}. We will continue in this vein and use existing
metrics whenever possible to place our work in a fair context within the
state of the art.

An important component of this analysis is cognizance of
recent attacks on PPE and related schemes such as
\cite{CCS:CGPR15,CCS:KKNO16}. These attacks often depend crucially on
the datasets used, and we will use the same (or equivalent) attacks
against our schemes to provide a meaningful comparison.

Much of the work of this proposal is in extending existing
non-interactive and interactive schemes to the multi-dimensional setting
via some support of distance queries. Some existing metrics can apply
directly in this setting, for example the notion of \emph{incomparable
pairs} introduced in \cite{CCS:RACY16}. Other notions related to
closeness, i.e., \emph{pairwise distance}, also make sense in a
Euclidean space. Rather than develop entirely new definitions, when
possible we will use existing metrics to quantify the security
improvements our schemes provide.

\subsection{Implementation and experimental analysis}

\emph{Throughout all phases of the project}, the most promising of our
schemes will be implemented and tested on realistic scenarios.
Indeed, we plan to include
experimental components in many of our published artifacts, with
links to the open source implementations developed.

As compatibility with existing database software is a key priority of
our approach, we will seek to incorporate our schemes within existing
open-source projects such as the SSE library Clusion
(\url{https://github.com/encryptedsystems/Clusion}) and
the cloud-based graph database software
Apache Rya (\url{https://rya.apache.org/}).

In order to evaluate the real-world performance of our schemes, we will
be careful to use realistic and relevant datasets and queries. Tools
such as the automatic test-suite generator developed in the IARPA SPAR
project \cite{HH14,varia2015automated} will be used to generate
repeatable and realistic experiments.

Our experimental evaluations will also cover the \emph{security} of our
schemes. Frequently the development of applied cryptographic tools
relies exclusively on security proofs in
discussing privacy. Our scalable approaches indeed contain security
proofs, but also usually entail some limited information leakage.
We will use empirical tools to highlight the practical
implications of this limited leakage, including leakage under known
attacks. For example, in \cite{CCS:RACY16}, two of the co-PIs not only
evaluated the number of incomparable ciphertext pairs in theory, but
also tested this leakage on a publicly-available salary database.
