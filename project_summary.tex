\documentclass[11pt]{article}

\usepackage[margin=.75in]{geometry}

\begin{document}


\begin{centering}
\LARGE
Project Summary
\hline
\end{centering}

\paragraph{Background:}  

The volume of information associated with individuals, devices, and organizations is growing exponentially.  This growth is outpacing organizations' ability to manage and organize this data even given the rich functionality of modern databases. Organizations are turning to external cloud providers to manage their growing data needs.  In the information age, this data is many online businesses most valuable asset.  This migration presents a serious risk: can cloud providers be trusted to maintain confidentiality of the data?  Encryption can preserve confidentiality at the cost of removing the data' organization.  However, the very utility of the stored data derives from the ability to quickly query, process, and derive insights.

Two main research threads have focused on closing this gap, using cryptography to create databases that are capable of efficiently answering a client's queries without revealing information to the server.  
Property-preserving encryption creates cryptography that can be used with current unprotected databases. Examples include deterministic encryption and order-preserving encryption which can be used to answer equality and range queries respectively.  This approach has seen adoption in academia, the start-up community, and established enterprises.  Unfortunately, recent attacks against these schemes indicates even ideal property-preserving encryption can't be secure for order-preserving encryption.  The second thread starts from secure multi-party computation and optimizes solutions for common database tasks.  This approach requires substantially more work including designing database indexing mechanisms in conjunction with cryptographic techniques.  This work has resulted better security at the cost of decreased efficiency.  Furthermore, functionality in both approaches is limited to SQL and noSQL functionality.

Emerging databases include large scale graph databases, databases optimized for machine learning analytics, and biometric databases.  All of these databases share a fundamental operation of computing proximity of tuples of points.  The proposed research will use proximity queries as a case study to understand if the sharp tradeoff between property-preserving encryption and multi-party computation based approaches is necessary.  The research will show that for many metric spaces, the ideal object for property-preserving encryption that allows distance computation is even weaker than order-preserving encryption.  Furthermore, we will extend multi-party computation based techniques and expect unaccepted overhead in high latency networks.

The core of our proposal work is on a third approach which uses ideas from property-preserving encryption but with improved security.  The idea is to use standard database indexing mechanisms but involve the client in sensitive operations.  Previous instantiations of this approach including property-revealing encryption, partial property-preserving encryption.  We will extend these techniques to handle distance queries, add data dynamism, forward security, and formally analyze leakage (deviation from an ideal functionality).

\paragraph{Intellectual Merits:}  

Encrypted databases and searchable encryption has a rich history rooted in the design of oblivious random access machines in the early 90s.  Despite a plethora of work, the current landscape is littered with approaches with uncertain security or inadequate functionality.  The leakage inference attacks of the last few years have taught us the importance of limiting leakage and forward security.  At the same time, the rapid deployment of property-preserving techniques has reinforced the importance of simplicity, efficiency and backward compatibility.  These two developments inform the core of our approach: using interactivity to strengthen property-preserving encryption.

\paragraph{Broader Impacts:}

The protection of sensitive data is a core societal concern as more of our lives moves online.  Currently deployed encrypted databases provide little security and may even hurt as they provide a false sense of security.  There is a tremendous need for fundamental research in this area to understand the optimum security, functionality, and efficiency tradeoff.

The PIs are committed to broad dissemination of research material.  All PIs have contributed to the area of searchable encryption participating in large scale evaluation of proposed techniques and have worked on both constructions and attacks.  Furthermore, the PIs have open-sourced previous work including work on searchable encryption.  Lastly, all PIs are dedicated to engaging with undergraduates, with one of the PIs being at an undergraduate only institution, the US Naval Academy.
\end{document}