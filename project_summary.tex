\documentclass[11pt]{article}

\usepackage[margin=.75in]{geometry}

\begin{document}


\begin{centering}
\LARGE
Project Summary
\hline
\end{centering}

\paragraph{Background:}  
The growth of data volume is outpacing organizations' ability to manage and organize this data. Organizations are turning to external cloud providers to manage their data needs.  In the information age, this data is many online businesses most valuable asset.  This migration presents a serious risk: can cloud providers be trusted to maintain confidentiality of the data?  Encryption can preserve confidentiality at the cost of removing the data' organization.  However, the very utility of the stored data derives from the ability to quickly query, process, and derive insights.  Ideally, we could use more sophisticated cryptography to create databases capable of efficiently answering a client's queries without revealing information to the server.  

Two research threads have focused on closing this gap (collectively known as searchable encryption).
Property-preserving encryption creates symmetric encryption techniques compatible with an unprotected database. Examples include deterministic encryption which can answer equality queries and order-preserving encryption which can answer range queries. Academic teams, start-up companies, and Fortune 500 companies all offer versions of property-preserving encryption.  Recent attacks against these schemes indicate even ideal property-preserving encryption can't be secure for many applications.

The second thread starts from generic secure multi-party computation and optimizes solutions for common database tasks.  This approach requires redesign of database indexing mechanisms.  This approach results in better security at the cost of decreased efficiency.  Furthermore, both approaches only offer SQL and NoSQL functionality.

Emerging databases include large scale graph databases, analytic databases, and biometric databases.  These databases share a fundamental operation: computing distance/proximity of tuples of points.  The proposed research will use proximity queries as a case study to introduce a third approach to encrypted search between property-preserving encryption and multi-party computation based approaches.  The proposed approach is to use standard database indexing mechanisms but (interactively) involve the client to help with sensitive operations. Previous instantiations of this approach including property-revealing encryption and partial property-preserving encryption.  We will extend these techniques to handle distance queries, dynamic data, forward security, and formally analyze leakage (deviation from an ideal functionality).

To show the promise of this approach, the proposed research will proximity functionality of the two standard approaches.  The research will show that for many metric spaces, the ideal object for distance-preserving encryption computation is weaker than order-preserving encryption.  Furthermore, we will extend multi-party computation based techniques.  We expect unacceptable overhead in networks with realistic latency.
\paragraph{Intellectual Merits:}  

Encrypted databases and searchable encryption has a rich history rooted in the design of oblivious random access machines.  The current landscape is littered with approaches with uncertain security or inadequate functionality.  Recent leakage inference attacks have taught us the impact of leakage and forward security.  At the same time, the rapid deployment of property-preserving techniques has reinforced the importance of simplicity, efficiency and backward compatibility.  These two developments inform the core of our approach: using interactivity to strengthen property-preserving encryption.

\paragraph{Broader Impacts:}

The confidentiality of data is a core societal tenant.  Deployed encrypted databases provide little security and may even hurt by providing a false sense of security.  There is a tremendous need for research in this area to understand the tradeoffs between security, functionality, and efficiency.

The PIs are committed to broad dissemination of research material.  The PIs have participated in large scale evaluation of searchable encryption, working on both constructions and attacks.  Furthermore, the PIs have open-sourced previous work including work on searchable encryption.  Lastly, all PIs are dedicated to engaging with undergraduates, with one of the PIs being at an undergraduate only institution, the US Naval Academy.
\end{document}