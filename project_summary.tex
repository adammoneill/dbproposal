\documentclass[11pt]{article}

\usepackage[margin=.75in]{geometry}
\usepackage{titlesec}
\titlespacing{\paragraph}{%
  0pt}{%              left margin
  0.5\baselineskip}{% space before (vertical)
  1em}%               space after (horizontal)


\begin{document}


\begin{center}
\LARGE
Project Summary 
\end{center}

\vspace{-1.5em}
\noindent \hrulefill



\paragraph{Motivation.}
%The growth of data volume is outpacing organizations' ability to manage and
%organize this data. Organizations are turning to external cloud providers to
%manage their data needs.  In the information age, this data is many online
%businesses most valuable asset.  This migration presents a serious risk: can
%cloud providers be trusted to maintain confidentiality of the data?  Encryption
%can preserve confidentiality at the cost of removing the data' organization.
%However, the very utility of the stored data derives from the ability to
%quickly query, process, and derive insights.  Ideally, we could use more
%sophisticated cryptography to create databases capable of efficiently answering
%a client's queries without revealing information to the server.  
%
The growth of data is outpacing organizations' abilities to manage and
organize this data. Organizations are turning to external cloud providers
to manage their data needs.  
%
The migration should maintain confidentiality of their
valuable data, thus the data is often outsourced in an encrypted from. 
%
However, employing encryption prevents the cloud server
from {\em quickly processing the data in plaintext form and answering
complex queries from the client}. 
%
Ideally, we could use more sophisticated cryptography to {\em create databases
capable of efficiently answering a client's queries without revealing
information to the cloud server}.  

%\paragraph{Previous works.}
Two research threads have focused on closing this gap.
%(collectively known as
%searchable encryption).  
Property-preserving encryption creates ciphertexts with weaker
security but better functionality so that certain operations can be seamlessly
performed within an unprotected database.
%
For example,
order-preserving encryption can answer range queries. 
Academic teams,
start-up companies, and Fortune 500 companies all offer variants of
property-preserving encryption.
%Recent attacks against these schemes indicate even ideal property-preserving
%encryption can't be secure for many applications.
%
The second thread starts from generic secure multi-party computation and
optimizes for database tasks. This approach requires redesign
of database indexing mechanisms and results in better security at the cost of
decreased efficiency and compatibility. 
%
So far, both approaches offer only the {\em limited functionality
of SQL and NoSQL}. 

\paragraph{Proposed work.}
The wide deployment of property-preserving encryption highlights the need for \emph{encrypted search} solutions to be maximally backward compatible.  At the same time, solutions should provide meaningful security.

Emerging databases include graph, analytic, and
biometric databases. These databases share a fundamental operation: {\em
computing distance/proximity of tuples of points}. The proposed research will
consider proximity queries as a case study for the following question: \emph{can encrypted search solutions achieve security while maintaining backwards compatibility?}  

Our approach uses standard database mechanisms but involves the
client to help with sensitive operations. This approach was previously used to make order-preserving encryption more secure. Very roughly, this scheme overrides the comparison operation with an interactive protocol.  We will extend these techniques to
{\em handle distance queries and dynamic data}.
%As an example, one can override a comparison operator to use an interactive protocol.
%The resulting constructions will have
%better security than property-preserving encryption schemes and better compatibility and 
%efficiency than MPC-based schemes. 

To validate this approach, we will determine which metrics and definitions allow for a secure instantiation of distance-preserving encryption. For settings where distance-preserving encryption is inadequate, we will design
customized, interactive schemes for distance/proximity. Given the variety of claims in the literature, it critical to evaluate security, functionality, and efficiency.  We will develop security and efficiency benchmarks for proximity databases. We will implement and evaluate the most promising developed schemes.

%To show the promise of this approach, the proposed research will first give
%constructions for the proximity functionality of the two standard approaches.
%
%and show that for many metric spaces, the ideal object for distance-preserving
%encryption computation has a different characteristic from that for the
%order-preserving encryption,   
%
%Furthermore, we will extend multi-party computation based techniques.  We
%expect unacceptable overhead in networks with realistic latency.  

\paragraph{Intellectual merit.}  
Encrypted search has a rich history starting with oblivious random access machines.  Current approaches provide uncertain security or inadequate functionality.  Recent
leakage inference attacks have shown the importance of understanding security guarantees.  Simultaneously, the deployment of property-preserving
techniques has highlighted the benefits of simplicity, efficiency, and backward
compatibility.  These two developments inform the core of our approach: designing interactive schemes while maximally maintaining compatibility.  Our schemes will be thoroughly evaluated with respect to security and efficiency.

\paragraph{Broader impacts.}
The confidentiality of data is a core societal tenant.  Deployed encrypted search solutions provide little security and may even hurt by providing a false sense
of security.  There is a tremendous need to
understand the tradeoffs between security, functionality, and efficiency.

The PIs are committed to dissemination of research artifacts.  In addition to constructions, the PIs
have evaluated the efficiency and security of encrypted databases at scale.  The PIs have open-sourced
previous work on secure databases.  A major component of this proposal is contributing benchmarks to the community. Lastly, all PIs are
dedicated to engaging with undergraduates, with two co-PIs being at an
undergraduate only institution, the U.S.\ Naval Academy.  
\end{document}
