\documentclass[11pt]{article}

\usepackage[margin=.75in]{geometry}
\usepackage{titlesec}
\titlespacing{\paragraph}{%
  0pt}{%              left margin
  0.5\baselineskip}{% space before (vertical)
  1em}%               space after (horizontal)


\begin{document}


\begin{center}
\LARGE
Project Summary 
\end{center}

\vspace{-1.5em}
\noindent \hrulefill



\paragraph{Motivation.}
%The growth of data volume is outpacing organizations' ability to manage and
%organize this data. Organizations are turning to external cloud providers to
%manage their data needs.  In the information age, this data is many online
%businesses most valuable asset.  This migration presents a serious risk: can
%cloud providers be trusted to maintain confidentiality of the data?  Encryption
%can preserve confidentiality at the cost of removing the data' organization.
%However, the very utility of the stored data derives from the ability to
%quickly query, process, and derive insights.  Ideally, we could use more
%sophisticated cryptography to create databases capable of efficiently answering
%a client's queries without revealing information to the server.  
%
The growth of data is outpacing organizations' abilities to manage and
organize this data. Organizations are turning to external cloud providers
to manage their data needs.  
%
The migration should maintain confidentiality of their
valuable data, thus the data is often outsourced in an encrypted from. 
%
However, employing encryption comes at the cost of preventing the cloud server
from {\em quickly processing the data in plaintext form and answering
complex queries from the client}. 
%
Ideally, we could use more sophisticated cryptography to {\em create databases
capable of efficiently answering a client's queries without revealing
information to the cloud server}.  

\paragraph{Previous works.}
Two research threads have focused on closing this gap.
%(collectively known as
%searchable encryption).  
Property-preserving encryption creates symmetric encryption techniques
compatible with an unprotected database. Examples include deterministic
encryption which can answer equality queries and order-preserving encryption
which can answer range queries. Academic teams, start-up companies, and Fortune
500 companies all offer variants of property-preserving encryption.
%Recent attacks against these schemes indicate even ideal property-preserving
%encryption can't be secure for many applications.
%
The second thread starts from generic secure multi-party computation (MPC) and
optimizes solutions for common database tasks. This approach requires redesign
of database indexing mechanisms and results in better security at the cost of
decreased efficiency and compatibility. 
%
We note that so far, both approaches offer only the {\em limited functionality
of SQL and NoSQL}. 

\paragraph{Proposed work.}
The wide deployment of property-preserving encryption highlights the need for cryptographic solutions to be maximally backward compatible.  At the same time, solutions should provide meaningful security.

Emerging databases include graph, analytic, and
biometric databases. These databases share a fundamental operation: {\em
computing distance/proximity of tuples of points}. The proposed research will
consider proximity queries as a case study for the following question: \emph{can secure databases achieve security while maintaining backwards compatibility?}  As an example, one can override a comparison operator to use an interactive protocol without rewriting the database software.

To show the promise of this approach, we will consider which metrics and definitions allow for a secure instantiation of distance-preserving encryption. For settings where distance-preserving encryption is inadequate, we will design
customized, interactive schemes for distance/proximity. Lastly, we will extend the security of these schemes, creating forward secure versions and removing information leakage.

The resulting constructions will have
better security than property-preserving encryption schemes and better compatibility and 
efficiency than MPC-based schemes. Our approach will use standard database indexing mechanisms but (interactively) involve the
client to help with sensitive operations. This approach has been used in the
context of order-preserving encryption, and we will extend these techniques to
{\em handle distance queries and dynamic data}.

%To show the promise of this approach, the proposed research will first give
%constructions for the proximity functionality of the two standard approaches.
%
%and show that for many metric spaces, the ideal object for distance-preserving
%encryption computation has a different characteristic from that for the
%order-preserving encryption,   
%
%Furthermore, we will extend multi-party computation based techniques.  We
%expect unacceptable overhead in networks with realistic latency.  

\paragraph{Intellectual merit.}  
Encrypted databases and searchable encryption have rich histories starting with  oblivious random access machines.  Current approaches provide uncertain security or inadequate functionality.  Recent
leakage inference attacks have shown the importance of leakage and the need for forward
security.  Simultaneously, the rapid deployment of property-preserving
techniques has reinforced the importance of simplicity, efficiency and backward
compatibility.  These two developments inform the core of our approach: using
interactivity while maximally maintaining compatibility.

\paragraph{Broader impacts.}
The confidentiality of data is a core societal tenant.  Deployed encrypted
databases provide little security and may even hurt by providing a false sense
of security.  There is a tremendous need for research in this area to
understand the tradeoffs between security, functionality, and efficiency.

The PIs are committed to dissemination of research material and code artifacts.  The PIs
have participated in large scale evaluation of searchable encryption, working
on both constructions and attacks.  The PIs have open-sourced
previous work on secure databases.  Lastly, all PIs are
dedicated to engaging with undergraduates, with two co-PIs being at an
undergraduate only institution, the US Naval Academy.  
\end{document}
