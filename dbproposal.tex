\documentclass[11pt]{article}

\usepackage{fullpage,varwidth,url}
\usepackage{amssymb}
\usepackage[table]{xcolor}
\usepackage{amsmath,amsthm}
\usepackage{relsize}
\usepackage{color}
%
%    \setlength{\evensidemargin}{-0.2in}
%    \setlength{\oddsidemargin}{-0.2in}
%    \setlength{\textwidth}{6.9in}
%    \setlength{\textheight}{9.5in}
%    \setlength{\topmargin}{-0.35in}
%    \setlength{\headheight}{0in}
%    \setlength{\headsep}{0in}
%    \setlength{\footskip}{0.5in}
\title{{{\Large{SaTC: CORE: Small:  Collaborative: \\Next-Generation Secure Outsourced Databases}}}}
%\author{Sanjam Garg \\University of California, Berkeley \and 	Mohammad Mahmoody \\ University of Virginia\and Adam O'Neill \\Georgetown University}

\usepackage[hidelinks]{hyperref}


%\usepackage{anysize} % to set up document margins.
%\marginsize{1.1in}{1.1in}{1.1in}{1.0in}
%\vspace{-1mm}
\usepackage{times}
\usepackage{etex}
\usepackage{framed}

\newcommand{\Att}{\textsc{Attack}}
\newcommand{\Hyb}{\textsc{Hybrid}}
\newcommand{\Idl}{\textsc{Ideal}}
\newcommand{\Lzy}{\textsc{Lazy}}


\newcommand{\OR}{\mathsf{OR}}
\newcommand{\AND}{\mathsf{AND}}


\newtheorem{thm}{Theorem}[section]
\newtheorem{lem}[thm]{Lemma}
\newtheorem{cor}[thm]{Corollary}
\newtheorem{propo}[thm]{Proposition}
\newtheorem{clm}[thm]{Claim}
\newtheorem{defn}[thm]{Definition}
\newtheorem{conj}[thm]{Conjecture}
\newtheorem{assm}[thm]{Assumption}
\newtheorem{rem}[thm]{Remark}
\newtheorem{obs}[thm]{Observation}
\newtheorem{egs}[thm]{Example}
\newtheorem{fct}[thm]{Fact}
\newtheorem{expr}{Experiment}
\newtheorem{cons}[thm]{Construction}
\newtheorem{nte}[thm]{Note}
\newtheorem{aim}[thm]{Aim}

% keys mathsf
%\newcommand{\sk}{\mathsf{sk}}
%\newcommand{\SK}{\mathsf{SK}}
%\newcommand{\pk}{\mathsf{pk}}
%\newcommand{\PK}{\mathsf{PK}}
%\newcommand{\dk}{\mathsf{dk}}
%\newcommand{\DK}{\mathsf{DK}}


%\newcommand{\att}{\cA}
%\newcommand{\prd}{\cP}


%%%%%%%%%%%%%%%%%%%%%%%%%%%%%%%%%%%%%%%%%%%%%%%%%%%55
\newcommand{\remove}[1]{}
\newcommand{\num}[1]{{\bf (#1)}}

\newcommand{\procfont}[1]{\mathsc{#1}}
\newcommand{\tablefont}[1]{\mathsf{#1}}

\newcommand{\mpk}{\pi}
\newcommand{\msk}{\varfont{msk}}
\newcommand{\usk}{\varfont{dk}}
\newcommand{\usknew}{\overline{\varfont{dk}}}
\newcommand{\ctxt}{\varfont{C}}
\newcommand{\ctxtnew}{\overline{\varfont{C}}}
\newcommand{\IBE}{\schemefont{IBE}}
\newcommand{\IBEnew}{\overline{\schemefont{IBE}}}
\newcommand{\mskspace}{{\cal S}}
%\newcommand{\kg}{{\cal K}}
\newcommand{\kg}{\mathsf{Kg}}
\newcommand{\kgnew}{\overline{{\cal K}}}
%\newcommand{\pg}{{\cal P}}
\newcommand{\pg}{\mathsf{Pg}}
\newcommand{\Enc}{\enc}
\newcommand{\Encnew}{\encnew}
\newcommand{\Dec}{\decc}
\newcommand{\Decnew}{\deccnew}

\def\vk{\mathit{vk}}
\def\pars{\pi}
\def\ek{\varfont{ek}}
\def\pk{\varfont{pk}}
\def\sk{\varfont{sk}}
\newcommand{\vecsk}{\mathbf{sk}}
\newcommand{\vecpk}{\mathbf{pk}}
\newcommand{\vecm}{\mathbf{m}}
\newcommand{\vecM}{\mathbf{M}}
\def\dk{\varfont{dk}}
\def\ak{\varfont{ak}}
\def\stateinfo{\varfont{st}}
\def\st{\varfont{state}}
\def\stt{\varfont{St}}
\newcommand{\RO}{H}
\newcommand{\keygen}{KG}
\newcommand{\pargen}{PG}
\newcommand{\trapdoor}{Td}
\newcommand{\test}{Test}
%\newcommand{\setup}{Setup}
\renewcommand{\AE}{\mathcal{AE}}
%\newcommand{\enc}{\mathcal{E}}
\newcommand{\enc}{\mathsf{Enc}}
%\newcommand{\decc}{\mathcal{D}}
\newcommand{\decc}{\mathsf{Dec}}
\newcommand{\eval}{\mathsf{Eval}}
\newcommand{\genc}{\algfont{\Enc}}
\newcommand{\dec}{\mathsf{Dec}}
\newcommand{\gdec}{\algfont{\Dec}}
%\newcommand{\com}{Com}
\newcommand{\ver}{Ver}
\newcommand{\open}{\ver}
\def\Hash{H}

\newcommand{\Funcs}{\mathsf{Funcs}}

\newcommand{\ol}{\overline}
\newcommand{\wt}[1]{\widetilde{#1}}
\newcommand{\wh}[1]{\widehat{#1}}
\newcommand{\nin}{\not \in}

\newcommand{\es}{\varnothing} % empty set
\newcommand{\se}{\subseteq}
\newcommand{\sm}{\setminus}
\newcommand{\nse}{\not \se}
\newcommand{\nf}[2]{\nicefrac{#1}{#2}}
\newcommand{\rv}[1]{\mathbf{#1}} % Random Variable
\newcommand{\mal}[1]{\widehat{#1}} % Malicious version of the alg

\newcommand{\schemefont}[1]{{\mathsf{#1}}}
\newcommand{\primfont}[1]{{\mathsf{#1}}}
\newcommand{\algfont}[1]{{\mathsf{#1}}}
\newcommand{\advfont}[1]{{\mathsf{#1}}}
\newcommand{\orfont}[1]{\mathsc{#1}}
\newcommand{\varfont}[1]{\mathit{#1}}
\newcommand{\randvarfont}[1]{\mathbf{#1}}
\newcommand{\constfont}[1]{\mathtt{#1}}
\newcommand{\notionfont}[1]{\mathrm{#1}}
\newcommand{\eventfont}[1]{{\mathsc{#1}}}
\newcommand{\vectorfont}[1]{\mathslbf{#1}}
\newcommand{\transformfont}[1]{\mathsf{#1}}


\newcommand{\tuple}[1]{\langle{#1}\rangle}
\def\from{\mbox{from}\ }
\def\From{\mbox{From}\ }
\def\bits{\{0,1\}}
\def\cross{\times}
\newcommand{\xor}{{\oplus}}
\newcommand{\Colon}{{:\;\;}}
\newcommand{\mystrut}{\rule{0em}{15pt}}
\newcommand{\namestrut}{\rule{0em}{20pt}}
\def\poly{\mathop{\rm poly}\nolimits}
\def\emptystring{\varepsilon}
\def\emptyid{()}
\newcommand{\Dom}{\mathsf{Dom}}
\newcommand{\DomR}{\mathsf{DomR}}
\newcommand{\Rng}{\mathsf{Rng}}
\newcommand{\RngR}{\mathsf{RngR}}
\newcommand{\Keys}{\mathsf{Keys}}
\newcommand{\calA}{{\cal A}}
\newcommand{\calB}{{\cal B}}
\newcommand{\calC}{{\cal C}}
\newcommand{\calF}{{\cal F}}
\newcommand{\calL}{{\cal L}}
\newcommand{\calM}{{\cal M}}
\newcommand{\calO}{{\cal O}}
\newcommand{\calR}{{\cal R}}
\newcommand{\calH}{{\cal H}}
\newcommand{\calG}{{\cal G}}
\newcommand{\calD}{{\cal D}}
\newcommand{\calE}{{\cal E}}
\newcommand{\calP}{{\cal P}}
\newcommand{\calS}{{\cal S}}
\newcommand{\calU}{{\cal U}}
\newcommand{\calX}{{\cal X}}
\newcommand{\calY}{{\cal Y}}
\newcommand{\calEE}{{\cal EE}}
\newcommand{\N}{{{\mathbb N}}}
\newcommand{\Z}{{{\mathbb Z}}}
\newcommand{\R}{{{\mathbb R}}}
\newcommand{\goesto}{{\rightarrow}}
\newcommand{\eqdef}{\;\stackrel{\rm def}{=}\;}
\newcommand{\then}{{\;;\;\;}}   % for [ ; ; ; : ] notation
\newcommand{\andthen}{{\;:\;\;}}
\def\union{\cup}
\def\bigunion{\bigcup}
\def\intersection{\cap}
\def\bigintersection{\bigcap}
\def\suchthatt{\: :\:}
\newcommand{\suchthat}{{\mbox{s.t.\ }}}
\def\next{\:;\:}
\def\nextt{\:;\:}
\newcommand{\sett}[1]{\{#1\}}
\newcommand{\set}[2]{\{\:#1 \suchthatt #2\:\}}
\newcommand{\setsize}[1]{\left|{#1}\right|}
\def\leqq{\;\leq\;}
\def\eqq{\;=\;}
\def\geqq{\;\geq\;}
\def\equivv{\;\equiv\;}
\def\prn#1{\left(#1\right)}
\newcommand{\getsr}{{\:{\leftarrow{\hspace*{-3pt}\raisebox{.75pt}{$\scriptscriptstyle\$$}}}\:}}
% \def\getsr{\stackrel{{\scriptscriptstyle\$}}{\leftarrow}}
% \renewcommand{\choose}[2]{{{#1}\atopwithdelims(){#2}}}

\newcommand{\Var}{{\mbox{\bf Var}}}
\newcommand{\E}{\mathbf{E}}
\newcommand{\EE}[1]{{\E\left[{#1}\right]}}
\newcommand{\EEE}[2]{{\E_{#1}\left[{#2}\right]}}
\newcommand{\Prob}[1]{{\Pr\left[\,{#1}\,\right]}}
\newcommand{\probb}[2]{{\Pr}_{#1}\left[\,{#2}\,\right]}
\newcommand{\probbs}[3]{{\Pr}^{#1}_{#2}\left[\,{#3}\,\right]}
\newcommand{\probbp}[2]{{\Pr}_{#1}'\left[\,{#2}\,\right]}
\newcommand{\Probb}[2]{{{\Pr}_{#1}\left[\,{#2}\,\right]}}
\newcommand{\Probbp}[2]{{{\Pr}_{#1}'\left[\,{#2}\,\right]}}
\newcommand{\condProb}[2]{{\Pr}\left[\,#1\,|\,#2\,\right]}
\newcommand{\CondProb}[2]{{\Pr}\left[\: #1\:\left|\right.\:#2\:\right]}
\newcommand{\CondProbb}[3]{{\Pr}_{#1}\left[\: #2\:\left|\right.\:#3\:\right]}
\newcommand{\CondProbbp}[3]{%
{\Pr}_{#1}^{'}\left[\: #2\:\left|\right.\:#3\:\right]}


% ===================================================================
\newcommand{\kwfont}[1]{{\ensuremath{\mathrm{#1}}}}
\newcommand{\kwfunction}{{\kwfont{function}\ }}
\newcommand{\kwlabel}{{\kwfont{label}\ }}
\newcommand{\kwfor}{{\kwfont{for}\ }}
\newcommand{\kwand}{{\kwfont{and}\ }}
\newcommand{\kwor}{{\kwfont{or}\ }}
\newcommand{\kwnot}{{\kwfont{not}}}
\newcommand{\kwdo}{{\kwfont{do}\ }}
\newcommand{\kwreturn}{{\kwfont{return}\ }}
\newcommand{\kwReturn}{{\kwfont{Return}\ }}
\newcommand{\kwalgorithm}{{\ensuremath{\mathbf{Algorithm}\ }}}
\newcommand{\kwprotocol}{{\kwfont{Protocol}\ }}
\newcommand{\kwexperiment}{{\kwfont{Experiment}\ }}
\newcommand{\kwadversary}{{\kwfont{Adversary}\ }}
\newcommand{\kworacle}{{\kwfont{Oracle}\ }}
\newcommand{\kwuntil}{{\kwfont{until}\ }}
\newcommand{\kwrepeat}{{\kwfont{repeat}\ }}
\newcommand{\kwif}{{\kwfont{If}\ }}
\newcommand{\kwthen}{{\kwfont{Then}\ }}
\newcommand{\kwelse}{{\kwfont{Else}\ }}
\newcommand{\kwabort}{{\kwfont{abort}\ }}
\newcommand{\kwgoto}{{\kwfont{goto}\ }}
\newcommand{\kwwhile}{{\kwfont{while}\ }}
\newcommand{\kwparse}{{\kwfont{parse}\ }}
\newcommand{\kwas}{{\kwfont{as}\ }}
\newcommand{\kwstatic}{{\kwfont{static}\ }}
\newcommand{\kwrun}{{\kwfont{run}\ }}
\newcommand{\kwbegin}{{\kwfont{begin}\ }}
\newcommand{\kwend}{{\kwfont{end}\ }}
\newcommand{\kwstart}{{\kwfont{start}\ }}
\newcommand{\kwcontinue}{{\kwfont{continue}\ }}
\newcommand{\kwdefine}{{\kwfont{define}\ }}
\newcommand{\kwflip}{{\kwfont{flip}\ }}
\newcommand{\kwlet}{{\kwfont{let}\ }}
\newcommand{\kwof}{{\kwfont{of}\ }}
\newcommand{\kwcase}{{\kwfont{case}\ }}
\newcommand{\kwswitch}{{\kwfont{switch}\ }}
\newcommand{\kwpick}{{\kwfont{pick}\ }}
\newcommand{\kwset}{{\kwfont{set}\ }}
\newcommand{\kwcompute}{{\kwfont{compute}\ }}
\newcommand{\comment}[1]{\hspace{15pt}{\small /$\!\!$/\ #1}}
\newcommand{\Comment}[1]{\hspace{5pt}{/$\!\!$/\ #1}}
\newcommand{\sComment}[1]{\hspace{2pt}{/$\!\!$/#1}}

\newcommand{\Coins}{\mathsf{Coins}}

\newcommand{\ProbExp}[2]{{\Pr}\left[\: #1\:\suchthatt\:#2\:\right]}

\newcommand{\PRG}{\schemefont{PRG}}
\newcommand{\PRGpg}{\schalg{PRG}{Ev}}
\newcommand{\PRGkg}{\schalg{PRG}{Kg}}

\newcommand{\FSE}{\schemefont{FSE}}

\newcommand{\PE}{\schemefont{PE}}
\newcommand{\FE}{\schemefont{FE}}
\newcommand{\setup}{\mathsf{Setup}}
\newcommand{\keyder}{\mathsf{KeyDer}}

\newcommand{\OT}{\schemefont{OT}}
\newcommand{\Send}{\procfont{Sender}}
\newcommand{\Rec}{\procfont{Receiver}}
\newcommand{\PKE}{\schemefont{PKE}}
\newcommand{\FPKE}{\schemefont{FPKE}}
\newcommand{\DPKE}{\schemefont{DPKE}}
\newcommand{\DE}{\schemefont{DE}}
\newcommand{\PKEpg}{\schalg{PKE}{Pg}}
\newcommand{\oPKEpg}{\overline{{\cal P}}}
\newcommand{\PKEEnc}{\schalg{PKE}{Enc}}
\newcommand{\oPKEEnc}{\schalg{\overline{PKE}}{Enc}}
\newcommand{\PKEDec}{\schalg{PKE}{Dec}}
\newcommand{\oPKEDec}{\overline{{\cal D}}}
\newcommand{\PKEkg}{\schalg{PKE}{Kg}}
\newcommand{\oPKEkg}{\overline{{\cal K}}}
\newcommand{\PKEpk}{\mathit{ek}}
\newcommand{\PKEPK}{\schalg{PKE}{PK}}
\newcommand{\PKESK}{\schalg{PKE}{SK}}
\newcommand{\oPKEpk}{\overline{\mathit{ek}}}
\newcommand{\PKEsk}{\mathit{dk}}
\newcommand{\oPKEsk}{\overline{\mathit{dk}}}
\newcommand{\PKEpars}{\pi}
\newcommand{\oPKEpars}{\overline{\pi}}

\newcommand{\SE}{\schemefont{SE}}
\newcommand{\SEpg}{{\cal P}}
\newcommand{\SEEnc}{{\cal E}}
\newcommand{\SEDec}{{\cal D}}
\newcommand{\SEkg}{{\cal K}}
\newcommand{\SEkey}{K}
\newcommand{\SEpars}{\pi}

%%%  English %%%%%%%%%%%%%%%%%%%%%%%%%%%%%%%%%%%%%%%%%%%%%%%%%%%%%%

\newcommand{\etal}{et~al.\ }
\newcommand{\aka}{also known as,\ }
\newcommand{\resp}{resp.,\ }
\newcommand{\ie}{i.e.,\ }
\newcommand{\wolog}{w.l.o.g.\ }
\newcommand{\Wolog}{W.l.o.g.\ }
\newcommand{\eg}{e.g.,\ }
\newcommand{\Eg}{E.g.,\ }
\newcommand{\wrt} {with respect to\ }
\newcommand{\cf}{{cf.,\ }}

%%%  math %%%%%%%%%%%%%%%%%%%%%%%%%%%%%%%%%%%%%%%%%%%%%%%%%%%%%%

\newcommand{\round}[1]{\lfloor #1 \rceil}
\newcommand{\ceil}[1]{\lceil #1 \rceil}
\newcommand{\floor}[1]{\lfloor #1 \rfloor}
\newcommand{\angles}[1]{\langle #1 \rangle}
\newcommand{\parens}[1]{( #1 )}
\newcommand{\bracks}[1]{[ #1 ]}
\newcommand{\bra}[1]{\langle#1\rvert}
\newcommand{\ket}[1]{\lvert#1\rangle}


\newcommand{\adjRound}[1]{\left\lfloor #1 \right\rceil} % Adjusted Round
\newcommand{\adjCeil}[1]{\left\lceil #1 \right\rceil}
\newcommand{\adjFloor}[1]{\left\lfloor #1 \right\rfloor}
\newcommand{\adjAngles}[1]{\left\langle #1 \right\rangle}
\newcommand{\adjParens}[1]{\left( #1 \right)}
\newcommand{\adjBracks}[1]{\left[ #1 \right]}
\newcommand{\adjBra}[1]{\left\langle#1\right\rvert}
\newcommand{\adjKet}[1]{\left\lvert#1\right\rangle}
\newcommand{\adjSet}[1]{\left\{ #1 \right\}}
\newcommand{\half}{\tfrac{1}{2}}
\newcommand{\third}{\tfrac{1}{3}}
\newcommand{\quarter}{\tfrac{1}{4}}
%\newcommand{\eqdef}{:=}
%\newcommand{\zo}{\{0,1\}}




\newcommand{\cA}{{\mathcal A}}
\newcommand{\cB}{{\mathcal B}}
\newcommand{\cC}{{\mathcal C}}
\newcommand{\cD}{{\mathcal D}}
\newcommand{\cE}{{\mathcal E}}
\newcommand{\cF}{{\mathcal F}}
\newcommand{\cG}{{\mathcal G}}
\newcommand{\cH}{{\mathcal H}}
\newcommand{\cI}{{\mathcal I}}
\newcommand{\cJ}{{\mathcal J}}
\newcommand{\cK}{{\mathcal K}}
\newcommand{\cL}{{\mathcal L}}
\newcommand{\cM}{{\mathcal M}}
\newcommand{\cN}{{\mathcal N}}
\newcommand{\cO}{{\mathcal O}}
\newcommand{\cP}{{\mathcal P}}
\newcommand{\cQ}{{\mathcal Q}}
\newcommand{\cR}{{\mathcal R}}
\newcommand{\cS}{{\mathcal S}}
\newcommand{\cT}{{\mathcal T}}
\newcommand{\cU}{{\mathcal U}}
\newcommand{\cV}{{\mathcal V}}
\newcommand{\cW}{{\mathcal W}}
\newcommand{\cX}{{\mathcal X}}
\newcommand{\cY}{{\mathcal Y}}
\newcommand{\cZ}{{\mathcal Z}}

\newcommand{\bfA}{\mathbf{A}}
\newcommand{\bfB}{\mathbf{B}}
\newcommand{\bfC}{\mathbf{C}}
\newcommand{\bfD}{\mathbf{D}}
\newcommand{\bfE}{\mathbf{E}}
\newcommand{\bfF}{\mathbf{F}}
\newcommand{\bfG}{\mathbf{G}}
\newcommand{\bfH}{\mathbf{H}}
\newcommand{\bfI}{\mathbf{I}}
\newcommand{\bfJ}{\mathbf{J}}
\newcommand{\bfK}{\mathbf{K}}
\newcommand{\bfL}{\mathbf{L}}
\newcommand{\bfM}{\mathbf{M}}
\newcommand{\bfN}{\mathbf{N}}
\newcommand{\bfO}{\mathbf{O}}
\newcommand{\bfP}{\mathbf{P}}
\newcommand{\bfQ}{\mathbf{Q}}
\newcommand{\bfR}{\mathbf{R}}
\newcommand{\bfS}{\mathbf{S}}
\newcommand{\bfT}{\mathbf{T}}
\newcommand{\bfU}{\mathbf{U}}
\newcommand{\bfV}{\mathbf{V}}
\newcommand{\bfW}{\mathbf{W}}
\newcommand{\bfX}{\mathbf{X}}
\newcommand{\bfY}{\mathbf{Y}}
\newcommand{\bfZ}{\mathbf{Z}}


\newcommand{\bfa}{\mathbf{a}}
\newcommand{\bfb}{\mathbf{b}}
\newcommand{\bfc}{\mathbf{c}}
\newcommand{\bfd}{\mathbf{d}}
\newcommand{\bfe}{\mathbf{e}}
\newcommand{\bff}{\mathbf{f}}
\newcommand{\bfg}{\mathbf{g}}
\newcommand{\bfh}{\mathbf{h}}
\newcommand{\bfi}{\mathbf{i}}
\newcommand{\bfj}{\mathbf{j}}
\newcommand{\bfk}{\mathbf{k}}
\newcommand{\bfl}{\mathbf{l}}
\newcommand{\bfm}{\mathbf{m}}
\newcommand{\bfn}{\mathbf{n}}
\newcommand{\bfo}{\mathbf{o}}
\newcommand{\bfp}{\mathbf{p}}
\newcommand{\bfq}{\mathbf{q}}
\newcommand{\bfr}{\mathbf{r}}
\newcommand{\bfs}{\mathbf{s}}
\newcommand{\bft}{\mathbf{t}}
\newcommand{\bfu}{\mathbf{u}}
\newcommand{\bfv}{\mathbf{v}}
\newcommand{\bfw}{\mathbf{w}}
\newcommand{\bfx}{\mathbf{x}}
\newcommand{\bfy}{\mathbf{y}}
\newcommand{\bfz}{\mathbf{z}}



\newcommand{\sfA}{\mathsf{A}}
\newcommand{\sfB}{\mathsf{B}}
\newcommand{\sfC}{\mathsf{C}}
\newcommand{\sfD}{\mathsf{D}}
\newcommand{\sfE}{\mathsf{E}}
\newcommand{\sfF}{\mathsf{F}}
\newcommand{\sfG}{\mathsf{G}}
\newcommand{\sfH}{\mathsf{H}}
\newcommand{\sfI}{\mathsf{I}}
\newcommand{\sfJ}{\mathsf{J}}
\newcommand{\sfK}{\mathsf{K}}
\newcommand{\sfL}{\mathsf{L}}
\newcommand{\sfM}{\mathsf{M}}
\newcommand{\sfN}{\mathsf{N}}
\newcommand{\sfO}{\mathsf{O}}
\newcommand{\sfP}{\mathsf{P}}
\newcommand{\sfQ}{\mathsf{Q}}
\newcommand{\sfR}{\mathsf{R}}
\newcommand{\sfS}{\mathsf{S}}
\newcommand{\sfT}{\mathsf{T}}
\newcommand{\sfU}{\mathsf{U}}
\newcommand{\sfV}{\mathsf{V}}
\newcommand{\sfW}{\mathsf{W}}
\newcommand{\sfX}{\mathsf{X}}
\newcommand{\sfY}{\mathsf{Y}}
\newcommand{\sfZ}{\mathsf{Z}}

\newcommand{\sfa}{\mathsf{a}}
\newcommand{\sfb}{\mathsf{b}}
\newcommand{\sfc}{\mathsf{c}}
\newcommand{\sfd}{\mathsf{d}}
\newcommand{\sfe}{\mathsf{e}}
\newcommand{\sff}{\mathsf{f}}
\newcommand{\sfg}{\mathsf{g}}
\newcommand{\sfh}{\mathsf{h}}
\newcommand{\sfi}{\mathsf{i}}
\newcommand{\sfj}{\mathsf{j}}
\newcommand{\sfk}{\mathsf{k}}
\newcommand{\sfl}{\mathsf{l}}
\newcommand{\sfm}{\mathsf{m}}
\newcommand{\sfn}{\mathsf{n}}
\newcommand{\sfo}{\mathsf{o}}
\newcommand{\sfp}{\mathsf{p}}
\newcommand{\sfq}{\mathsf{q}}
\newcommand{\sfr}{\mathsf{r}}
\newcommand{\sfs}{\mathsf{s}}
\newcommand{\sft}{\mathsf{t}}
\newcommand{\sfu}{\mathsf{u}}
\newcommand{\sfv}{\mathsf{v}}
\newcommand{\sfw}{\mathsf{w}}
\newcommand{\sfx}{\mathsf{x}}
\newcommand{\sfy}{\mathsf{y}}
\newcommand{\sfz}{\mathsf{z}}

\newcommand{\eps}{\epsilon}
%\newcommand{\e}{\epsilon}
\newcommand{\veps}{\varepsilon}
%\newcommand{\vare}{\varepsilon}
\newcommand{\vphi}{\varphi}
\newcommand{\vsigma}{\varsigma}
\newcommand{\vrho}{\varrho}
\newcommand{\vpi}{\varpi}
\newcommand{\tO}{\widetilde{O}}
\newcommand{\tOmega}{\widetilde{\Omega}}
\newcommand{\tTheta}{\widetilde{\Theta}}
\newcommand{\field}{\ensuremath \Bbbk}
\newcommand{\reals}{\ensuremath \mathbb{R}}

\newcommand{\client}{\mathsl{client}}
\newcommand{\server}{\mathsl{server}}
\newcommand{\TTP}{\procfont{TTP}}
\newcommand{\rec}{\mathsf{rec}}
\newcommand{\doc}{\mathsf{doc}}
\newcommand{\pred}{\mathsf{pred}}
\newcommand{\scrub}{\mathsf{scrub}}

\newcommand{\ODS}{\schemefont{ODS}}
\newcommand{\commit}{\mathsf{Commit}}
\newcommand{\query}{\mathsf{Query}}
\newcommand{\DB}{\mathsf{DB}}


\newcommand{\sanjam}[1]{\textcolor{red}{Sanjam: #1}}
\newcommand{\moh}[1]{}%{\textcolor{red}{Moh: #1}}
\newcommand{\adam}[1]{\textcolor{red}{Adam: #1}}
\newcommand{\ben}[1]{\textcolor{red}{Ben: #1}}

%---
\newtheorem{definition}{Definition}[section]
\newtheorem{question}{Question}[section]
\newtheorem{fact}{Fact}[section]
\newtheorem{lemma}{Lemma}[section]
\newtheorem{corollary}{Corollary}[section]
\newtheorem{theorem}{Theorem}[section]
\newtheorem{claim}{Claim}[section]
\newtheorem{assumption}{Assumption}[section]
\newtheorem{proposition}{Proposition}[section]
\newtheorem{hypothesis}{Hypothesis}[section]
\newtheorem{observation}{Observation}[section]
\theoremstyle{remark}
\newtheorem{construction}{Construction}[section]
\newtheorem{remark}{Remark}[section]
\newtheorem{example}{Example}[section]

\newcommand{\pone}{\mbox{$P_1$}}
\newcommand{\ptwo}{\mbox{$P_2$}}

\renewcommand{\S}{{\cal S}}
\newcommand{\F}{{\cal F}}
\newcommand{\G}{{\mathbb{G}}}

\newcommand{\vecx}{{\mathbf{x}}}
\newcommand{\vecy}{{\mathbf{y}}}

\newcommand{\numcorr}{{\mathsf{ncorr}}}
\newcommand{\numencq}{{\mathsf{nencq}}}
\newcommand{\numkeyq}{{\mathsf{nkeyq}}}


\renewenvironment{proof}{\noindent{\bf Proof:~~}}{\qed}
\newcommand{\BPF}{\begin{proof}} \newcommand {\EPF}{\end{proof}}
\newenvironment{proofsketch}{\noindent{\bf Proof Sketch:~~}}{\qed}
\newcommand{\BPFS}{\begin{proofsketch}} \newcommand {\EPFS}{\end{proofsketch}}

%\def\qed{\quad\blackslug\lower 8.5pt\null\par}


\newcommand{\BPR}{\begin{myprotocol}}   \newcommand{\EPR}{\end{myprotocol}}
\newcommand{\ourfigg}[5]{
{\begin{figure}[#4]
\begin{center}
\framebox[\width][c]{
    \small
    \hbox{\quad
    \begin{varwidth}[c]{0.9\textwidth}
    %\begin{center}
    \begin{myfigure}
    [#1]
    \label{#2}
    \end{myfigure}
    %\end{center}
    \vspace{-3ex}
    #5
    \end{varwidth}
    \quad}
    }
    \begin{center} #3 \end{center}
    \vspace{-6ex}
\end{center}
\end{figure}
} }


% USAGE: \ourfig{TITLE}{LABEL}{CAPTION}{BODY}
\newcommand{\ourfig}[4]{\ourfigg{#1}{#2}{#3}{htb}{#4}}

\newcommand{\prott}[5]{
{\begin{figure}[#4]
\begin{center}
\framebox[\width][c]{
    \small
    \hbox{\quad
    \begin{varwidth}[c]{0.9\textwidth}
    %\begin{center}
    \begin{myprotocol}
    [#1]
    \label{#2}
    \end{myprotocol}
    %\end{center}
    \vspace{-3ex}
    #5
    \end{varwidth}
    \quad}
    }
    \begin{center} #3 \end{center}
    \vspace{-6ex}
\end{center}
\end{figure}
} }

% USAGE: \prot{TITLE}{LABEL}{CAPTION}{BODY}
\newcommand{\prot}[4]{\prott{#1}{#2}{#3}{htb}{#4}}

\newcommand{\view}{{\sf view}}
\newcommand{\trans}{{\sf trans}}
\newcommand{\com}{Z}
\newcommand{\cecom}{{\sf CECom}}
\newcommand{\mxcom}{{\sf MXCom}}
\newcommand{\mxzk}{{\sf MXZK}}
\newcommand{\nmmxcom}{{\sf NMMXCom}}


\newcommand{\crsgen}{{\sf CRSGen}}
\newcommand{\crs}{{\sf CRS}}


\newcommand{\Commit}{{\sf Com}}


\newcommand{\scheme} {{\mathcal{S}}}

\newcommand{\siggen} {{\sf SigKeyGen}}
\newcommand{\sig} {{\sf Sig}}

%\newcommand{\size} {{\sf SIZE}}
\newcommand{\state} {\mathrm{state}}

\newcommand{\idx} {\mathrm{index}}

\newcommand{\kdm}{{\scriptscriptstyle\mathrm{KDM}}}
\newcommand{\skdm}{{\scriptscriptstyle\mathrm{SKDM}}}
\newcommand{\cpa}{{\scriptscriptstyle\mathrm{CPA}}}
\newcommand{\new}{{\scriptscriptstyle\mathrm{NEW}}}


\newcommand{\rsetup}[1]{\R({\sf setup}, #1 ) }
\newcommand{\rquery}[1]{\R({\sf query}, #1 ) }
\newcommand{\rchall}[1]{\R({\sf challenge}, #1 ) }
\newcommand{\rfinal}[1]{\R({\sf final}, #1 ) }
\newcommand{\iO}{i\mathcal{O}}

\newcommand{\PRF}{{\mathsf{F}}}
\newcommand{\PRFGen}{\mathsf{PRF{.}Gen}}
\newcommand{\PRFPunc}{\mathsf{PRF{.}Punc}}
\newcommand{\seed}{\ensuremath{{K}}}
\newcommand{\secpar}{\secparam}
\newcommand{\Adv}{\mathcal{A}}
\newcommand{\rsample}{\gets}

\newcommand{\tf}{\mathrm{tf}}
\newcommand{\idf}{\mathrm{idf}}
\newcommand{\df}{\mathrm{df}}
\newcommand{\p}{\mathrm{P}}



\date{}
\begin{document}

\maketitle
%\vspace{-20mm}
%\begin{abstract}
%ent results have resolved functional encryption for all circuits. Does this solve most problems for functional encryption. We think not. In this paper we present the future vision for functional encryption. Our vision for functional encryption is to realize encryption systems which (at least asymptotically) approach the efficiency levels approached in insecure solutions.
%\end{abstract}
\vspace{-22mm}

%!TEX root = dbproposal.tex

\section{Introduction}

The importance of collecting and storing data is universal, with use cases in governmental~\cite{Powers2014}, commercial~\cite{Linoff:2002:MWT:560274,insightdata}, and personal sectors~\cite{Mons2011}.  Tremendous value can be extracted from data, enabling better decisions, improved health and economic growth.

This data is stored in database systems which have a rich history in both academia and the commercial spaces.  The volume of this data is growing exponentially and is outpacing organizations' ability to manage and
organize this data.  Organizations are turning to external cloud providers
to manage their data needs.  
%
Nation-state actors target other governments' systems, corporate repositories, and individual data for espionage and competitive advantages~\cite{apt1}.   Attacks occur against both government~\cite{CyberAttacksOPM} and commercial~\cite{CyberAttacks,gressin2017equifax} datasets.

The natural response to this risk is to encrypted data before outsourcing. 
%
However, employing encryption comes at the cost of disabling the cloud server
from {\em quickly processing the data in the plaintext form and answering
complex queries from the client}. 
%
Ideally, we could use more sophisticated cryptography to {\em create databases
capable of efficiently answering a client's queries without revealing
information to the cloud server}.  


Secure outsourced database search use advanced cryptography to achieve this goal.  This field encompasses a variety of cryptographic techniques, including property-preserving encryption or PPE~\cite{EC:PanRou12}, searchable symmetric encryption or SSE~\cite{CCS:CGKO06}, private information retrieval by keyword~\cite{EPRINT:ChoGilNao98}.  

This research has largely split into two research threads: PPE which emphasizes background compatibility and use of legacy database management systems and searchable encryption which emphases security.  PPE creates symmetric encryption techniques
compatible with an unprotected database. Examples include deterministic
encryption~\cite{C:BelBolONe07} which can answer equality queries and order-preserving encryption~\cite{C:BolCheONe11,EC:BCLO09}
which can answer range queries. Academic teams, start-up companies (including Bitglass, Ciphercloud, Crypteron, PreVeil, Skyhigh, ZeroDB) and Fortune
500 companies (including Microsoft's SQL Server 2016 and Azure and Google's Encrypted BigQuery)  offer variants of property-preserving encryption.
%Recent attacks against these schemes indicate even ideal property-preserving
%encryption can't be secure for many applications.
%
SSE schemes can be viewed as starting from secure multi-party computation and
optimizes solutions for common database tasks. This approach requires redesign
of database indexing mechanisms and results in better security at the cost of
decreased efficiency. 

These systems have been implemented at moderate scale.  Both property-preserving solutions~\cite{EPRINT:PodBoePop16,CACM:PRZB12} and searchable encryption~\cite{SP:PKVKMC14,SP:FVKKKM15,C:CJJKRS13,CCS:JJKRS13,NDSS:CJJJKR14,ESORICS:FJKNRS15,RSA:IKLO16} solutions have been tested on datasets with billions of records.

%
%  Folks--we need to be careful here.  DBMS functions are much broader than search; there's transformation and presentation, access control, backup, recovery management, etc., and we're talking about none of that here.


\subsection{Use Cases}
Emerging databases include large scale graph databases, analytic databases, and
biometric databases. 
\begin{enumerate}
\item Many data sets are naturally interpreted as large but sparse graphs.  Examples include social networks including online networks such as Facebook and Twitter and more traditional communities such as academic co-authorship and the co-stardom network.  In addition, this type of network is crucial for doing large scale analysis of Internet properties that rely on connections between nodes.  Example algorithms for these graphs involve computing triangles (sets of nodes $\{a,b,c\}$ where all pairs are close according to a metric), shortest path algorithms, network diameter, and degree distribution.
\item  Increasingly, databases are not asked to return subsets of data but rather derive statistics and analytics about the stored data.  Machine learning as a service has emerged as a business model for large and small companies~\cite{mlservice}.  Many machine learning algorithms depend on computing distance between points. Linear regression finds the line that minimizes the sum of distances between the line and each data point.  Similarly, the first principal component minimizes the sum of distances between the selected line and the dataset~\cite{wold1987principal}.  The remaining principal components are defined similarly subject to being orthogonal to previously defined principal components.
\item Biometric databases have long been used for identification of criminals, the FBI has long held a fingerprint database with hundreds of millions of records~\cite{brislawn1996fbi}.  Increasingly, countries are using biometrics as an identifier for citizens, linking biometrics with unique identifiers.  For example, the Aadhaar system in India links biometrics with a unique 12 digit number with over 1 billion numbers issued~\cite{daugman2014600}.  Increasingly, passports are biometric enabled~\cite{stanton2008icao}.The fundamental operation is these databases is comparing the distance between a target point $a$ all stored points. Biometric databases return the nearest match $b^*$ if the distance $d(a,b^*)$ is less than some defined threshold.  
\end{enumerate}

These databases share a fundamental operation: {\em
computing distance/proximity of tuples of points}. The proposed research will
consider proximity queries as a case study and give constructions supporting
these queries. 


\subsection{Inadequacy of Prior Work}

In 2000, Song, Wagner, and Perrig provided the first scheme with communication proportional to the description of the query and the server performing (roughly) a linear scan of the encrypted database~\cite{SP:SonWagPer00}.  Since that time there has been tremendous work and both PPE and SSE approaches can handle much of SQL and NoSQL queries for billions of records.  However, neither approach is capable of handling inherently geometric data that is prominent in many current applications. (There is some work on computing shortest path in networks~\cite{CCS:MKNK15})  Furthermore, each approach also has a second weakness:
\begin{enumerate}
\item PPE has been subject to a number of leakage-abuse attacks that show that order-preserving encryption (and sometimes deterministic encryption) cannot be safely used in most cases~\cite{CCS:NavKamWri15,CCS:CGPR15,CCS:KKNO16,CCS:PouWri16,CCS:GMNRS16,EPRINT:GSBNR16,EPRINT:ZhaKatPap16}.
\item SSE offers more limited functionality and efficiency than PPE based solutions.  The fastest SSE based solutions report overhead of roughly 300\%~\cite{C:CJJKRS13,CCS:JJKRS13,NDSS:CJJJKR14,ESORICS:FJKNRS15} while PPE based solutions report overhead of around 30\%~\cite{CACM:PRZB12}.  We are only aware of a single SSE solution that can handle JOIN statements~\cite{EPRINT:KamMoa16}.  As of this writing, the information  leaked by this scheme is not clear.  Recall that these solutions usually replace the entire database software stack with a custom cryptographic approach.  We posit that the lack of backward compatibility and difficulty of managing these systems also hurts industry adoption.  Industry adoption has been dominated by PPE solutions.
\end{enumerate}

It seems unlikely the academic community can prevent the deployment of PPE systems.  It is imperative to ask if it is possible to design an approach which retains the benefits of PPE including backward compatibility, use of legacy software, improved visibility of data processing.  The core of our approach is to only overload operators (for comparison, equality, proximity) using interactive protocols but allow databases systems to operate ``unencrypted'' otherwise.  
We do not address approaches based on fully-homomorphic encryption or functional encryption.  These solutions are too slow to be used for the scale of current data sets.


\subsection{Proposed work.}
Our goal is to create a third approach to achieving secure outsourced databases: restricting cryptographic operations to database operators (used to create index structures) but allowing these operators to call interactive protocols. The proposed research will
consider proximity queries as a case study and give constructions supporting
these queries. 

To show the promise of this approach, we will first show that for many metric
spaces, we first show the ideal object for distance-preserving encryption is rather weak encouraging investigation of distance-revealing and approximate distance revealing encryption.  We will then show that interactivity can greatly increase the security of this construction, similar to a recent interactive version of order-preserving encryption~\cite{CCS:RACY16}. The resulting constructions will have
better security than the property-preserving encryption schemes and better
efficiency than MPC-based schemes. In particular, the proposed approach will
use standard database indexing mechanisms but (interactively) involve the
client to help with sensitive operations. 
In summary, the main proposed contributions are:
\begin{itemize}
\item Constructions of PPE designed for proximity queries
\begin{enumerate}
\item Analysis of security provided by distance-preserving encryption for common metrics
\item Design of distance-revealing encryption
\item Design of approximate distance-revealing encryption and equivalence with distance-comparison revealing encryption.
\end{enumerate}
\item Use of interactivity to achieve added functionality:
\begin{enumerate}
\item Proximity using Partial Distance-Preserving Encryption
\item Substring Search using Partial Suffix Tree Encryption
\item Improved efficiency of MPC approaches in 2 party setting
\item Improved dynamism for interactive approaches
\end{enumerate}
\item Use of interactivity to achieve improved security:
\begin{enumerate}
\item Modular partial order-preserving encryption
\item Forward security for POPE
\item Merge of Ostrovsky and POPE
\end{enumerate}
\end{itemize}


\paragraph{Intellectual merits.}  
Encrypted databases and searchable encryption has a rich history rooted in the
design of oblivious random access machines.  The current landscape is littered
with approaches with uncertain security or inadequate functionality.  Recent
leakage inference attacks have taught us the impact of leakage and forward
security.  At the same time, the rapid deployment of property-preserving
techniques has reinforced the importance of simplicity, efficiency and backward
compatibility.  These two developments inform the core of our approach: using
interactivity to strengthen property-preserving encryption.

\paragraph{Broader impacts.}
The confidentiality of data is a core societal tenant.  Deployed encrypted
databases provide little security and may even hurt by providing a false sense
of security.  There is a tremendous need for research in this area to
understand the tradeoffs between security, functionality, and efficiency.

The PIs are committed to broad dissemination of research material.  The PIs
have participated in large scale evaluation of searchable encryption, working
on both constructions and attacks.  Furthermore, the PIs have open-sourced
previous work including work on searchable encryption.  Lastly, all PIs are
dedicated to engaging with undergraduates, with two co-PIs being at an
undergraduate only institution, the US Naval Academy.  





%!TEX root = dbproposal.tex

\section{Component A:  Property-Preserving Encryption for Proximity}

In spatial databases, nearest neighbor queries (\emph{e.g.}, finding the closest soldier in the field) and clustering queries are pervasive.  Algorithms for executing these query types need to perform the fundamental operation of \emph{distance comparison} between points in the database, \emph{i.e.}, determining which of two candidates points are closer to a target point.  Similarly, in large scale biometric databases the fundamental operation is comparison of the distance between a target point and all stored points. \ben{need examples of databases, india, other countries?}  There is some work on computing distance on encrypted data (reviewed below).  The work does not evaluate the fundamental soundness of the techniques and does not work for metrics that are necessary for deployed databases.

There are two common techniques used in proximity graphs.  The first technique relies on computing the distance between two points $a$ and $b$.  This distance can be granular for example $d(a, b) =15$ according to a metric $d$ or course $a$ and $b$ are either equal, close, far, or neither.  Usually, users expect equal, close, far, and near to function as a metric.  Thus, for the time being we consider a functionality that allows queries of the type $d(a,b)$.  We call property-preserving encryption that achieves this functionality \emph{distance revealing encryption} or DRE.

The second technique does not allow explicit computation of the distance between $a$ and $b$.  Instead it takes a triple of points $a,b$ and $c$ and outputs the bit $d(a,c)<d(b,c)$.  That is, whether $a$ or $b$ is closer to $c$.  This type of computation is frequently used in learning algorithms.  \ben{Adam is this right?}  We call a property-preserving encryption that achieves this functionality \emph{distance-comparison revealing encryption} or DCRE.

Our research in proximity will follow three threads: 1) constructing distance revealing encryption 3) constructing distance comparison revealing encryption and 3) approximate distance revealing: connecting distance revealing and distance comparison revealing encryption.

\subsection{Component A-I: Constructing Distance Revealing Encryption}
\paragraph{Prior Work}
The notion of distance-revealing encryption has been implicitly studied in prior work.  The most relevant This work of Boldyreva and Chenette appeared at FSE in 2014~\cite{boldyreva2014efficient}.  It is the only work I'm aware of that comes from the theoretical crypto community.  

\paragraph{Summary of Techniques}  This paper can primarily be viewed as contributing to understanding the feasibility of this problem.  They consider a general closeness metric where two items can be ``close'', ``near'', or ``far.''  A fuzzy SSE scheme should be required to return all close records and no far records.  Using this general notion they construct a scheme very similar to Li et al.~\cite{li2010fuzzy}.  However, they pad the number of neighbors so this is not revealed to the server.  They view the problem as constructing a graph of what items in the metric space are close.  They then insert into index all edges of this graph (the scheme of Li et al. can be viewed as inserted neighboring vertices).  Then searching is performed by retrieving all indicent edges of the search term.

The work shows that for an arbitrary closeness graph it is necessary to have a ciphertext length proportional to the maximum degree of this graph.  

Their second contribution is to relax their definition to allow leakage.  They say they are going to reveal some local structure.  They then have a construction using lattices.  Basically they map the input point to the nearest lattice point.  The information that is revealed is exactly the offset to the lattice point.  This is very similar to what is done in the basic fuzzy extractor construction~\cite{DBLP:journals/siamcomp/DodisORS08}.

The work also briefly mentions LSH but this is covered better in other works.

\paragraph{Security}  Their general construction for closeness domains appears to be relatively secure.  As they mention this construction is unlikely to be secure for all but the smallest distances.  \ben{Looking back I don't understand the sentence above.}  The interpretation of their relaxed definition depends entirely on the nature of this local information.  This originally wasn't thought to be a problem in the fuzzy extractor world.  However, there are distributions where the entire point is determined by exactly the local information that is revealed.

\paragraph{Major Gaps}
This work is a good overview of some of the foundational issues and why this problem might be hard.  Their second contribution appears connected to fuzzy extractors and could be considered for other distance notions.  Their definition is fairly relaxed and is based on the indistinguishability definitions used for things like deterministic encryption.\footnote{There is a lot of detail in the second part of the paper that I didn't follow all the way.  May be worth having a conversation since Adam knows both authors well.}



\subsection{Component A-II: Distance Comparison Revealing Encryption}
We call encryption that supports distance comparison \emph{distance-comparison revealing} encryption (DCRE).  Following the literature on order-revealing vs.~order-preserving encryption, we call the special case where ciphertexts themselves are spatial points \emph{distance-comparison preserving encryption} (DCPE).  We propose to study DCRE and DCPE analogously to the on-going study of order-revealing and order-preserving encryption.  
This leads to the following questions:

\begin{question}
Can we design efficient DCPE?  What security can be achieved by such schemes?
\end{question}


\begin{question}
Can we design efficient DCRE with better security?
\end{question}

To answer the first question, in on-going work we have found that distance comparison preserving functions do not seem to have a nice ``recursive'' property as in the case of order-preserving functions, which was crucially exploited by~\cite{EC:BCLO09}.  However, based on computer experiments, we conjecture that \emph{distance-comparison preserving functions are approximately distance-preserving}.    So far, we have proven this conjecture in one dimension.

%On the one hand, this would be depressing because it would mean that DCPE may not be able to provide much better security than \emph{distance-preserving} encryption, which seeks weak. 
If this conjecture is true, then for the first question we could equivalently turn our attention to the design and analysis of  an \emph{approximately distance-preserving} encryption scheme. 
Distance-preserving functions are easy to characterize geometrically, in terms of a scaling factor plus flips, rotations and reflections.  To approximately preserve distance, we can also  ``perturb'' each image point within a ball of given radius.  
We can show that independent random such perturbations yields a function that, while not strictly DCP, is \emph{approximately} so, and that encrypting via an approximately DCP function still guarantees accuracy of nearest neighbor  search within the approximation.  Moreover, as such perturbations can easily be derandomized, this gives an efficient \emph{approximate} DCPE scheme from PRFs.  Finally, we will conduct a separate analysis in the spirit of~\cite{C:BolCheOne11} to answer the question of what privacy such a scheme provides.  

\subsection{Component A-III: Approximate Distance Revealing Encryption}
\paragraph{Prior Work}
A fuzzy extractor~\cite{EC:DodReySmi04} can be seen as an approximate distance preserving encryption where the distance only preserved around one point which is selected at setup.  However, fuzzy extractors shrink the area around the selected point.  A weaker version called a pseudoentropic isometry~\cite{EPRINT:ABCFG16} was recently constructed by the co-PI (albeit it only for large alphabet Hamming metrics).

GRECS constructs a scheme where multiple points (logarithmic in a graph size) are comparable with input points~\cite{meng2015grecs}.  The authors show that by providing comparison with a random set of points and taking the minimum it is possible to approximate all pairs distance.
Finally, the work on fuzzy searchable encryption by Boldyreva and Chenette can be viewed as building a distance comparison preserving function~\cite{boldyreva2014efficient}.  They allow comparisons between any two points, their approach is based on tags and is designed not to be distance-preserving but only distance comparison preserving.


\textbf{Outline of proposed work}
\begin{itemize}
\item Biometric data is being stored online and is privacy sensitive.
\item There do not exist strong approaches to protect biometric data stored online while allowing search.
\item We will produce searchable encryption that simultaneously protects privacy while allowing search for important use cases.
\item If we don't, data breaches will continue to reveal individuals' sensitive information and this information cannot be recovered or modified.
\item Some tasks:
\begin{itemize}
\item New definitions
\item Tokenizing with LSH to not leak on subfeatures
\item LSH with threshold schemes.
\item Direct indexing mechanism
\item Support more metrics
\item Implement results
\end{itemize}
\item Extend noisy protected search to more scenarios including new definitions and new constructions.  In particular, consider scenarios that require security against the client and provide security constructions.
\item Construct backend direct indexing mechanisms designed for noisy data
\item Tailor noisy, protected search based on near-term applications (iris, fingerprint, photo including face photos).  Consider multiple use cases including: consumer-grade authentication, law enforcement (criminal investigation), identity cards of the future (real-ID), known person, office of personal management stores all fingerprints.
\end{itemize}


\textbf{Open problems discussion:} 
\begin{itemize}
\item Understand the set of LSH and how they can be used for practical noisy sources.
\item Is there any power to interactive solutions?  Everything I've seen is single message each way.
\item Integrate LSH and an m-out-of-n search scheme to mitigate the leakage associated with the Kuzu scheme.
\item Design a stronger primitive than LSH that has some entropy preservation properties so we can talk about what cost there is in revealing when LSH collisions occur.  Adam's student did some work on distance preserving transforms.  Ben is thinking about this as well.  
\item None of these schemes talk about updates.  Is there anything interesting to consider there?
\end{itemize}

\paragraph{Prior Work}

\ben{These are my notes from last year, still need to condense}

Work by Li et al.~\cite{li2010fuzzy,wang2013efficient} that was published at INFOCOM 2010.  

\paragraph{Summary of Techniques}  Here we focus on how they deal with noise.  We ignore the searching index structure in the rest of their work.  This work is compatible with an arbitrary index structure.  The scheme is built on a system that can handle keyword equality.  The basic idea of this paper is to expand noise into a set of possible neighbors.  Three strategies are proposed:

\begin{enumerate}
\item When inserting a keyword, for example `Alice', also insert all of the neighbors into the index structure.  For example if considering Hamming distance of $1$, insert `Blice', `Clice', etc.  This technique results in an index structure size proportional to the number of neighbors.  Furthermore, the number of neighbors is leaked to the server as well documents that have neighboring keywords.  However, search proceeds the same way as in standard equality search.  The authors note that this approach is not tenable for any large amount of noise.
\item The second technique is a hybrid technique where both the inserted keyword set and the searched queries are modified.  Instead of inserting `Alice', `Blice', etc. the following keywords are inserted `*lice', `A*lice', etc.  Then when performing the search the querier with the word `Alica' asks all documents containing one of `*lice', ..., `Alic*', etc.  Note that this approach requires as underlying scheme that can handle disjunctive queries.  As such there may be more leakage associated with this approach.
\item The third approach is to insert word GRAMS.  The idea is to insert all possible nearby substrings of a particular length.  I didn't really follow their discussion well on why this was better than the wildcard based approach.  But this technique seems to be similar to what is being done in conjunctive search system out of IBM research.
\end{enumerate}

\paragraph{Security}
This work never presents a formal definition of security.  They roughly say that the server is allowed to learn ``the outcome and the pattern of search queries.''  They claim to be using the security definition of Curtmola et al.~\cite{curtmola2011searchable}.  For their basic scheme they leak when two documents have the same neighbor which is the same as leaking if they are within twice the noise tolerance.  

\paragraph{Major gaps}
This paper presents a relatively weak efficiency guarantees.  All of the schemes involve exhaustively listing the neighbors in one form or another.  Furthermore, the number of neighbors and the neighborhood pattern of all documents is revealed.

\subsubsection{Efficient Similarity Search over Encrypted Data}
Work by Kuzu, Islam, and Kantarcioglu~\cite{kuzu2012efficient}.  I don't know the reputation of the conference IEEE Conference on Data Engineering.  

\paragraph{Summary of Techniques} The basic idea is to build fuzzy search over a primitive called locality sensitive hashing.  A locality sensitive hash is function that is more likely to have collisions when two inputs are ``close'' in the input space.  It is often used to construct efficient algorithms for solving the nearest neighbor problem~\cite{datar2004locality,slaney2008locality}.  

\begin{definition}
Let $(R,d)$ be a metric space.  A family of functions $\mathcal{H}$ is a $(r_1, r_2, p_1, p_2)$, for $r_1< r_2$ and $p_1 >p_2$, locality sensitive hash if for any $x, y$ the following hold:
\begin{itemize}
\item If $d(x, y) \le r_1$ then $\Pr_{h\leftarrow \mathcal{H}}[h(x) = h(y)] \ge p_1$, and 
\item If $d(x, y) \ge r_2$ then $\Pr_{h\leftarrow \mathcal{H}}[h(x) = h(y)] \le p_2$.
\end{itemize}
\end{definition}

The idea of this work is to sample multiple locality sensitive hashes and insert the results of each hash into the keyword index.  Then the client will query for all results that match each locality sensitive hash (again this requires the ability to perform disjunctive queries).\footnote{More precisely, they build an inverted index for each locality sensitive hash that allows them to retrieve the document identifiers.}  The client locally retrieves results and then restricts to those documents that have a high number of hash matches.  With good probability, these will correspond to those records that were close to the original query.  Note this scheme overcomes the limitations of~\cite{li2010fuzzy} and allows fuzzy searching even in the case of an exponential number of neighbors (assuming a good locality-sensitive hash) is known for the metric space.

\paragraph{Security}  This scheme has several important security drawbacks.  We separate these into client and server concerns.

\begin{itemize}
\item \textbf{Client} There is no security provided against the client.  They consider the two party model.  Notice that the client retrieves all records that match even a single locality sensitive hash.  It is clients responsibility to determine which of these documents are relevant.  \textbf{Possible extension:} This seems a very obvious place for improvement, using an m-out-of-n technique would allow the client to directly retrieve these records.  Would need to worry about number of required hashes and the leakage for this type of query.
\item \textbf{Server} The provided security against the server is quite weak.  They acknowledge that many SSE schemes leak when two documents share the same keyword or feature.  This work projects a single feature to multiple subfeatures and leaks equality of those as well.  This seems crucial to their approach.  Its not clear how much is revealed by learning this partial information.  \textbf{Possible work:} Seems like a natural place to consider inference attacks.
\end{itemize}

\paragraph{Major gaps} As stated above, the provided security is not very strong.  The scheme is fairly space efficient.  It would be interesting to really explore the space of locality sensitive hashing and see what is appropriate.  It would be great to hide the subfeature patterns that are revealed by this scheme.  Note that they actually appear to have implemented this scheme.

\subsubsection{Identification with Encrypted Biometric Data}
This work of Bringer et al.~\cite{bringer2011identification} is a journal version of a work that was presented at ICC~\cite{bringer2009error}.  It uses similar techniques to the work of Kuzu et al.~\cite{kuzu2012efficient} and neither work cites the other.  Given the proximity in time my assumption is that the two works were developed independently.  They also build locality sensitive hashing.  They add a Bloom filter to check membership.  The idea is to insert the result of the LSH into the Bloom filter and then search for each keyword.  Most of their technical work is in combining the Bloom filter and LSH.  Since their definitions don't allow for leakage they use PIR to actually perform the lookups throughout their scheme.  Thus, this scheme is unlikely to actually be implemented.


\paragraph{Security} Unlike the work of Kuzu et al. this work considers three parties, a sender, receiver, and a server.  However, their definitions of security are very strange.  They consider two notions called sender and receiver privacy.  Sender privacy says that the server can't tell which of two records is being inserted.  Receiver privacy is the same, a server can't tell which of two records was retrieved.  So despite considering three parties there is no provided security against the sender or receiver.

\paragraph{Major Gaps}  While this scheme has a stronger notion of security that Kuzu et al. all of the complexity is hidden by using PIR for the actual retrieval.  Furthermore the system still does not consider any notion of security against the data owner or data querier.  There's a lot of things that are just abstracted away.  Would need to figure out these details for a reals scheme.

\subsection{Component A-III:  Property-Preserving Encryption for Graph Data} 

For graph data, shortest path and disease propagation queries are common.  Previous work looks at supporting shortest path queries in the context of searchable symmetric encryption.  Accordingly, we propose to investigate \emph{shortest path revealing  and preserving encryption} (SPRE and SPPE) and \emph{random walk revealing and preserving encryption} (RWRE and RWPE).   We believe that the study of such encryption schemes will lead to interesting questions in graph theory.  We will particularly try to leverage work on differentially private graph sanitization. 

Regarding shortest path revealing schemes: 

\begin{question}
Can we design efficient SPPE?  What security can be achieved by such schemes?
\end{question}


\begin{question}
Can we design efficient SPRE with better security?
\end{question}

\begin{question}
Can we design ``partial'' SPPE with better security?
\end{question}

And then regarding random walk preserving schemes:

\begin{question}
Can we design efficient RWPE?  What security can be achieved by such schemes?
\end{question}


\begin{question}
Can we design efficient RWRE with better security?
\end{question}

\begin{question}
Can we design ``partial'' RWPE with better security?
\end{question}




\iffalse
\subsection{Component A-II:  Property-Preserving Encryption for Time Series Data} 

In time series data one is often interested in correlations and anomalies.  Accordingly, we propose to look at \emph{correlation revealing and preserving encryption}  (CRE and CPE) and \emph{anomaly revealing and preserving encryption} (ARE and APE).  In other words, in the ``preserving'' case we are interested in perturbing statistical data in a way that preserves statistics or the fact that a point is an anomaly.  

Regarding correlation revealing schemes
\begin{question}
Can we design efficient CPE for correlations of interest?  Which correlations should we target?  What security can be achieved by such schemes?
\end{question}


\begin{question}
Can we design efficient CRE with better security?
\end{question}

\begin{question}
Can we design ``partial'' CPE with better security?
\end{question}


Regarding anomaly  evealing schemes
\begin{question}
Can we design efficient APE?  How should anomaly thresholds be set?  What security can be achieved by such schemes?
\end{question}


\begin{question}
Can we design efficient ARE with better security?
\end{question}

\begin{question}
Can we design ``partial'' APE with better security?
\end{question}

\subsection{Component A-III:  Property-Preserving Encryption for Graph Data} 

For graph data, shortest path and disease propagation queries are common.  Previous work looks at supporting shortest path queries in the context of searchable symmetric encryption.  Accordingly, we propose to investigate \emph{shortest path revealing  and preserving encryption} (SPRE and SPPE) and \emph{random walk revealing and preserving encryption} (RWRE and RWPE).   We believe that the study of such encryption schemes will lead to interesting questions in graph theory.  We will particularly try to leverage work on differentially private graph sanitization. 

Regarding shortest path revealing schemes: 

\begin{question}
Can we design efficient SPPE?  What security can be achieved by such schemes?
\end{question}


\begin{question}
Can we design efficient SPRE with better security?
\end{question}

\begin{question}
Can we design ``partial'' SPPE with better security?
\end{question}

And then regarding random walk preserving schemes:

\begin{question}
Can we design efficient RWPE?  What security can be achieved by such schemes?
\end{question}


\begin{question}
Can we design efficient RWRE with better security?
\end{question}

\begin{question}
Can we design ``partial'' RWPE with better security?
\end{question}
\fi

%\section{Component B:  Property-Preserving Encryption for Time Series Data} 

In time series data one is often interested in correlations and anomalies.  Accordingly, we propose to look at \emph{correlation revealing and preserving encryption}  (CRE and CPE) and \emph{anomaly revealing and preserving encryption} (ARE and APE).  In other words, in the ``preserving'' case we are interested in perturbing statistical data in a way that preserves statistics or the fact that a point is an anomaly.  

Regarding correlation revealing schemes
\begin{question}
Can we design efficient CPE for correlations of interest?  Which correlations should we target?  What security can be achieved by such schemes?
\end{question}


\begin{question}
Can we design efficient CRE with better security?
\end{question}

\begin{question}
Can we design ``partial'' CPE with better security?
\end{question}


Regarding anomaly  evealing schemes
\begin{question}
Can we design efficient APE?  How should anomaly thresholds be set?  What security can be achieved by such schemes?
\end{question}


\begin{question}
Can we design efficient ARE with better security?
\end{question}

\begin{question}
Can we design ``partial'' APE with better security?
\end{question}



\section{Allowing Interaction}
\label{sec:interactive}
%!TEX root = dbproposal.tex

\subsection{Extensions to Property-Preserving Encryption}
The direct use of property-preserving encryption has a mixed history
with leakage attacks showing that deterministic and order-preserving
encryption reveal the entire stored dataset for many applications (see
work of the co-PI for an overview of leakage
attacks~\cite{SP:FVYSHG17}).

Despite these attacks, the ease and speed of configuring and using PPE
without replacing the entire software stack has encouraged its momentum
in the commercial sector. A natural question is, \emph{can we maintain
the benefits of PPE while avoiding these attacks?}

In this section, we investigate solutions that keep intact
the design principal of PPE: the only place that the database should
have to change is the comparison operator (equality, comparison, or
distance).  The database should still be able use standard indexing
mechanisms.  We note it is possible to override this operator to be
interactive and require help from a client without altering the overall
indexing structure.

\paragraph{Prior work.}
We are aware of two main works that follow the approach of PPE with added interactivity.  Two of the co-PIs recently introduced a new cryptographic approach, called POPE, to
support range queries over encrypted data, providing stronger security than
previous order-preserving encryptions~\cite{CCS:RACY16}.  In contrast to
existing OPE schemes, the server builds a novel indexing structure called a
POPE tree, in which each node has a {\it unsorted buffer} and a sorted list of
elements.  Thanks to this, the scheme can perform {\it lazy indexing, by
sorting values only when necessary}. In particular, on each range query, the
scheme sorts the part of the tree that forms the boundary of the search,
leaving much of the data untouched.

The other work is Arx due to Poddar, Boelter, and Popa~\cite{EPRINT:PodBoePop16}.  The
Arx protocol builds an index for answering
range queries without revealing all order relationships to the
server. The index stores all encrypted values in a binary
tree so range queries can be answered by traversing this
tree for the end points. Using Yao's garbled circuits, the
server traverses the index without learning the values it is
comparing or the result of the comparison at each stage.  Each garbled circuit can only be used once for a single comparison.  The client and server work together to create new garbled circuits (and OT pairs).

\paragraph{POPE for similarity of high-dimensional data.}
Our main approach is developing PPE that operates using just-in-time
advice from the client.  In POPE, this took the form of asking the client to
sort a small number of nodes to build out a tree and comparing ciphertexts only
with those nodes.  This limited leakage to comparisons between the dataset and
these nodes, reducing the leakage from a quadratic number of
comparisons to only linear.

Our first task will be extending the POPE
paradigm to high dimensional data.  The approach uses random hyperplanes
to form a partially-sorted search tree over multidimensional keys.
Random hyperplanes
have previously been used in the concept of random projection trees and
locality-sensitive hashing~\cite{STOC:DasFre08,charikar2002similarity}.  The
idea is as follows:

\begin{enumerate}\setlength\itemsep{0em}
\item The server initially stores an unsorted buffer of the entire dataset.
\item The client initiates a search for items that are similar to $a$.
\item The server asks the client to split the root node.
\item The client generates a random hyperplane $x$ and splits 
  the stored elements $b_i$ based on whether $b_i$ is above $x$ or not.
  (In the Euclidean
    space, this means checking whether
    $\langle x^T,b_i \rangle$ is positive.)
\item The client and server repeat the process with the relevant subtree
  until reaching an upper bound on the size of an unsorted leaf node.
\item The leaf node containing $a$ is returned to the client.
\end{enumerate}

The problem with this preliminary approach is a substantial chance that
nearby items will end up in different subtrees and therefore be
missed in the returned set.
To deal with this
problem, we propose to have the client can select a \emph{collection}
of hyperplanes $x_1,..., x_k$ and
split the tree based on $\sum_{i=1}^k \sign( \langle x_i^T, b \rangle )$.  This
technique allows us to control the probability that $a$ and $b$ will be denoted
as far (lying in different subtrees) when they are close (using tail bounds for
the binomial distribution). % Furthermore, it also creates a larger number of children for
%each node pushing from a binary to a B$^+$-tree.

\paragraph{Edit distance using partial suffix-tree encryption.}
Our second task is to build proximity search for edit distance.  For this
approach, we will build a partial version of a suffix tree which is often used
in string algorithms~\cite{mccreight1976space}.  Prior work by Chase and
Shen~\cite{chase2015substring} used an an encrypted suffix tree to answer
substring queries.

Our idea is to apply the online algorithm of a suffix-tree construction by
Ukkonen~\cite{Ukkonen95} and to build the suffix tree just-in-time as in
the POPE protocol. Again, the advantage will be that leakage is limited
to a ``need-to-know'' basis relative to the queries performed, rather
than leaking information about the entire dataset up-front.

The client will build a single level of the
suffix tree.  When the client searches for strings that are close $a$ they will
traverse the partially constructed suffix tree with the client/server
interactively building out the tree as necessary.  This approach will require
augmentations to a traditional suffix tree as the original searchable string
must be stored and queryable to build the tree on demand.  Balancing the speed
and privacy of querying the original string represents an important tradeoff in
this approach.

\subsection{Extending BlindSeer}

\paragraph{Overview of BlindSeer.}
Roughly speaking, BlindSeer~\cite{SP:PKVKMC14,SP:FVKKKM15} has three main
players. The {\em server} $\sf S$, who holds the DB and hands an encrypted copy
of the DB to a third party, called the {\em index server} $\sf IS$. The server
$\sf S$  also builds an encrypted Bloom filter (BF) tree index to the DB and
sends it to $\sf IS$. The {\em client} $\sf C$ sends search queries to $\sf IS$
and obtained encrypted results, to each of which $\sf C$ obtains the decryption
key secret-shared by $\sf S$ and $\sf IS$. 

BlindSeer provides the following privacy guarantees:
\begin{itemize}
\item Query privacy: $\sf S$ is not involved with the search protocol at all,
the client's queries are hidden from $\sf S$. Moreover, $\sf IS$ holds only the
{\em encrypted} database, the client's queries remain private to a great
degree. 

\item Data privacy: $\sf IS$ deals with only the encrypted database, so the
records are hidden from $\sf IS$. Moreover, $\sf IS$ returns only the records
that satisfy the client query, and therefore, $\sf C$ cannot arbitrarily access
all the records in the database.  
\end{itemize}

\subsubsection{Proposed Work}

\paragraph{Adapting Blindseer to a simpler cloud setting.}
When a database is outsourced, the database owner is often the client itself.
In this case, we don't need to worry about data privacy against the client, and
we can significantly simplify the architecture of the Blindseer system. 

\begin{itemize}
\item {\em Much more efficient setup and record retrieval.}
The client plays as both $\sf S$ and $\sf C$ in the original setting. The
client can just hold one symmetric key to decrypt all the records. Before, in
order to use secret-sharing, a slow public-key encryption scheme and a random
shuffling must be introduced in the setup and record-retrieval phase, none of
which we need in the current setting. 

\item {\em MPC is not necessary.} In the original Blindseer system, in order to
hide from $\sf C$ the BF data in each node of the BF tree, $\sf C$ and $\sf IS$
have to execute costly MPC computation. In the current setting, the client can
simply ask for the encrypted data from the cloud server and decrypt it. 

\item {\em Dynamic record insertion.} In the original BlindSeer, it was
difficult to insert the BF index to the BF tree dynamically, since it will
cause the BF tree to have a more complicated encryption structure and the MPC
computation to be much more costly. In the current setting, since MPC is not necessary, this opportunity of adding BF indices directly to the BF tree is widely open. 
\end{itemize}

We believe that it is important to provide an efficient solution for various
use case scenarios. Given the appropriate use case, simplified version provide
the solution running {\em at least x5 faster than the original system}.  

\paragraph{Supporting a rich set of functionalities.}
We note that BlindSeer provides a rich set of functionality including
conjunctions and range query. 

*** to do: write about how to support conjunctions and range query?

Based on this, we can support various types of queries. 
\begin{itemize}
\item Spatial data. A query that searches for the points $(x, y)$ in DB that
belongs to a 2D square defined by the top-left point $(x_1, y_1)$ and the
bottom-right point $(y_1, y_2)$ can be described with range queries and conjunctions as follows:
  $$ x_1 \le x \le x_2 ~~{\sf AND}~~~ y_1 \le y \le y_2 $$

\item ***to do: What else? 
\end{itemize}




\section{Strengthening Security}
\label{sec:improving security}
%!TEX root = dbproposal.tex

\subsection{Enhancing security of POPE}
In this section, we propose novel security improvements to the interactive approaches proposed in the previous section.  Many of these defenses are designed to defeat leakage inference attacks.  The first approach is \emph{modularity} in the
sense of modular order-preserving encryption (M-OPE)~\cite{C:BolCheONe11}.  The
idea of M-OPE is to apply a secret random offset modulo the largest possible
message to a message before encrypting it (the secret random offset is chosen
once and fixed in the secret key), so that everything gets ``shifted.''
Since attacks still seem to apply to M-OPE, we propose to investigate more
fine-grained modularity as a defense. 

\paragraph{Random offset for each digit.}
We propose \emph{digit-modular} OPE (DM-OPE) where there is a
secret modular offset applied to each digit.  The base in which the data is
written could even itself be secret.  It becomes more complicated to make range
queries with DM-OPE, as it requires an exponential number of queries in the
number of ``wrap around'' digits in the query.  However, we propose to
investigate approximating the queries efficiently (with some false positives
that the client can filter out).

%\subsection{Enhancing security of POPE}
\paragraph{Forward Security} Leakage attacks are particularly problematic when the adversary is able to correlated leakage from multiple queries.  Forward security searchable encryption can decouple the adversary's leakage and force them to execute their attack with less information.  

POPE natively has forward security (i.e., updating an element
don't leak information about the other elements), although it was not considered in the publication~\cite{CCS:RACY16}.  This is because the scheme doesn't maintain any other
index structure except the POPE tree. So, to update an element, one can simply
delete the element from the node it belongs and insert an updated element to
the unsorted buffer of the root. 

In this proposal, we plan to investigate whether we can further reduce the
leakage of the POPE scheme. 
%
%\begin{question}
%Can we reduce the leakage for the POPE scheme?
%\end{question}
In POPE, each search query leaks the ordering  for the following reasons: 

\begin{itemize}\setlength\itemsep{0em}
\item The tree is a search tree. For example, all elements in the
  left sub-tree are smaller than every element in the right subtree.
 
\item Once a ciphertext is inserting into a POPE tree, it never changes. This {\em
  deterministic} nature allows the attacker to trace when the ciphertext came
  in the tree, and how it was brought down to some leaf node.  
\end{itemize}

We plan to address these issues as follows. To address the first issue, for
each POPE node, the order of the links to its children may be shuffled. Since
POPE is an interactive protocol, when a tree node is created, we can slightly
modify the original protocol so that the client additionally change the order
of the links with a randomly selected permutation.\footnote{Similar ideas were used in the protocol of Ishai et al.~\cite{RSA:IKLO16} who use MPC and private information retrieval to hide tree traversal.}
%
Search can still be performed correctly even with this
modification, since the interactive guidance of the client can help the server
traverse the tree nodes correctly. 
    
To address the second issue, we observe that when the ciphertexts in an
unsorted buffer are streamed down to lower-level buffers during a search query,
the client first {\em decrypts} them in order to indicate the correct unsorted
lower-level buffer to which the ciphertexts should move. This procedure can be
easily augmented so that the client can stream the {re-reandomized cipehrtexts}
(instead of using the original ones deterministically) to lower-level unsorted
buffers. This way, along with the shuffling idea above, we can significantly
hide information about the location of lower-level buffers to which ciphertexts
have been streamed down. 

%\paragraph{Leakage over time.}
While POPE allows each search query to {\it gradually} leak the ordering
information of the underlying plaintexts, {\em most} of the ordering
information will be leaked over many queries. This begs the following question: 

\begin{question}
Can we still maintain a sufficient number of incomparable pairs of elements, even after
  many range queries have been performed?  
\end{question}

One approach is to leave more ciphertexts in an unsorted buffer without
affecting the performance too much. For example, a ciphertext in an unsorted
buffer of the root node is incomparable with all the other ciphertexts in the
entire tree.  Moreover, in the POPE construction, when an unsorted leaf buffer
is full (i.e.,  containing more than $L$ ciphertexts), where $L$ is a threshold
parameter, new $L$ leaf nodes are created, and the ciphertexts are partitioned
into $L$ newly created buffers. This procedure greatly reduces the number of
ciphertexts in unsorted buffers and affects the security; in the worst case,
some unsorted buffer will have only a single ciphertext. With more careful
partitioning, we can maintain the number of incomparable pairs to be reasonably
high even over many queries. 


Another possible direction is taking advantage of ORAM (Oblivious Random Acess
Memory). Although they are too slow to used throughout the entire system
containing a large amount of data, we hope that ORAMs can be used effectively
for achieving stronger privacy for the sensitive sub-part of the system (e.g.,
the bottom parts of the POPE tree).  In fact, ORAMs have been used to minimize
the leakage for SSE (symmetric searchable encryption) schemes which support
keyword search over encrypted data~\cite{NDSS:StePapShi14,C:GarMohPap16,RSA:IKLO16}. We will investigate how to incorporate ORAM into POPE trees so that the resulting system enjoys stronger privacy while only
having marginal performance degradation.  


\subsection{Component C-III: Enhancing security of BlindSeer}

In Blindseer, the search query is executed by the client traversing the BF
search tree. In particular, in each node of the tree, the client and the server
execute the following:

\begin{enumerate}
\item For each keyword $\alpha$ in the query, the client computes the hash
function $H(\alpha)$ to identify the BF indices $(h_1, h_2, ..., h_\eta)$ where
$\eta$ is a system parameter. The client sends $(h_1, h_2, \ldots, h_\eta)$ to
the server. 

\item The client and the server execute a protocol so that the client may know
whether the BF bits for the indices $(h_1, h_2, \ldots, h_\eta)$ are set, based
on which the client decides whether to proceed based on the result.
\end{enumerate}

Note this leakage $(h_1, \ldots, h_\eta)$ is per keyword in the query. Based on
this leakage, the server can infer whether two queries contain the same keyword
with decent probability. 


to do: ORAM to hide this leakage



\section{Prior Accomplishments and NSF Support}

\paragraph{Adam O'Neill:}
In his Ph.D.~work, the PI  developed the notions of deterministic encryption~\cite{C:BelBolONe07,Amanatidis2007,C:BolFehONe08,C:BFOR08,TCC:FulNeiRey12} and order-preserving encryption~\cite{EC:BCLO09,C:BolCheONe11} to help enable search on encrypted data with processing time comparable to that for unencrypted data, while providing as-strong-as-possible security guarantees subject to this constraint.
The PI has also worked on  instantiating random oracles~\cite{C:KilOneSmi10,TCC:GoyONeRao11,EC:LewONeSmi13},   aggregate signatures~\cite{CCS:BGOY07,AC:GLOW12}, deniable encryption~\cite{C:OnePeiWat11},  chosen-ciphertext security~\cite{EC:KilMohOne10,PKC:DFMO14}, and  functional encryption~\cite{EPRINT:ONeill10b,C:DIJOPP13, CANS:BelONe13}.   %Furthermore, he collaborated with database faculty George Kollios on refinements to order-preserving encryption~\cite{}.
Since joining Georgetown, he has also been working on  applications of indistinguishability obfuscation~\cite{PKC:DGLOZ16} and on integrating cryptography with emerging applications, such as outsourced database systems using modular order-preserving encryption~\cite{mavroforakis2015modular} and  privacy preserving network provenance using structured encryption~\cite{zhang2017privacy}.


Prior support: ``EAGER: Guaranteed-Secure and Searchable Genomic Data Repositories.'' (PI). Proposal Number 1650419.  2016 - 2017. \$99,9999.
``Program Obfuscation: From Theory to Practice." NSF Research Experiences for Undergraduates Supplement (PI).
Supplement to Award \#IIP-1362046,   2014 - 2019, \$8,000.

\paragraph{Benjamin Fuller:}
In his Ph.D.~work, the co-PI worked on deterministic encryption with Dr. O'Neill \cite{TCC:FulNeiRey12,JC:FulONeRey15}.  His main focus was on cryptography with noise developing new fuzzy extractors~\cite{AC:FulMenRey13,AC:FulReySmi16,EC:CFPRS16}.  Fuzzy extractors can be thought of as a special case of distance preserving encryption where only a single point is comparable.  He then oversaw evaluation and implementation of encrypted search systems at MIT Lincoln Laboratory as part of the IARPA SPAR project~\cite{spar_baa} including BlindSeer codeveloped by Dr. Choi.  Since joining the University of Connecticut in 2016, his work has focused on driving cryptography to practice including authentication and fuzzy extractors~\cite{EPRINT:HFDD17,EPRINT:BKFY17,EPRINT:ABCFG16}, secure outsourced databases~\cite{SP:FVYSHG17}, and multi-party computation~\cite{EPRINT:CunFulYak16}.  

Prior NSF support: not applicable.

\paragraph{Seung Geol Choi:}
The co-PI is mainly interested in achieving privacy in practice. As for the
works directly related to this proposal, he participated in the IARPA SPAR
project~\cite{spar_baa} as a member of the team of Columbia University and Bell
Labs to build the BlindSeer system~\cite{SP:PKVKMC14}, and he also recently
introduced a new system, called POPE, that supports range queries over
encrypted data~\cite{CCS:RACY16}.  

He has also worked on topics related to this proposal such as
ORAMs~\cite{SP:RocAviCho16,NDSS:ACMR17,CCS:RACM17}, secure multi-party
computation~\cite{AC:CEJMY07,TCC:CDMW09,AC:CEMY09,RSA:CHKMR12,TCC:CKKZ12,PKC:CKWZ13,C:CKMZ14,TCC:CKSYZ14},
and various encryption schemes~\cite{TCC:CDMW08,AC:CDMW09,AC:LCLPY13}. 

Prior NSF support: ``RUI: Achieving Practical Privacy for the Cloud.'' (co-PI) Award number 1618269, 2016-2019, \$355K. 

\paragraph{Daniel S.\ Roche:}
This co-PI comes from an algorithms background, having worked
extensively in the area of computer algebra and publishing frequently in
the top venues of that area
\cite{Roc09,Roc11,GR10,GR11,GR11a,GRT10,GRT12,HR10,AGR14,AR14,AGR15}. Recently, his interests have turned to
developing improved algorithms and data structures for ensuring privacy
in remote storage, which is closely related to the topic of this
proposal
\cite{SP:RocAviCho16,CCS:RACY16,NDSS:ACMR17,CCS:RACM17}.

The co-PI has a proven track record of working with undergraduates and
graduate students at other institutions,
including multiple publications from such collaborations
\cite{AGR13,AGR14,AGR15,AR14,KRT15,GR16}.

Prior NSF support: ``AF: Small: RUI: Faster Arithmetic for Sparse
Polynomials and Integers.'' (PI) Award number 1319994, 2013-2016, \$123K.

Prior NSF support: ``RUI: Achieving Practical Privacy for the Cloud.''
(co-PI) Award number 1618269, 2016-2019, \$355K.

\begin{table}[t]
\small
\begin{tabular}{l | c | c | c | c | c | c | c | c | c | c | c | c}
Thrust & \multicolumn{4}{c}{Year 1} & \multicolumn{4}{c}{Year 2} & \multicolumn{4}{c}{Year 3} \\
& Q1 & Q2 & Q3 & Q4 & Q1 & Q2 & Q3 & Q4 & Q1 & Q2 & Q3 & Q4\\\hline
Distance-Preserving Encryption\\
\,\,\,\,Security of Ideal DPE (Fuller) & \multicolumn{4}{c}{\cellcolor{blue!25}}\\
\,\,\,\,DRE (Fuller) & \multicolumn{4}{c}{} &\multicolumn{5}{c}{\cellcolor{blue!25}} \\
\,\,\,\,Approximate DRE (O'Neill) &\multicolumn{6}{c}{\cellcolor{blue!25}} \\
Allowing Interaction \\
\,\,\,\,Euclidean Dist. (Choi) &\multicolumn{4}{c}{\cellcolor{blue!25}} \\
\,\,\,\,Edit Dist. (Roche) & & \multicolumn{6}{c}{\cellcolor{blue!25}} \\
\,\,\,\,Simplify Blind Seer (Choi) & \multicolumn{4}{c}{} & \multicolumn{4}{c}{\cellcolor{blue!25}} \\
\,\,\,\,Blind Seer for Graphs (O'Neill) & \multicolumn{7}{c}{} & \multicolumn{5}{c}{\cellcolor{blue!25}}\\
Improving Security\\
\,\,\,\,Modular POPE (Roche) & \multicolumn{3}{c}{}& \multicolumn{4}{c}{\cellcolor{blue!25}}\\
\,\,\,\,Randomized POPE (Choi) & \multicolumn{7}{c}{} & \multicolumn{3}{c}{\cellcolor{blue!25}}\\
\,\,\,\,Forward Security (O'Neill) & \multicolumn{7}{c}{} & \multicolumn{5}{c}{\cellcolor{blue!25}}\\
\,\,\,\,ORAM \& Blind Seer (Choi) & \multicolumn{6}{c}{}& \multicolumn{6}{c}{\cellcolor{blue!25}}\\
Impl./Eval. (Fuller) & \multicolumn{4}{c}{} &  \multicolumn{8}{c}{\cellcolor{blue!25}}\\
\end{tabular}
\caption{Work schedule}
\label{tab:work}
\end{table}
\section{Schedule and Management Plan}
The work described in this proposal will be performed by a combination of the PIs, graduate students at Georgetown University and University of Connecticut, and undergraduate students at all three institutions.  A collaboration plan is attached as supplementary material.  In this section, we provide a brief overview of the timeline of the proposed research.  For each task, we have identified the lead PI for that task but we expect all tasks to involve collaboration between all PIs.  The work is summarized in Table~\ref{tab:work}.

\clearpage
\bibliographystyle{alpha}
\bibliography{cryptobib/abbrev3,cryptobib/crypto,Bencrypto,dan}





\end{document}
