\documentclass[11pt]{article}

\usepackage{fullpage,varwidth,url}
\usepackage{amssymb}
\usepackage[table]{xcolor}
\usepackage{amsmath,amsthm}
\usepackage{relsize}
\usepackage{color}
%
%    \setlength{\evensidemargin}{-0.2in}
%    \setlength{\oddsidemargin}{-0.2in}
%    \setlength{\textwidth}{6.9in}
%    \setlength{\textheight}{9.5in}
%    \setlength{\topmargin}{-0.35in}
%    \setlength{\headheight}{0in}
%    \setlength{\headsep}{0in}
%    \setlength{\footskip}{0.5in}
\title{{{\Large{SaTC: CORE: Small:  Collaborative: \\Next-Generation Secure Outsourced Databases}}}}
%\author{Sanjam Garg \\University of California, Berkeley \and 	Mohammad Mahmoody \\ University of Virginia\and Adam O'Neill \\Georgetown University}

\usepackage[hidelinks]{hyperref}
\usepackage{cleveref}


%\usepackage{anysize} % to set up document margins.
%\marginsize{1.1in}{1.1in}{1.1in}{1.0in}
%\vspace{-1mm}
\usepackage{times}
\usepackage{etex}
\usepackage{framed}

\newcommand{\Att}{\textsc{Attack}}
\newcommand{\Hyb}{\textsc{Hybrid}}
\newcommand{\Idl}{\textsc{Ideal}}
\newcommand{\Lzy}{\textsc{Lazy}}


\newcommand{\OR}{\mathsf{OR}}
\newcommand{\AND}{\mathsf{AND}}


\newtheorem{thm}{Theorem}[section]
\newtheorem{lem}[thm]{Lemma}
\newtheorem{cor}[thm]{Corollary}
\newtheorem{propo}[thm]{Proposition}
\newtheorem{clm}[thm]{Claim}
\newtheorem{defn}[thm]{Definition}
\newtheorem{conj}[thm]{Conjecture}
\newtheorem{assm}[thm]{Assumption}
\newtheorem{rem}[thm]{Remark}
\newtheorem{obs}[thm]{Observation}
\newtheorem{egs}[thm]{Example}
\newtheorem{fct}[thm]{Fact}
\newtheorem{expr}{Experiment}
\newtheorem{cons}[thm]{Construction}
\newtheorem{nte}[thm]{Note}
\newtheorem{aim}[thm]{Aim}

% keys mathsf
%\newcommand{\sk}{\mathsf{sk}}
%\newcommand{\SK}{\mathsf{SK}}
%\newcommand{\pk}{\mathsf{pk}}
%\newcommand{\PK}{\mathsf{PK}}
%\newcommand{\dk}{\mathsf{dk}}
%\newcommand{\DK}{\mathsf{DK}}


%\newcommand{\att}{\cA}
%\newcommand{\prd}{\cP}


%%%%%%%%%%%%%%%%%%%%%%%%%%%%%%%%%%%%%%%%%%%%%%%%%%%55
\newcommand{\remove}[1]{}
\newcommand{\num}[1]{{\bf (#1)}}

\newcommand{\procfont}[1]{\mathsc{#1}}
\newcommand{\tablefont}[1]{\mathsf{#1}}

\newcommand{\mpk}{\pi}
\newcommand{\msk}{\varfont{msk}}
\newcommand{\usk}{\varfont{dk}}
\newcommand{\usknew}{\overline{\varfont{dk}}}
\newcommand{\ctxt}{\varfont{C}}
\newcommand{\ctxtnew}{\overline{\varfont{C}}}
\newcommand{\IBE}{\schemefont{IBE}}
\newcommand{\IBEnew}{\overline{\schemefont{IBE}}}
\newcommand{\mskspace}{{\cal S}}
%\newcommand{\kg}{{\cal K}}
\newcommand{\kg}{\mathsf{Kg}}
\newcommand{\kgnew}{\overline{{\cal K}}}
%\newcommand{\pg}{{\cal P}}
\newcommand{\pg}{\mathsf{Pg}}
\newcommand{\Enc}{\enc}
\newcommand{\Encnew}{\encnew}
\newcommand{\Dec}{\decc}
\newcommand{\Decnew}{\deccnew}

\def\vk{\mathit{vk}}
\def\pars{\pi}
\def\ek{\varfont{ek}}
\def\pk{\varfont{pk}}
\def\sk{\varfont{sk}}
\newcommand{\vecsk}{\mathbf{sk}}
\newcommand{\vecpk}{\mathbf{pk}}
\newcommand{\vecm}{\mathbf{m}}
\newcommand{\vecM}{\mathbf{M}}
\def\dk{\varfont{dk}}
\def\ak{\varfont{ak}}
\def\stateinfo{\varfont{st}}
\def\st{\varfont{state}}
\def\stt{\varfont{St}}
\newcommand{\RO}{H}
\newcommand{\keygen}{KG}
\newcommand{\pargen}{PG}
\newcommand{\trapdoor}{Td}
\newcommand{\test}{Test}
%\newcommand{\setup}{Setup}
\renewcommand{\AE}{\mathcal{AE}}
%\newcommand{\enc}{\mathcal{E}}
\newcommand{\enc}{\mathsf{Enc}}
%\newcommand{\decc}{\mathcal{D}}
\newcommand{\decc}{\mathsf{Dec}}
\newcommand{\eval}{\mathsf{Eval}}
\newcommand{\genc}{\algfont{\Enc}}
\newcommand{\dec}{\mathsf{Dec}}
\newcommand{\gdec}{\algfont{\Dec}}
%\newcommand{\com}{Com}
\newcommand{\ver}{Ver}
\newcommand{\open}{\ver}
\def\Hash{H}

\newcommand{\Funcs}{\mathsf{Funcs}}

\newcommand{\ol}{\overline}
\newcommand{\wt}[1]{\widetilde{#1}}
\newcommand{\wh}[1]{\widehat{#1}}
\newcommand{\nin}{\not \in}

\newcommand{\es}{\varnothing} % empty set
\newcommand{\se}{\subseteq}
\newcommand{\sm}{\setminus}
\newcommand{\nse}{\not \se}
\newcommand{\nf}[2]{\nicefrac{#1}{#2}}
\newcommand{\rv}[1]{\mathbf{#1}} % Random Variable
\newcommand{\mal}[1]{\widehat{#1}} % Malicious version of the alg

\newcommand{\schemefont}[1]{{\mathsf{#1}}}
\newcommand{\primfont}[1]{{\mathsf{#1}}}
\newcommand{\algfont}[1]{{\mathsf{#1}}}
\newcommand{\advfont}[1]{{\mathsf{#1}}}
\newcommand{\orfont}[1]{\mathsc{#1}}
\newcommand{\varfont}[1]{\mathit{#1}}
\newcommand{\randvarfont}[1]{\mathbf{#1}}
\newcommand{\constfont}[1]{\mathtt{#1}}
\newcommand{\notionfont}[1]{\mathrm{#1}}
\newcommand{\eventfont}[1]{{\mathsc{#1}}}
\newcommand{\vectorfont}[1]{\mathslbf{#1}}
\newcommand{\transformfont}[1]{\mathsf{#1}}


\newcommand{\tuple}[1]{\langle{#1}\rangle}
\def\from{\mbox{from}\ }
\def\From{\mbox{From}\ }
\def\bits{\{0,1\}}
\def\cross{\times}
\newcommand{\xor}{{\oplus}}
\newcommand{\Colon}{{:\;\;}}
\newcommand{\mystrut}{\rule{0em}{15pt}}
\newcommand{\namestrut}{\rule{0em}{20pt}}
\def\poly{\mathop{\rm poly}\nolimits}
\def\emptystring{\varepsilon}
\def\emptyid{()}
\newcommand{\Dom}{\mathsf{Dom}}
\newcommand{\DomR}{\mathsf{DomR}}
\newcommand{\Rng}{\mathsf{Rng}}
\newcommand{\RngR}{\mathsf{RngR}}
\newcommand{\Keys}{\mathsf{Keys}}
\newcommand{\calA}{{\cal A}}
\newcommand{\calB}{{\cal B}}
\newcommand{\calC}{{\cal C}}
\newcommand{\calF}{{\cal F}}
\newcommand{\calL}{{\cal L}}
\newcommand{\calM}{{\cal M}}
\newcommand{\calO}{{\cal O}}
\newcommand{\calR}{{\cal R}}
\newcommand{\calH}{{\cal H}}
\newcommand{\calG}{{\cal G}}
\newcommand{\calD}{{\cal D}}
\newcommand{\calE}{{\cal E}}
\newcommand{\calP}{{\cal P}}
\newcommand{\calS}{{\cal S}}
\newcommand{\calU}{{\cal U}}
\newcommand{\calX}{{\cal X}}
\newcommand{\calY}{{\cal Y}}
\newcommand{\calEE}{{\cal EE}}
\newcommand{\N}{{{\mathbb N}}}
\newcommand{\Z}{{{\mathbb Z}}}
\newcommand{\R}{{{\mathbb R}}}
\newcommand{\goesto}{{\rightarrow}}
\newcommand{\eqdef}{\;\stackrel{\rm def}{=}\;}
\newcommand{\then}{{\;;\;\;}}   % for [ ; ; ; : ] notation
\newcommand{\andthen}{{\;:\;\;}}
\def\union{\cup}
\def\bigunion{\bigcup}
\def\intersection{\cap}
\def\bigintersection{\bigcap}
\def\suchthatt{\: :\:}
\newcommand{\suchthat}{{\mbox{s.t.\ }}}
\def\next{\:;\:}
\def\nextt{\:;\:}
\newcommand{\sett}[1]{\{#1\}}
\newcommand{\set}[2]{\{\:#1 \suchthatt #2\:\}}
\newcommand{\setsize}[1]{\left|{#1}\right|}
\def\leqq{\;\leq\;}
\def\eqq{\;=\;}
\def\geqq{\;\geq\;}
\def\equivv{\;\equiv\;}
\def\prn#1{\left(#1\right)}
\newcommand{\getsr}{{\:{\leftarrow{\hspace*{-3pt}\raisebox{.75pt}{$\scriptscriptstyle\$$}}}\:}}
% \def\getsr{\stackrel{{\scriptscriptstyle\$}}{\leftarrow}}
% \renewcommand{\choose}[2]{{{#1}\atopwithdelims(){#2}}}

\newcommand{\Var}{{\mbox{\bf Var}}}
\newcommand{\E}{\mathbf{E}}
\newcommand{\EE}[1]{{\E\left[{#1}\right]}}
\newcommand{\EEE}[2]{{\E_{#1}\left[{#2}\right]}}
\newcommand{\Prob}[1]{{\Pr\left[\,{#1}\,\right]}}
\newcommand{\probb}[2]{{\Pr}_{#1}\left[\,{#2}\,\right]}
\newcommand{\probbs}[3]{{\Pr}^{#1}_{#2}\left[\,{#3}\,\right]}
\newcommand{\probbp}[2]{{\Pr}_{#1}'\left[\,{#2}\,\right]}
\newcommand{\Probb}[2]{{{\Pr}_{#1}\left[\,{#2}\,\right]}}
\newcommand{\Probbp}[2]{{{\Pr}_{#1}'\left[\,{#2}\,\right]}}
\newcommand{\condProb}[2]{{\Pr}\left[\,#1\,|\,#2\,\right]}
\newcommand{\CondProb}[2]{{\Pr}\left[\: #1\:\left|\right.\:#2\:\right]}
\newcommand{\CondProbb}[3]{{\Pr}_{#1}\left[\: #2\:\left|\right.\:#3\:\right]}
\newcommand{\CondProbbp}[3]{%
{\Pr}_{#1}^{'}\left[\: #2\:\left|\right.\:#3\:\right]}


% ===================================================================
\newcommand{\kwfont}[1]{{\ensuremath{\mathrm{#1}}}}
\newcommand{\kwfunction}{{\kwfont{function}\ }}
\newcommand{\kwlabel}{{\kwfont{label}\ }}
\newcommand{\kwfor}{{\kwfont{for}\ }}
\newcommand{\kwand}{{\kwfont{and}\ }}
\newcommand{\kwor}{{\kwfont{or}\ }}
\newcommand{\kwnot}{{\kwfont{not}}}
\newcommand{\kwdo}{{\kwfont{do}\ }}
\newcommand{\kwreturn}{{\kwfont{return}\ }}
\newcommand{\kwReturn}{{\kwfont{Return}\ }}
\newcommand{\kwalgorithm}{{\ensuremath{\mathbf{Algorithm}\ }}}
\newcommand{\kwprotocol}{{\kwfont{Protocol}\ }}
\newcommand{\kwexperiment}{{\kwfont{Experiment}\ }}
\newcommand{\kwadversary}{{\kwfont{Adversary}\ }}
\newcommand{\kworacle}{{\kwfont{Oracle}\ }}
\newcommand{\kwuntil}{{\kwfont{until}\ }}
\newcommand{\kwrepeat}{{\kwfont{repeat}\ }}
\newcommand{\kwif}{{\kwfont{If}\ }}
\newcommand{\kwthen}{{\kwfont{Then}\ }}
\newcommand{\kwelse}{{\kwfont{Else}\ }}
\newcommand{\kwabort}{{\kwfont{abort}\ }}
\newcommand{\kwgoto}{{\kwfont{goto}\ }}
\newcommand{\kwwhile}{{\kwfont{while}\ }}
\newcommand{\kwparse}{{\kwfont{parse}\ }}
\newcommand{\kwas}{{\kwfont{as}\ }}
\newcommand{\kwstatic}{{\kwfont{static}\ }}
\newcommand{\kwrun}{{\kwfont{run}\ }}
\newcommand{\kwbegin}{{\kwfont{begin}\ }}
\newcommand{\kwend}{{\kwfont{end}\ }}
\newcommand{\kwstart}{{\kwfont{start}\ }}
\newcommand{\kwcontinue}{{\kwfont{continue}\ }}
\newcommand{\kwdefine}{{\kwfont{define}\ }}
\newcommand{\kwflip}{{\kwfont{flip}\ }}
\newcommand{\kwlet}{{\kwfont{let}\ }}
\newcommand{\kwof}{{\kwfont{of}\ }}
\newcommand{\kwcase}{{\kwfont{case}\ }}
\newcommand{\kwswitch}{{\kwfont{switch}\ }}
\newcommand{\kwpick}{{\kwfont{pick}\ }}
\newcommand{\kwset}{{\kwfont{set}\ }}
\newcommand{\kwcompute}{{\kwfont{compute}\ }}
\newcommand{\comment}[1]{\hspace{15pt}{\small /$\!\!$/\ #1}}
\newcommand{\Comment}[1]{\hspace{5pt}{/$\!\!$/\ #1}}
\newcommand{\sComment}[1]{\hspace{2pt}{/$\!\!$/#1}}

\newcommand{\Coins}{\mathsf{Coins}}

\newcommand{\ProbExp}[2]{{\Pr}\left[\: #1\:\suchthatt\:#2\:\right]}

\newcommand{\PRG}{\schemefont{PRG}}
\newcommand{\PRGpg}{\schalg{PRG}{Ev}}
\newcommand{\PRGkg}{\schalg{PRG}{Kg}}

\newcommand{\FSE}{\schemefont{FSE}}

\newcommand{\PE}{\schemefont{PE}}
\newcommand{\FE}{\schemefont{FE}}
\newcommand{\setup}{\mathsf{Setup}}
\newcommand{\keyder}{\mathsf{KeyDer}}

\newcommand{\OT}{\schemefont{OT}}
\newcommand{\Send}{\procfont{Sender}}
\newcommand{\Rec}{\procfont{Receiver}}
\newcommand{\PKE}{\schemefont{PKE}}
\newcommand{\FPKE}{\schemefont{FPKE}}
\newcommand{\DPKE}{\schemefont{DPKE}}
\newcommand{\DE}{\schemefont{DE}}
\newcommand{\PKEpg}{\schalg{PKE}{Pg}}
\newcommand{\oPKEpg}{\overline{{\cal P}}}
\newcommand{\PKEEnc}{\schalg{PKE}{Enc}}
\newcommand{\oPKEEnc}{\schalg{\overline{PKE}}{Enc}}
\newcommand{\PKEDec}{\schalg{PKE}{Dec}}
\newcommand{\oPKEDec}{\overline{{\cal D}}}
\newcommand{\PKEkg}{\schalg{PKE}{Kg}}
\newcommand{\oPKEkg}{\overline{{\cal K}}}
\newcommand{\PKEpk}{\mathit{ek}}
\newcommand{\PKEPK}{\schalg{PKE}{PK}}
\newcommand{\PKESK}{\schalg{PKE}{SK}}
\newcommand{\oPKEpk}{\overline{\mathit{ek}}}
\newcommand{\PKEsk}{\mathit{dk}}
\newcommand{\oPKEsk}{\overline{\mathit{dk}}}
\newcommand{\PKEpars}{\pi}
\newcommand{\oPKEpars}{\overline{\pi}}

\newcommand{\SE}{\schemefont{SE}}
\newcommand{\SEpg}{{\cal P}}
\newcommand{\SEEnc}{{\cal E}}
\newcommand{\SEDec}{{\cal D}}
\newcommand{\SEkg}{{\cal K}}
\newcommand{\SEkey}{K}
\newcommand{\SEpars}{\pi}

%%%  English %%%%%%%%%%%%%%%%%%%%%%%%%%%%%%%%%%%%%%%%%%%%%%%%%%%%%%

\newcommand{\etal}{et~al.\ }
\newcommand{\aka}{also known as,\ }
\newcommand{\resp}{resp.,\ }
\newcommand{\ie}{i.e.,\ }
\newcommand{\wolog}{w.l.o.g.\ }
\newcommand{\Wolog}{W.l.o.g.\ }
\newcommand{\eg}{e.g.,\ }
\newcommand{\Eg}{E.g.,\ }
\newcommand{\wrt} {with respect to\ }
\newcommand{\cf}{{cf.,\ }}

%%%  math %%%%%%%%%%%%%%%%%%%%%%%%%%%%%%%%%%%%%%%%%%%%%%%%%%%%%%

\newcommand{\round}[1]{\lfloor #1 \rceil}
\newcommand{\ceil}[1]{\lceil #1 \rceil}
\newcommand{\floor}[1]{\lfloor #1 \rfloor}
\newcommand{\angles}[1]{\langle #1 \rangle}
\newcommand{\parens}[1]{( #1 )}
\newcommand{\bracks}[1]{[ #1 ]}
\newcommand{\bra}[1]{\langle#1\rvert}
\newcommand{\ket}[1]{\lvert#1\rangle}


\newcommand{\adjRound}[1]{\left\lfloor #1 \right\rceil} % Adjusted Round
\newcommand{\adjCeil}[1]{\left\lceil #1 \right\rceil}
\newcommand{\adjFloor}[1]{\left\lfloor #1 \right\rfloor}
\newcommand{\adjAngles}[1]{\left\langle #1 \right\rangle}
\newcommand{\adjParens}[1]{\left( #1 \right)}
\newcommand{\adjBracks}[1]{\left[ #1 \right]}
\newcommand{\adjBra}[1]{\left\langle#1\right\rvert}
\newcommand{\adjKet}[1]{\left\lvert#1\right\rangle}
\newcommand{\adjSet}[1]{\left\{ #1 \right\}}
\newcommand{\half}{\tfrac{1}{2}}
\newcommand{\third}{\tfrac{1}{3}}
\newcommand{\quarter}{\tfrac{1}{4}}
%\newcommand{\eqdef}{:=}
%\newcommand{\zo}{\{0,1\}}




\newcommand{\cA}{{\mathcal A}}
\newcommand{\cB}{{\mathcal B}}
\newcommand{\cC}{{\mathcal C}}
\newcommand{\cD}{{\mathcal D}}
\newcommand{\cE}{{\mathcal E}}
\newcommand{\cF}{{\mathcal F}}
\newcommand{\cG}{{\mathcal G}}
\newcommand{\cH}{{\mathcal H}}
\newcommand{\cI}{{\mathcal I}}
\newcommand{\cJ}{{\mathcal J}}
\newcommand{\cK}{{\mathcal K}}
\newcommand{\cL}{{\mathcal L}}
\newcommand{\cM}{{\mathcal M}}
\newcommand{\cN}{{\mathcal N}}
\newcommand{\cO}{{\mathcal O}}
\newcommand{\cP}{{\mathcal P}}
\newcommand{\cQ}{{\mathcal Q}}
\newcommand{\cR}{{\mathcal R}}
\newcommand{\cS}{{\mathcal S}}
\newcommand{\cT}{{\mathcal T}}
\newcommand{\cU}{{\mathcal U}}
\newcommand{\cV}{{\mathcal V}}
\newcommand{\cW}{{\mathcal W}}
\newcommand{\cX}{{\mathcal X}}
\newcommand{\cY}{{\mathcal Y}}
\newcommand{\cZ}{{\mathcal Z}}

\newcommand{\bfA}{\mathbf{A}}
\newcommand{\bfB}{\mathbf{B}}
\newcommand{\bfC}{\mathbf{C}}
\newcommand{\bfD}{\mathbf{D}}
\newcommand{\bfE}{\mathbf{E}}
\newcommand{\bfF}{\mathbf{F}}
\newcommand{\bfG}{\mathbf{G}}
\newcommand{\bfH}{\mathbf{H}}
\newcommand{\bfI}{\mathbf{I}}
\newcommand{\bfJ}{\mathbf{J}}
\newcommand{\bfK}{\mathbf{K}}
\newcommand{\bfL}{\mathbf{L}}
\newcommand{\bfM}{\mathbf{M}}
\newcommand{\bfN}{\mathbf{N}}
\newcommand{\bfO}{\mathbf{O}}
\newcommand{\bfP}{\mathbf{P}}
\newcommand{\bfQ}{\mathbf{Q}}
\newcommand{\bfR}{\mathbf{R}}
\newcommand{\bfS}{\mathbf{S}}
\newcommand{\bfT}{\mathbf{T}}
\newcommand{\bfU}{\mathbf{U}}
\newcommand{\bfV}{\mathbf{V}}
\newcommand{\bfW}{\mathbf{W}}
\newcommand{\bfX}{\mathbf{X}}
\newcommand{\bfY}{\mathbf{Y}}
\newcommand{\bfZ}{\mathbf{Z}}


\newcommand{\bfa}{\mathbf{a}}
\newcommand{\bfb}{\mathbf{b}}
\newcommand{\bfc}{\mathbf{c}}
\newcommand{\bfd}{\mathbf{d}}
\newcommand{\bfe}{\mathbf{e}}
\newcommand{\bff}{\mathbf{f}}
\newcommand{\bfg}{\mathbf{g}}
\newcommand{\bfh}{\mathbf{h}}
\newcommand{\bfi}{\mathbf{i}}
\newcommand{\bfj}{\mathbf{j}}
\newcommand{\bfk}{\mathbf{k}}
\newcommand{\bfl}{\mathbf{l}}
\newcommand{\bfm}{\mathbf{m}}
\newcommand{\bfn}{\mathbf{n}}
\newcommand{\bfo}{\mathbf{o}}
\newcommand{\bfp}{\mathbf{p}}
\newcommand{\bfq}{\mathbf{q}}
\newcommand{\bfr}{\mathbf{r}}
\newcommand{\bfs}{\mathbf{s}}
\newcommand{\bft}{\mathbf{t}}
\newcommand{\bfu}{\mathbf{u}}
\newcommand{\bfv}{\mathbf{v}}
\newcommand{\bfw}{\mathbf{w}}
\newcommand{\bfx}{\mathbf{x}}
\newcommand{\bfy}{\mathbf{y}}
\newcommand{\bfz}{\mathbf{z}}



\newcommand{\sfA}{\mathsf{A}}
\newcommand{\sfB}{\mathsf{B}}
\newcommand{\sfC}{\mathsf{C}}
\newcommand{\sfD}{\mathsf{D}}
\newcommand{\sfE}{\mathsf{E}}
\newcommand{\sfF}{\mathsf{F}}
\newcommand{\sfG}{\mathsf{G}}
\newcommand{\sfH}{\mathsf{H}}
\newcommand{\sfI}{\mathsf{I}}
\newcommand{\sfJ}{\mathsf{J}}
\newcommand{\sfK}{\mathsf{K}}
\newcommand{\sfL}{\mathsf{L}}
\newcommand{\sfM}{\mathsf{M}}
\newcommand{\sfN}{\mathsf{N}}
\newcommand{\sfO}{\mathsf{O}}
\newcommand{\sfP}{\mathsf{P}}
\newcommand{\sfQ}{\mathsf{Q}}
\newcommand{\sfR}{\mathsf{R}}
\newcommand{\sfS}{\mathsf{S}}
\newcommand{\sfT}{\mathsf{T}}
\newcommand{\sfU}{\mathsf{U}}
\newcommand{\sfV}{\mathsf{V}}
\newcommand{\sfW}{\mathsf{W}}
\newcommand{\sfX}{\mathsf{X}}
\newcommand{\sfY}{\mathsf{Y}}
\newcommand{\sfZ}{\mathsf{Z}}

\newcommand{\sfa}{\mathsf{a}}
\newcommand{\sfb}{\mathsf{b}}
\newcommand{\sfc}{\mathsf{c}}
\newcommand{\sfd}{\mathsf{d}}
\newcommand{\sfe}{\mathsf{e}}
\newcommand{\sff}{\mathsf{f}}
\newcommand{\sfg}{\mathsf{g}}
\newcommand{\sfh}{\mathsf{h}}
\newcommand{\sfi}{\mathsf{i}}
\newcommand{\sfj}{\mathsf{j}}
\newcommand{\sfk}{\mathsf{k}}
\newcommand{\sfl}{\mathsf{l}}
\newcommand{\sfm}{\mathsf{m}}
\newcommand{\sfn}{\mathsf{n}}
\newcommand{\sfo}{\mathsf{o}}
\newcommand{\sfp}{\mathsf{p}}
\newcommand{\sfq}{\mathsf{q}}
\newcommand{\sfr}{\mathsf{r}}
\newcommand{\sfs}{\mathsf{s}}
\newcommand{\sft}{\mathsf{t}}
\newcommand{\sfu}{\mathsf{u}}
\newcommand{\sfv}{\mathsf{v}}
\newcommand{\sfw}{\mathsf{w}}
\newcommand{\sfx}{\mathsf{x}}
\newcommand{\sfy}{\mathsf{y}}
\newcommand{\sfz}{\mathsf{z}}

\newcommand{\eps}{\epsilon}
%\newcommand{\e}{\epsilon}
\newcommand{\veps}{\varepsilon}
%\newcommand{\vare}{\varepsilon}
\newcommand{\vphi}{\varphi}
\newcommand{\vsigma}{\varsigma}
\newcommand{\vrho}{\varrho}
\newcommand{\vpi}{\varpi}
\newcommand{\tO}{\widetilde{O}}
\newcommand{\tOmega}{\widetilde{\Omega}}
\newcommand{\tTheta}{\widetilde{\Theta}}
\newcommand{\field}{\ensuremath \Bbbk}
\newcommand{\reals}{\ensuremath \mathbb{R}}

\newcommand{\client}{\mathsl{client}}
\newcommand{\server}{\mathsl{server}}
\newcommand{\TTP}{\procfont{TTP}}
\newcommand{\rec}{\mathsf{rec}}
\newcommand{\doc}{\mathsf{doc}}
\newcommand{\pred}{\mathsf{pred}}
\newcommand{\scrub}{\mathsf{scrub}}

\newcommand{\ODS}{\schemefont{ODS}}
\newcommand{\commit}{\mathsf{Commit}}
\newcommand{\query}{\mathsf{Query}}
\newcommand{\DB}{\mathsf{DB}}


\newcommand{\sanjam}[1]{\textcolor{red}{Sanjam: #1}}
\newcommand{\moh}[1]{}%{\textcolor{red}{Moh: #1}}
\newcommand{\adam}[1]{\textcolor{red}{Adam: #1}}
\newcommand{\ben}[1]{\textcolor{red}{Ben: #1}}

%---
\newtheorem{definition}{Definition}[section]
\newtheorem{question}{Question}[section]
\newtheorem{fact}{Fact}[section]
\newtheorem{lemma}{Lemma}[section]
\newtheorem{corollary}{Corollary}[section]
\newtheorem{theorem}{Theorem}[section]
\newtheorem{claim}{Claim}[section]
\newtheorem{assumption}{Assumption}[section]
\newtheorem{proposition}{Proposition}[section]
\newtheorem{hypothesis}{Hypothesis}[section]
\newtheorem{observation}{Observation}[section]
\theoremstyle{remark}
\newtheorem{construction}{Construction}[section]
\newtheorem{remark}{Remark}[section]
\newtheorem{example}{Example}[section]

\newcommand{\pone}{\mbox{$P_1$}}
\newcommand{\ptwo}{\mbox{$P_2$}}

\renewcommand{\S}{{\cal S}}
\newcommand{\F}{{\cal F}}
\newcommand{\G}{{\mathbb{G}}}

\newcommand{\vecx}{{\mathbf{x}}}
\newcommand{\vecy}{{\mathbf{y}}}

\newcommand{\numcorr}{{\mathsf{ncorr}}}
\newcommand{\numencq}{{\mathsf{nencq}}}
\newcommand{\numkeyq}{{\mathsf{nkeyq}}}


\renewenvironment{proof}{\noindent{\bf Proof:~~}}{\qed}
\newcommand{\BPF}{\begin{proof}} \newcommand {\EPF}{\end{proof}}
\newenvironment{proofsketch}{\noindent{\bf Proof Sketch:~~}}{\qed}
\newcommand{\BPFS}{\begin{proofsketch}} \newcommand {\EPFS}{\end{proofsketch}}

%\def\qed{\quad\blackslug\lower 8.5pt\null\par}


\newcommand{\BPR}{\begin{myprotocol}}   \newcommand{\EPR}{\end{myprotocol}}
\newcommand{\ourfigg}[5]{
{\begin{figure}[#4]
\begin{center}
\framebox[\width][c]{
    \small
    \hbox{\quad
    \begin{varwidth}[c]{0.9\textwidth}
    %\begin{center}
    \begin{myfigure}
    [#1]
    \label{#2}
    \end{myfigure}
    %\end{center}
    \vspace{-3ex}
    #5
    \end{varwidth}
    \quad}
    }
    \begin{center} #3 \end{center}
    \vspace{-6ex}
\end{center}
\end{figure}
} }


% USAGE: \ourfig{TITLE}{LABEL}{CAPTION}{BODY}
\newcommand{\ourfig}[4]{\ourfigg{#1}{#2}{#3}{htb}{#4}}

\newcommand{\prott}[5]{
{\begin{figure}[#4]
\begin{center}
\framebox[\width][c]{
    \small
    \hbox{\quad
    \begin{varwidth}[c]{0.9\textwidth}
    %\begin{center}
    \begin{myprotocol}
    [#1]
    \label{#2}
    \end{myprotocol}
    %\end{center}
    \vspace{-3ex}
    #5
    \end{varwidth}
    \quad}
    }
    \begin{center} #3 \end{center}
    \vspace{-6ex}
\end{center}
\end{figure}
} }

% USAGE: \prot{TITLE}{LABEL}{CAPTION}{BODY}
\newcommand{\prot}[4]{\prott{#1}{#2}{#3}{htb}{#4}}

\newcommand{\view}{{\sf view}}
\newcommand{\trans}{{\sf trans}}
\newcommand{\com}{Z}
\newcommand{\cecom}{{\sf CECom}}
\newcommand{\mxcom}{{\sf MXCom}}
\newcommand{\mxzk}{{\sf MXZK}}
\newcommand{\nmmxcom}{{\sf NMMXCom}}


\newcommand{\crsgen}{{\sf CRSGen}}
\newcommand{\crs}{{\sf CRS}}


\newcommand{\Commit}{{\sf Com}}


\newcommand{\scheme} {{\mathcal{S}}}

\newcommand{\siggen} {{\sf SigKeyGen}}
\newcommand{\sig} {{\sf Sig}}

%\newcommand{\size} {{\sf SIZE}}
\newcommand{\state} {\mathrm{state}}

\newcommand{\idx} {\mathrm{index}}

\newcommand{\kdm}{{\scriptscriptstyle\mathrm{KDM}}}
\newcommand{\skdm}{{\scriptscriptstyle\mathrm{SKDM}}}
\newcommand{\cpa}{{\scriptscriptstyle\mathrm{CPA}}}
\newcommand{\new}{{\scriptscriptstyle\mathrm{NEW}}}


\newcommand{\rsetup}[1]{\R({\sf setup}, #1 ) }
\newcommand{\rquery}[1]{\R({\sf query}, #1 ) }
\newcommand{\rchall}[1]{\R({\sf challenge}, #1 ) }
\newcommand{\rfinal}[1]{\R({\sf final}, #1 ) }
\newcommand{\iO}{i\mathcal{O}}

\newcommand{\PRF}{{\mathsf{F}}}
\newcommand{\PRFGen}{\mathsf{PRF{.}Gen}}
\newcommand{\PRFPunc}{\mathsf{PRF{.}Punc}}
\newcommand{\seed}{\ensuremath{{K}}}
\newcommand{\secpar}{\secparam}
\newcommand{\Adv}{\mathcal{A}}
\newcommand{\rsample}{\gets}

\newcommand{\tf}{\mathrm{tf}}
\newcommand{\idf}{\mathrm{idf}}
\newcommand{\df}{\mathrm{df}}
\newcommand{\p}{\mathrm{P}}



\date{}
\begin{document}

\maketitle
%\vspace{-20mm}
%\begin{abstract}
%ent results have resolved functional encryption for all circuits. Does this solve most problems for functional encryption. We think not. In this paper we present the future vision for functional encryption. Our vision for functional encryption is to realize encryption systems which (at least asymptotically) approach the efficiency levels approached in insecure solutions.
%\end{abstract}
\vspace{-22mm}

%!TEX root = dbproposal.tex

\section{Introduction}

The importance of collecting and storing data is universal, with use
cases in governmental~\cite{Powers2014},
commercial~\cite{Linoff:2002:MWT:560274,insightdata}, and personal
sectors~\cite{Mons2011}.
Storing and querying large datasets has tremendous value in improving
decision making, but this growth in size and complexity is increasingly
resulting in organizations relying on external cloud providers for their
data needs.

This outsourced storage represents a natural attack target.  Attacks occur against both government \cite{CyberAttacksOPM} and commercial~\cite{CyberAttacks,gressin2017equifax} datasets.
One natural response to this risk is to encrypt data before outsourcing. 
%
However, employing encryption comes at the cost of disabling the cloud server
from {\em quickly processing the data and answering
complex queries from the client}. 
%
Ideally, we could use more sophisticated cryptographic techniques to {\em create databases
capable of efficiently answering a client's queries without revealing
information to the cloud server}.  

%Secure outsourced databases use advanced cryptography to achieve this goal.  This field encompasses a variety of cryptographic techniques, including property-preserving encryption or PPE~\cite{EC:PanRou12}, searchable symmetric encryption or SSE~\cite{CCS:CGKO06}, private information retrieval by keyword~\cite{EPRINT:ChoGilNao98}, and public-key encryption with keyword search~\cite{EC:BDOP04}.  

Numerous advanced cryptographic techniques have been proposed to achieve
this goal \cite{EC:PanRou12,CCS:CGKO06,EPRINT:ChoGilNao98,EC:BDOP04}.
This research has largely split into two research threads: 
property-preserving encryption (PPE) which
emphasizes background compatibility and use of legacy database
management systems, and searchable secure encryption (SSE) which emphases security.

PPE creates
symmetric encryption techniques
compatible with an unprotected database. Examples include deterministic
encryption~\cite{C:BelBolONe07}, which can answer equality queries, and
order-preserving encryption~\cite{C:BolCheONe11,EC:BCLO09},
which can answer range queries. Academic teams, start-up companies (including Bitglass, Ciphercloud, Crypteron, PreVeil, Skyhigh, ZeroDB) and Fortune
500 companies (including Microsoft's SQL Server 2016 and Azure and Google's Encrypted BigQuery)  offer variants of property-preserving encryption.
%Recent attacks against these schemes indicate even ideal property-preserving
%encryption can't be secure for many applications.

By contrast, SSE schemes can be viewed as starting from secure multi-party
computation and
optimizing solutions for common database tasks (both \cite{SP:FVKKKM15} and \cite{RSA:IKLO16} explicitly use multi-party computation for sensitive subcomputations). This approach requires redesign
of database indexing mechanisms and achieves better security at the cost of decreased efficiency and backward compatibility. 

These systems have been implemented at moderate scale.  Both property-preserving solutions~\cite{CACM:PRZB12,EPRINT:PodBoePop16} and searchable encryption~\cite{SP:PKVKMC14,SP:FVKKKM15,C:CJJKRS13,CCS:JJKRS13,NDSS:CJJJKR14,ESORICS:FJKNRS15,RSA:IKLO16} solutions have been tested on datasets with billions of records.

%
%  Folks--we need to be careful here.  DBMS functions are much broader than search; there's transformation and presentation, access control, backup, recovery management, etc., and we're talking about none of that here.


\subsection{Use Cases}
Emerging databases include large scale graph databases, analytic databases, and
biometric databases. 
\begin{enumerate}
\item Many data sets are naturally interpreted as large sparse graphs.  Examples include social networks such as Facebook and Twitter and communities such as organizational communication, academic co-authorship, and the co-stardom network.  This type of network is also used to perform Internet scale analysis.  Common graph algorithms include computing triangles (sets of nodes $\{a,b,c\}$ where all pairs are close according to a metric), shortest path algorithms, network diameter, and degree distribution.
\item  Increasingly, databases are not asked to return subsets of data
but rather derive statistics and analytics about the stored data.
Machine-learning-as-a-service has emerged as a business model for large
and small companies~\cite{mlservice}.  Many machine learning algorithms
depend on computing distance between points. For example, linear regression finds the
line that minimizes the sum of distances between the line and data
points.  Similarly, the first principal component minimizes the sum of
distances between the selected line and the
dataset~\cite{wold1987principal}.
% The remaining principal components
% are defined similarly (they must also be orthogonal to previously
% defined components).
\item The FBI has long held a fingerprint database with hundreds of millions of records~\cite{brislawn1996fbi} for identifying criminals.  Increasingly, countries are using biometrics as an identifier for citizens, linking biometrics with unique identifiers.  The Aadhaar system in India links biometrics with a unique 12 digit number with over 1 billion numbers issued~\cite{daugman2014600}.  Increasingly, passports are biometric enabled~\cite{stanton2008icao}.  As these databases move to indexing all citizens, privacy concerns abound.
These databases compare the distance between a target point $a$ and the
set of stored points $b_i$, returning all points within some defined
threshold distance from the target.
\end{enumerate}

\noindent
These databases share a fundamental operation: {\em
computing distance/proximity of tuples of points}. 

\subsection{Inadequacy of Prior Work}

In 2000, Song, Wagner, and Perrig provided the first scheme with
communication proportional to the description of the query and the
server performing (roughly) a linear scan of the encrypted
database~\cite{SP:SonWagPer00}.  There has been tremendous work since
2000: both PPE and SSE approaches handle much of SQL and NoSQL queries
and scale to datasets of billions of records.
Though there has been some work on computing shortest paths
\cite{CCS:MKNK15}, in general neither PPE nor SSE
is capable of handling geometric data that is prominent in the
applications discussed above. 
Furthermore, each approach also
has a second weakness:
\begin{enumerate}
\item PPE has been subject to a number of leakage-abuse attacks that show that order-preserving encryption (and sometimes deterministic encryption) is not safe in most cases~\cite{CCS:NavKamWri15,CCS:CGPR15,CCS:KKNO16,CCS:PouWri16,CCS:GMNRS16,EPRINT:GSBNR16,EPRINT:ZhaKatPap16}.  
Nonetheless, industry has primarily adopted this approach.
\item SSE solutions are more limited in functionality and efficiency than PPE based solutions.  The fastest SSE based solutions report overhead of 300\%~\cite{C:CJJKRS13,CCS:JJKRS13,NDSS:CJJJKR14,ESORICS:FJKNRS15} while PPE based solutions report overhead of 30\%~\cite{CACM:PRZB12}.  We are only aware of a single SSE solution that can handle JOIN statements~\cite{EPRINT:KamMoa16}.  As of this writing, the information  leaked by this scheme is not clear.  These solutions replace the entire database software stack with a custom ``cryptographic'' database.  We posit that the lack of backward compatibility and administrative drawbacks hurt industry adoption.  
\end{enumerate}

The emergence of PPE systems indicates the approach has benefits beyond efficiency.  These benefits include backward compatibility, use of legacy software, and improved transparency.
The proposed research will
consider proximity queries as a case study to understand the following
question.  \begin{quote}\emph{Is it possible to securely outsource
modern databases while using a traditional database software stack?
What is the minimum amount that an unprotected database must be modified
to achieve security?}\end{quote}


\paragraph{Out of scope.} We do not address approaches based on
fully-homomorphic encryption~\cite{STOC:Gentry09} or functional
encryption~\cite{FOCS:GGHRSW13}, which are still too slow to be
practical for the data sets we are considering.
In addition, we do not
consider improvements to private information
retrieval~\cite{FOCS:CGKS95} or oblivious
RAM~\cite{STOC:Goldreich87,goldreich1996software} to be in scope for
this proposal, though we may make use of these constructions within our
schemes.
We will also not consider developing new leakage inference attacks.
However, the PIs are well aware of (and have contributed to) to these attacks on
secure databases, and they motivate the solutions we are
proposing.

Lastly, secure enclaves such as Intel SGX offer a promising hardware approach that is being used to isolate programs and provide security~\cite{EPRINT:CosDev16}.  SGX can be used to simulate cryptographic primitives~\cite{EPRINT:SasGorFle17,EPRINT:FVBG16}.  We believe that SGX can be used to upgrade the security of secure databases from honest-but-curious to malicious.  However, we leave combining our techniques with secure hardware to future work.


\subsection{Proposed Work}
The goal of this proposal is to create secure outsourced databases that
will be adopted and used by industry.  Towards achieving this goal we
recognize two lessons from the past: (1) PPE must be carefully analyzed
to understanding security and (2) there are tremendous benefits of
maintaining backwards compatibility with existing database systems.
Together, these lessons tell us that some modification of database
systems may be necessary but these modifications should be judicious.
This will be our guiding principle thorough this project.  Our goal is to create a third approach to secure outsourced databases: 
\begin{quote}\emph{Cryptographic operations are restricted to only
database operators (used to create index structures).  These
cryptographic operators can call interactive protocols but do not
require replacing the unprotected database. }
\end{quote}

\noindent
The proposed research will
consider proximity queries as a case study and give constructions supporting
these queries.  The research will be split into three components:
\begin{description}
\item[\Cref{sec:ppe}.] To show the promise of this approach, we
will first consider whether property-preserving encryption can be used
to answer proximity queries.  We expect that the strength and utility of
\emph{distance-preserving encryption} will depend on the definition of
security, in particular questions such as: (1) should distance be precisely preserved? 
(2) is distance-revealing encryption sufficient? and (3) what is the distance
metric? Building on the research of the PI on order-preserving
encryption~\cite{EC:BCLO09,C:BolCheONe11}, this component will consist
of the following tasks:
\begin{enumerate}
\setlength\itemsep{0em}
\item Analysis of security provided by distance-preserving and distance-comparison preserving encryption for common metrics
\item Design of distance-revealing encryption
\item Design of approximate distance-revealing encryption
\end{enumerate}

\item[\Cref{sec:interactive}.] When the security provided by
distance-revealing encryption appears inadequate for an application, we
will change the model to allow operators to call interactive protocols.
We will show what functionality can be built using \emph{interactive
operators}, drawing on recent work of the
co-PIs~\cite{SP:PKVKMC14,CCS:RACY16}.  This component will consist of
the following tasks:
\begin{enumerate}
\setlength\itemsep{0em}
\item Developing Partial \emph{Distance}-Preserving Encryption and
  applying it to Euclidean proximity and edit distance queries
\item Improving the security of POPE with forward security via
  re-insertions and random shuffling
\item Enabling multidimensional queries in BlindSeer and removing the need for
  additive homomorphic encryption and secure two-party computation in the cloud
  setting 
\item Combining ORAM and interactive approaches to reduce leakage
\item Developing a new hybrid approach with approximate
  distance-revealing encryption and partial interaction, to obtain
  better performance and backwards compatibility
\end{enumerate}

\item[\Cref{sec:analysis}.] Throughout the project, we will implement
and analyze our most promising schemes and place them in a fair context
with prior and ongoing work by others. The goal will be to give
practitioners as well as researchers a clear and \emph{consistent}
picture of the tradeoffs between security, performance, and backwards
compatibility.
We will use the UConn HPC Cluster to evaluate our
implementations on realistic-size databases (\url{https://hpc.uconn.edu}).
We expect this evaluation
to use and extend the database and query generator created as part of
the IARPA SPAR project~\cite{varia2015automated}. Specifically, our
tasks are:
\begin{enumerate}
\setlength\itemsep{0em}
\item Analyzing and developing metrics for the leakage of
information or access patterns inherent in our new PPE and interactive
schemes
\item Developing and providing open-source access to the new tools that
we develop
\item Examining the performance of our approaches on realistic and
repeatable experiments, using industrially-relevant datasets and query
benchmarks
\end{enumerate}
\end{description}

\subsection{Intellectual Merits and Broader Impacts}
\paragraph{Intellectual merits.}  
Encrypted databases and searchable encryption have rich histories rooted in the
design of oblivious random access machines.  The field has been the focus of multiple large scale projects including IARPA's APP and SPAR~\cite{spar_baa}.  The field is quite diverse bringing together cryptographers, system researchers, and database experts.  Furthermore, there is clear demand in industry for solutions.  Data breaches are becoming nearly daily events.  However, recent
leakage inference attacks have taught us that just because something is called encrypted does not make it secure.  This problem requires careful design that balances functionality, efficiency, and security needs.  The rapid deployment of property-preserving
techniques has reinforced the importance of simplicity, efficiency and backward
compatibility.  These two developments inform the core of our approach: using
interactivity to strengthen property-preserving encryption.

\paragraph{Broader impacts.}
The confidentiality of data is a core societal tenant.  Deployed encrypted
databases provide little security and may even hurt by providing a false sense
of security.  There is a tremendous need for research in this area to
understand the tradeoffs between security, functionality, and efficiency.

The PIs are committed to broad dissemination of research material.  The PIs
have participated in large scale evaluation of searchable encryption, working
on both constructions and attacks. The PIs have a track record of
releasing the source code implementations of their work (including
previous work on searchable encryption) and contributing
to open-source projects.
The protocols and other products of this proposal will be released
publicly as open-source implementations.

Lastly, all PIs are
dedicated to engaging with undergraduates. Two co-PIs being at an undergraduate
only institution, the US Naval Academy, where co-PI Roche was recently
recognized with the institution-wide Apgar Award for Teaching. PI O'Neill has
previously been awarded an REU and used it to  work with a female undergraduate
student. Co-PI Fuller supervises the cyber-security club at the University of
Connecticut and supports their efforts to understand and research computer
security.  This club allows students with varying educational preparation to
engage outside of the classroom and learn the impact of computer science and
security. 





%!TEX root = dbproposal.tex

\section{Property-Preserving Encryption for Proximity}
\label{sec:ppe}

In spatial databases, nearest neighbor queries (\emph{e.g.}, finding the closest soldier in the field) and clustering queries are pervasive.  %Algorithms for executing these query types need to understand distance relationships between tuples o perform the fundamental operation of \emph{distance comparison} between points in the database, \emph{i.e.}, determining which of two candidates points are closer to a target point.  
In large scale biometric databases, the fundamental operation is
comparison of the distance between a target point and all stored points.
Often, biometric databases return the nearest match if the distance is
less than some threshold.
% these examples are already in the intro
% For example, the Aadhaar system in India
% links biometrics with a unique 12 digit number with over 1 billion
% numbers issued~\cite{daugman2014600}.  Lastly, many machine learning
% algorithms depend on computing distance between points. Linear
% regression finds the line that minimizes the sum of distances between
% the line and each data point.  Similarly, the first principal component
% minimizes the sum of distances between the selected line and the
% dataset~\cite{wold1987principal}.  The remaining principal components
% are defined similarly subject to being orthogonal to previously defined
% principal components.
In this component of the proposal, we will consider encryption
algorithms which support various types of proximity queries.

\paragraph{Core operations.}
Despite the varied use of proximity functionality in databases, we find two core operations.     The first core operation is computing the distance between two points $a$ and $b$ or $d(a,b)$ according to some metric $d$.  This distance can be granular, for example $d(a, b) =15$ or coarse, $d(a,b)\in\{0,1,2\}$ corresponding to $\{equal, close, far\}$.  Even with this small set of outputs, users expect the function $d$ to behave as a metric so we restrict our attention to metric functions.  We call property-preserving encryption where $d(\enc(a), \enc(b)) = d(a,b)$ \emph{distance preserving encryption} or DPE.  Using the same language as for order-revealing and order-preserving encryption, we define a variant called \emph{distance-revealing encryption} or DRE where distance is computable  via some metric on the output space.

The second core operation takes a triple of points $a,b$ and $c$ and outputs the bit whether $d(a,c)<d(b,c)$.  This type of computation is frequently used in clustering, nearest neighbor, and learning algorithms.   We call a property-preserving encryption that achieves this functionality \emph{distance-comparison revealing encryption} or DCRE. When the distance comparison on ciphertexts uses the \emph{same} metric as on the plaintext we call \emph{distance-comparison preserving encryption} or DCPE.  The goal for this technique is to \emph{not} allow computation of $d(a,b)$.
 A related notion is \emph{approximate} distance-preserving encryption, where  $\delta_1 \cdot d(a,b) \leq d(\enc(a), \enc(b)) \leq \delta_2  \cdot d(a,b)$ for parameters $\delta_1, \delta_2$.  This can be used much like distance-preserving encryption in algorithms to give approximate guarantees.

\paragraph{Goals.}
The security achievable by DPE and DCPE is unclear even for the ideal objects.
Our research on encryption for proximity  consists of three tasks:
\begin{itemize}\setlength{\itemsep}{0em}
\item Understanding security of DPE and constructing DPE and DRE
schemes. (We expect the ideal security to vary widely based on the underlying metric
properties.)
\item Understanding the security of DCPE, and constructing DCPE and DCRE
schemes.
\item Exploring the connection to approximate distance-revealing
encryption. (We believe that approximate distance-revealing encryption
may be equivalent to distance-comparison preserving encryption.)
\end{itemize}

\subsection{DPE and DRE}

\paragraph{Prior work.}
Generalizing the work of and Li
et al.~\cite{li2010fuzzy,wang2013efficient},
Boldyreva and Chenette~\cite{boldyreva2014efficient}
defined a
\emph{closeness-preserving tagging function} where $\enc(a)$ and
$\enc(b)$ have associated tag sets.  If the tag sets intersect, this
implies that $d(a,b) \in\{equal, close\}$ with high probability;
otherwise the $d(a,b)$ is deemed to be $far$.
%This work generalizes the work of who construct a tagging function for particular string functions.  
%
The construction creates a tag for every possible neighbor and thus {\em
does not scale} to metrics where many values are considered close.
Boldyreva and Chenette also present a negative result: there is a
closeness function which \emph{requires} storage
proportional to the number of close neighbors~\cite[Theorem
5.2]{boldyreva2014efficient}.  This result is not known to hold if the
closeness function is a (well-known) metric.  

\iffalse
The approach can be summarized as follows:
\begin{enumerate}\setlength\itemsep{0em}
\item The clients appends to $\enc(a)$ a deterministic encryption of all $b$ such that $d(a,b) \le 1$.
\item To search, the client deterministically encrypts their search term $b$.
\item The server returns all ciphertexts where some encryption matches.\footnote{Some  modifications are necessary to meet ideal security such as hiding the number of neighbors.  See the prior work of Boldyreva and Chenette~\cite{boldyreva2014efficient}.}  
\end{enumerate}
\fi

Alternatively, the plaintext $a$ can be encrypted (without tags) and the client can search for the disjunction of all neighbors of $b$ (rather than just $b$).  Woodage et al.~apply this approach in the context of password authentication with typos~\cite{C:WCDJR17}.  Both of these approaches {\em require time linear in the number of neighbors} and are not viable for high dimensional data.

Lastly, the co-PI~\cite{EPRINT:ABCFG16} constructed an object called a
pseudoentropic isometric which is variant of distance-preserving
encryption.  This definition allows that
$d(\enc(a), \enc(b)) <d(a,b)$. It is possible to modify the
construction of \cite{EPRINT:ABCFG16} so that $d(\enc(a), \enc(b)) =
d(a,b)$, but this construction {\em only works for the set-difference
metric over large alphabets and requires strong cryptographic assumptions}.

\subsubsection{Proposed Work}

\paragraph{Understanding and constructing distance-preserving encryption.}
There is a scattering of work on variants of distance-revealing
encryption, but there is not a {\em solid theoretic foundation} for the
object.  Previous work defines noisy searchable encryption schemes but
never explicitly defines distance-revealing encryption.   Given the history of order-preserving encryption it is crucial to understand the security guarantees of distance-preserving and distance-revealing encryption. % In particular, we believe that distance-preserving encryption does not offer meaningful security.

%Ongoing work by the PIs indicates the viability of distance-preserving encryptionthis structure is highly metric dependent.  
The PIs first propose to study distance-revealing encryption for the Hamming metric space.
Consider the Hamming metric over $\mathcal{M}^\ell$ defined as
the number of positions in which two length-$\ell$ strings differ.
%$d(a,b) = | \{ 1\le i\le \ell | a_i \neq b_i\}|$.
Ongoing work by the PIs indicates that there is no secure
distance-preserving encryption for this metric unless the size of the
character set $|\mathcal{M}|$ is
super-polynomial.

% Define the following: 
% \begin{itemize}\setlength\itemsep{0em}
% \item Let $f: \mathcal{M}^\ell \rightarrow \mathcal{M}^\ell$ permute its input coordinates, and let $g_i: \mathcal{M}\rightarrow \mathcal{M}$ be a permutation.
% %\item Let $x\in\mathcal{M}^\ell$ be a plaintext points.  
% \end{itemize}
All distance-preserving encryption schemes
% $\enc: \mathcal{M}^\ell\rightarrow \mathcal{M}^\ell$
are of the form $\enc(x) =
f(g_1(x_1),..., g_\ell(x_\ell))$, where $f$ is a permutation of the
input coordinates and each $g_i$ is a permutation of $\mathcal{M}$.
% The permutation $f$ rotates the dimensions and $g_i$ shuffles individual dimensions.
These are the only
operations that preserve distance across the entire metric space, which
limits the potential security unless $\mathcal{M}$ is very large.

Whenever $|\mathcal{M}|$ is polynomial in the security parameter, the adversary can completely learn $\enc(x)$ with a polynomial number of plaintext/ciphertext pairs (linear for binary strings).  Prior work of the PI only provided security when $\mathcal{M}$ was superpolynomial in size~\cite{EPRINT:ABCFG16}, which our ongoing work shows is necessary.  However, the PIs also plan to investigate security of distance-preserving encryption for the Hamming metric when the adversary does not know any plaintext-ciphertext pairs (i.e., a so-called ciphertext-only attack) or a very small number of them, as was previously done for order-preserving encryption~\cite{C:BolCheONe11}.
 
 The PIs propose to study other metrics, in particular Euclidean,
 cosine-similarity, and Jaccard distance.    The PIs also plan to
 address the case of \emph{course grained} distance, e.g., where
 distance is in the set $\{far, equal, close\}$.  This problem is
 connected to property-preserving encryption for \emph{graph data}.  A
 course metric can be represented by a graph where neighboring vertices
 are near according to the metric.  Intuitively, our goal is to encrypt
 a graph as another graph with a subgraph of the same structure,
 which could lead to interesting questions in graph theory.

\paragraph{Understanding ideal distance-revealing encryption.}
In the case of \emph{ideal} distance-revealing encryption, a foundational question is whether it can be constructed from multilinear or bilinear maps.  There also may be interesting ``intermediate'' leakage profiles that are not ideal but leak less information than distance preserving encryption.   The PI has recently used this approach to find positive results for order-revealing encryption~\cite{EPRINT:CLOZ16}.

\paragraph{Distance-revealing encryption based on fuzzy extractors.}
We will construct distance-revealing encryption based on fuzzy
extractors which are a well known primitive for deriving a stable key
from a noisy source~\cite{EC:DodReySmi04}.  They consist of a pair of
algorithms: $\gen(a)$, which produces a stable key $key$ and a public
value $p$, and $\rep(b, p)$, which outputs $key$ if $d(a,b)\in\{equal,close\}$.  In ongoing work, we are constructing noisy searchable encryption from fuzzy extractors and distance-preserving encryption.  Let $\{a_i | 1\le i \le n\}$ be the set of plaintexts to be stored in the database.  Let $\enc$ be a distance-preserving encryption and let $H$ be a hash function.  Rather than directly storing $\enc(a_i)$ the client does the following:

\begin{itemize}\setlength\itemsep{0em}
\item Compute $c_i \leftarrow \enc(a_i)$ and then $k_i, p_i \leftarrow \gen(c_i)$. Send $p_i, H(k_i)$ to server.
\end{itemize}

To search for points close to $b$, the client encrypts $\enc(b)$ and presents this to the server which can rerun the fuzzy extractor for all stored terms.  The server can then return these terms to the client.  This approach has considerably less leakage than distance-preserving encryption as the server can only ``compare'' ciphertexts that are used in search.  The stored ciphertexts are not ``useful.''  We will extend this concept to remove the asymmetry between query and stored ciphertexts, transforming the construction into distance-revealing encryption.
We expect the output of this component to be 1) evidence on which metrics are suitable for DPE  and 2) constructions and analysis of DRE including constructions that use DPE.  

\subsection{Distance Comparison Revealing Encryption}
We call encryption that supports distance comparison \emph{distance-comparison revealing} encryption (DCRE).  As described above, many learning algorithms do not require direct computation of $d(a,b)$.  Rather it suffices to indicate which of two points is closer to a target point.  That is, it is sufficient to compute $d(a,c)\overset{?}<d(b,c)$.  Following the literature on order-revealing vs.~order-preserving encryption, we call the special case where ciphertexts themselves are spatial points \emph{distance-comparison preserving encryption} (DCPE).  The hope is that the weaker functionality of DCRE and DCPE can be secure for more metrics than distance-revealing and distance-preserving encryption respectively.
This leads to the following questions:

\begin{enumerate}\setlength{\itemsep}{0ex}
\item Can we design efficient DCPE?  What security can be achieved by such schemes?
\item Can we design efficient DCRE with better security?
\end{enumerate}

To answer the first question, in ongoing work we have found that
distance comparison preserving functions do not seem to have a nice
``recursive'' property as in the case of order-preserving functions,
which was crucially exploited by~\cite{EC:BCLO09}.  However, based on
computer experiments, we conjecture that \emph{distance-comparison
preserving functions are approximately distance-preserving}.    So far,
we have proven this conjecture in one dimension.  The intuition is that
as the number of points in the metric spaces increases, the number of degrees of freedom decreases.  The intuition is as follows:
\begin{enumerate}\setlength\itemsep{0em}
\item Suppose that $d(b,c) = k$ is known by the attacker, 
\item Learning that $d(a,c) < d(b,c)$ tells the attacker that $d(a,c)<k$.  
\item Suppose the attacker also knows that $d(f, c) = k/2$ and that $d(a,c) > d(f,c)$.  
\item The attacker can determine that $k/2< d(a,c) <k$.  
\item As more of these constraints are added, $d(a,c)$ is limited to smaller ranges.
\end{enumerate}
That is, the encryption mechanism reveals more accurate estimations on  the distance between $d(a,c)$.
If this conjecture is true, then for the first question we could equivalently turn our attention to the design and analysis of  an \emph{approximately distance-preserving} encryption scheme explored next.  That is, we expect a major outcome of this component to be 1) an understanding of the relationship between distance-comparison preserving encryption and approximate distance-preserving encryption (which we believe are equivalent), and 2) distance-comparison \emph{revealing} schemes (at least based on bilinear maps, hopefully even PRFs).

\subsection{Approximate Distance Revealing Encryption}
\label{sec:adre}
\paragraph{Prior work.}
The recent GRECS work allows minimum distance between any pair of
points~\cite{CCS:MKNK15}.  Rather than storing or computing distance
between pairs of points $a,b$, the work uses a primitive called a
sketch-based oracle.\footnote{This sketch-based oracle is different from
the notion of a secure sketch used in fuzzy extractor constructions.}  A logarithmic number of reference points $r_1,..., r_k$ are selected and the distance between $r_i$ and every point in the graph is computed and encrypted.  This structure is transmitted to the server.  Then at query time the client encrypts their pair $a,b$.  The server finds all reference points which are connected to $a,b$ and returns $\min_{r_i, r_j} d(a,r_i) + d(r_i, r_j) + d(r_j, b)$.  Given a careful selection of the reference points this distance can be shown to approximate the minimum distance between $d(a,b)$.

A second line of work uses locality-sensitive hashes which are designed to allow more efficient computation of nearest neighbor in high dimensional spaces~\cite{datar2004locality,slaney2008locality}.  A locality sensitive hash is function that is more likely to have collisions when two inputs are ``close'' in the input space.
% \begin{definition}
%   Let $(R,d)$ be a metric space.  A family of functions $\mathcal{H}$ is a $(r_1, r_2, p_1, p_2)$ locality sensitive hash (where $r_1< r_2$ and $p_1 >p_2$) if for any $x, y$ the following hold:
% \begin{itemize}\setlength\itemsep{0em}
% \item If $d(x, y) \le r_1$ then $\Pr_{h\leftarrow \mathcal{H}}[h(x) = h(y)] \ge p_1$, and 
% \item If $d(x, y) \ge r_2$ then $\Pr_{h\leftarrow \mathcal{H}}[h(x) = h(y)] \le p_2$.
% \end{itemize}
% \end{definition}

Kuzu, Islam and Kantarcioglu~\cite{kuzu2012efficient} use locality sensitive hashing to create an approximate distance-preserving data structure.  To create the index, for plaintext $a$ the client samples multiple locality sensitive hashes $h_1(a), h_2(a),...,h_k(a)$ and associates each output as a keyword with $a$ using deterministic encryption.
Then to query for $b$ the client queries computes $h_1(b),...,h_k(b)$. They then ask the server for all results that match
a single hash.% (this requires the ability to perform disjunctive queries).%
\footnote{To achieve this, they build an inverted index for each
locality sensitive hash that allows them to retrieve the document
identifiers.}  The client locally retrieves results and then restricts
to those documents that have a high number of hash matches.  With good
probability, these will correspond to those records that were close to
the original query.  Bringer et
al.~\cite{bringer2011identification,bringer2009error} use similar
techniques but insert the output of the locality sensitive hash into a
Bloom filter.  Both of these works provide relatively weak security and
make no attempt to hide records that match a small number of hash
function outputs.  

\paragraph{Proposed work.}
Distance-preserving functions are easy to characterize geometrically, in terms of a scaling factor plus flips, rotations and reflections.  To approximately preserve distance, we can also  ``perturb'' each image point within a ball of given radius.  
We can show that independent random such perturbations yields a function that, while not strictly DCPE, is \emph{approximately} so, and that encrypting via an approximately DCPE function still guarantees accuracy of nearest neighbor  search within the approximation.  Moreover, as such perturbations can easily be derandomized, this gives an efficient \emph{approximate} DCPE scheme from PRFs.  Finally, we will conduct a separate analysis in the spirit of~\cite{C:BolCheONe11} to answer the question of what privacy such a scheme provides.  

 A related question we plan to investigate is the privacy achievable using locality-sensitive hashing to perform nearest neighbor algorithms.  To our knowledge, prior work has not explicitly defined privacy properties for locality-sensitive hashing.  Moreover, the scheme of Kuzu et al.~\cite{kuzu2012efficient}  does not approach ideal security as the server learns which records match each subfeature.  In ongoing work, we are combining locality-sensitive hashing with the recent \emph{sample-then-hash} fuzzy extractor construction of the co-PI~\cite{EC:CFPRS16}.  The idea is as follows:

\begin{enumerate}\setlength\itemsep{0em}
\item For input $a$, compute $a_1 = h_1(a), a_2 = h_2(a),..., a_k = h_k(a)$.
\item For $i=1,...., \ell$,
sample $i_1,..., i_\eta\overset{\$}\leftarrow [k]$, compute $\alpha_i =
H(a_{i_1} || ... || a_{i_\eta})$, and append $\alpha_i$ as keyword to $a$
(using deterministic encryption).
\end{enumerate}
\noindent
This approach appends multiple locality sensitive hashes that individually have a small probability of matching but overall there is a high probability of a single match.  The probability of finding any matches between $a,b$ that are not close is small (see Canetti et al.~\cite{EC:CFPRS16}).
%\textbf{Open problems discussion:} 
%\begin{itemize}
%\item Understand the set of LSH and how they can be used for practical noisy sources.
%\item Is there any power to interactive solutions?  Everything I've seen is single message each way.
%\item Integrate LSH and an m-out-of-n search scheme to mitigate the leakage associated with the Kuzu scheme.
%\item Design a stronger primitive than LSH that has some entropy preservation properties so we can talk about what cost there is in revealing when LSH collisions occur.  Adam's student did some work on distance preserving transforms.  Ben is thinking about this as well.  
%\item None of these schemes talk about updates.  Is there anything interesting to consider there?
%\end{itemize}
%
%\paragraph{Prior Work}
%
%\ben{These are my notes from last year, still need to condense}
%
%Work by Li et al.~\cite{li2010fuzzy,wang2013efficient} that was published at INFOCOM 2010.  
%
%\paragraph{Summary of Techniques}  Here we focus on how they deal with noise.  We ignore the searching index structure in the rest of their work.  This work is compatible with an arbitrary index structure.  The scheme is built on a system that can handle keyword equality.  The basic idea of this paper is to expand noise into a set of possible neighbors.  Three strategies are proposed:
%
%\begin{enumerate}
%\item When inserting a keyword, for example `Alice', also insert all of the neighbors into the index structure.  For example if considering Hamming distance of $1$, insert `Blice', `Clice', etc.  This technique results in an index structure size proportional to the number of neighbors.  Furthermore, the number of neighbors is leaked to the server as well documents that have neighboring keywords.  However, search proceeds the same way as in standard equality search.  The authors note that this approach is not tenable for any large amount of noise.
%\item The second technique is a hybrid technique where both the inserted keyword set and the searched queries are modified.  Instead of inserting `Alice', `Blice', etc. the following keywords are inserted `*lice', `A*lice', etc.  Then when performing the search the querier with the word `Alica' asks all documents containing one of `*lice', ..., `Alic*', etc.  Note that this approach requires as underlying scheme that can handle disjunctive queries.  As such there may be more leakage associated with this approach.
%\item The third approach is to insert word GRAMS.  The idea is to insert all possible nearby substrings of a particular length.  I didn't really follow their discussion well on why this was better than the wildcard based approach.  But this technique seems to be similar to what is being done in conjunctive search system out of IBM research.
%\end{enumerate}
%
%\paragraph{Security}
%This work never presents a formal definition of security.  They roughly say that the server is allowed to learn ``the outcome and the pattern of search queries.''  They claim to be using the security definition of Curtmola et al.~\cite{curtmola2011searchable}.  For their basic scheme they leak when two documents have the same neighbor which is the same as leaking if they are within twice the noise tolerance.  
%
%\paragraph{Major gaps}
%This paper presents a relatively weak efficiency guarantees.  All of the schemes involve exhaustively listing the neighbors in one form or another.  Furthermore, the number of neighbors and the neighborhood pattern of all documents is revealed.


%\subsection{Component A-III:  Property-Preserving Encryption for Graph Data} 
%
%For graph data, shortest path and disease propagation queries are common.  Previous work looks at supporting shortest path queries in the context of searchable symmetric encryption.  Accordingly, we propose to investigate \emph{shortest path revealing  and preserving encryption} (SPRE and SPPE) and \emph{random walk revealing and preserving encryption} (RWRE and RWPE).   We believe that the study of such encryption schemes will lead to interesting questions in graph theory.  We will particularly try to leverage work on differentially private graph sanitization. 
%
%Regarding shortest path revealing schemes: 
%
%\begin{question}
%Can we design efficient SPPE?  What security can be achieved by such schemes?
%\end{question}
%
%
%\begin{question}
%Can we design efficient SPRE with better security?
%\end{question}
%
%\begin{question}
%Can we design ``partial'' SPPE with better security?
%\end{question}
%
%And then regarding random walk preserving schemes:
%
%\begin{question}
%Can we design efficient RWPE?  What security can be achieved by such schemes?
%\end{question}
%
%
%\begin{question}
%Can we design efficient RWRE with better security?
%\end{question}
%
%\begin{question}
%Can we design ``partial'' RWPE with better security?
%\end{question}




\iffalse
\subsection{Component A-II:  Property-Preserving Encryption for Time Series Data} 

In time series data one is often interested in correlations and anomalies.  Accordingly, we propose to look at \emph{correlation revealing and preserving encryption}  (CRE and CPE) and \emph{anomaly revealing and preserving encryption} (ARE and APE).  In other words, in the ``preserving'' case we are interested in perturbing statistical data in a way that preserves statistics or the fact that a point is an anomaly.  

Regarding correlation revealing schemes
\begin{question}
Can we design efficient CPE for correlations of interest?  Which correlations should we target?  What security can be achieved by such schemes?
\end{question}


\begin{question}
Can we design efficient CRE with better security?
\end{question}

\begin{question}
Can we design ``partial'' CPE with better security?
\end{question}


Regarding anomaly  evealing schemes
\begin{question}
Can we design efficient APE?  How should anomaly thresholds be set?  What security can be achieved by such schemes?
\end{question}


\begin{question}
Can we design efficient ARE with better security?
\end{question}

\begin{question}
Can we design ``partial'' APE with better security?
\end{question}

\subsection{Component A-III:  Property-Preserving Encryption for Graph Data} 

For graph data, shortest path and disease propagation queries are common.  Previous work looks at supporting shortest path queries in the context of searchable symmetric encryption.  Accordingly, we propose to investigate \emph{shortest path revealing  and preserving encryption} (SPRE and SPPE) and \emph{random walk revealing and preserving encryption} (RWRE and RWPE).   We believe that the study of such encryption schemes will lead to interesting questions in graph theory.  We will particularly try to leverage work on differentially private graph sanitization. 

Regarding shortest path revealing schemes: 

\begin{question}
Can we design efficient SPPE?  What security can be achieved by such schemes?
\end{question}


\begin{question}
Can we design efficient SPRE with better security?
\end{question}

\begin{question}
Can we design ``partial'' SPPE with better security?
\end{question}

And then regarding random walk preserving schemes:

\begin{question}
Can we design efficient RWPE?  What security can be achieved by such schemes?
\end{question}


\begin{question}
Can we design efficient RWRE with better security?
\end{question}

\begin{question}
Can we design ``partial'' RWPE with better security?
\end{question}
\fi

%\section{Component B:  Property-Preserving Encryption for Time Series Data} 

In time series data one is often interested in correlations and anomalies.  Accordingly, we propose to look at \emph{correlation revealing and preserving encryption}  (CRE and CPE) and \emph{anomaly revealing and preserving encryption} (ARE and APE).  In other words, in the ``preserving'' case we are interested in perturbing statistical data in a way that preserves statistics or the fact that a point is an anomaly.  

Regarding correlation revealing schemes
\begin{question}
Can we design efficient CPE for correlations of interest?  Which correlations should we target?  What security can be achieved by such schemes?
\end{question}


\begin{question}
Can we design efficient CRE with better security?
\end{question}

\begin{question}
Can we design ``partial'' CPE with better security?
\end{question}


Regarding anomaly  evealing schemes
\begin{question}
Can we design efficient APE?  How should anomaly thresholds be set?  What security can be achieved by such schemes?
\end{question}


\begin{question}
Can we design efficient ARE with better security?
\end{question}

\begin{question}
Can we design ``partial'' APE with better security?
\end{question}



\section{Allowing Interaction}
\label{sec:interactive}
%!TEX root = dbproposal.tex

\subsection{Extensions to Property-Preserving Encryption}
As stated above, the direct use of property-preserving encryption has a checked history with leakage attacks showing that deterministic and order-preserving encryption reveal the entire stored dataset for many applications (see work of the co-PI for an overview of leakage attacks~\cite{SP:FVYSHG17}).  However, PPE has many benefits that should not be overlooked by the research community.  PPE is easier to configure, use, deploy, does not require replacement of the entire software stack, and gives admins flexibility on the underlying database technology.  Given the momentum of PPE in the commercial sector, researchers should understand if a PPE solution can be used safely while retaining the benefits of the approach.  Our approach keeps the intact the design principal of PPE: the only place that the database should have to change is the comparison operator (equality, comparison, or distance).  The database should still be able use standard indexing mechanisms.  We note it is possible to override this operator to be interactive and require help from a client without altering the overall indexing structure.  Before describing our approach, we briefly describe prior work in the area.

\paragraph{Prior Work}
We are aware of two main works that follow the approach of PPE with added interactivity.  A co-PI recently introduced a new cryptographic approach, called POPE, to
supporting range queries over encrypted data, providing stronger security than
previous order-preserving encryptions~\cite{CCS:RACY16}.  In contrast to
existing OPE schemes, the server builds a novel indexing structure called a
POPE tree, in which each node has a {\it unsorted buffer} and a sorted list of
elements.  Thanks to this, the scheme can perform {\it lazy indexing, by
sorting values only when necessary}. In particular, on each range query the
scheme sorts the part of the data that is accessed during the search, leaving
much of the data untouched.

The other work is Arx due Poddar, Boelter, and Popa~\cite{EPRINT:PodBoePop16}.  The
Arx protocol builds an index for answering
range queries without revealing all order relationships to the
server. The index stores all encrypted values in a binary
tree so range queries can be answered by traversing this
tree for the end points. Using Yao's garbled circuits, the
server traverses the index without learning the values it is
comparing or the result of the comparison at each stage.  Since each garbled circuit can only be used once, as order operations are computed the client and server need to work together to prepare garbled circuits (and oblivious transfer).

\paragraph{POPE for Similarity of High-dimensional data}
Our main approach will be developing PPE that is just in time using advice from the client.  In the case, of POPE this took the form of asking the client to sort a small number of nodes to build out a tree and then comparing ciphertexts only with those nodes.  This limited leakage to comparison between the dataset and these nodes.  Roughly, this can be thought of as reducing leakage from a quadratic number of comparisons to a linear number.  Our first task will be extend the POPE paradigm to high dimensional data.  The approach uses random hyperplanes which have previously been used in the concept of locality-sensitive hashing~\cite{charikar2002similarity}.  The idea is as follows:

\begin{enumerate}
\item The server initially stores a unsorted buffer of the entire dataset.
\item The client initiates a search for items that are similar to $a$.
\item The servers asks the client to split the tree.
\item The client generates a random hyperplane $x$.  
\item The server asks the client to indicate $\sign(<x^T,b>)$ for each stored element $b$. In addition, the client indicates $\sign(<x^T,a>)$.
\item The client and server repeat the process with the relevant subtree.
\end{enumerate} 

This basic protocol above is not ready for deployment.  First, it assumes the creation of a binary tree rather than a B-tree with many children per tree node. The second problem is that random hyperplanes have a substantial error probability.  There will be many nodes $b$ such that $d(a, b)=close$ such that $a$ and $b$ lie on different sides of the selected hyperplane.  To deal with this problem, the client can select a collection of hyperplanes $x_1,..., x_k$ and split the tree based on the magnitude of $\sum_{i=1}^k \sign(<x_i^T, b>)$.  This technique allows us to control the probability that $a$ and $b$ will be denoted as far (lying in different subtrees) when they are close (using tail bounds for the binomial distribution).  Furthermore, it also creates a larger number of children for each node pushing from a binary to a B$^+$-tree.

\paragraph{POPE for substring search}
Our second task is to construct substring search with limited leakage.  Substring search has previously been addressed directly in work by Chase and Shen~\cite{chase2015substring} and Moataz and Blass~\cite{EPRINT:MoaBla15} and by augmented keyword equality~\cite{RSA:IKLO16} and Boolean formula search~\cite{CCS:JJKRS13}.  In this proposal, we focus on extending the approach of Chase and Shen.  Their approach was to build an encrypted suffix tree.  A suffix tree is a commonly used data structure for substring search~\cite{mccreight1976space}.  Our idea is to build the suffix tree just in time using ideas from the POPE protocol.  Essentially, the client will build a single level of the suffix tree.  When the client searches for a string $a$ they will traverse the partially constructed suffix tree with the client/server interactively building out the tree as necessary.  This approach will require augmentations to a traditional suffix tree as the original searchable string must be stored and queryable to build the tree on demand.  Balancing the speed and privacy of querying the original string represents an important tradeoff in this approach.
%!TEX root = dbproposal.tex

\subsection{Extending BlindSeer}
\label{sec:ext-blindseer}

\paragraph{Prior work.} A co-PI was a member of the team that developed
BlindSeer~\cite{SP:PKVKMC14,SP:FVKKKM15}, which is a database management system
that provides privacy.
%
BlindSeer has three main players. The {\em server} $\sf S$,
who holds the DB and outsources an encrypted copy of the DB to a third party,
called the {\em index server} $\sf IS$. The server $\sf S$  also builds an
encrypted Bloom filter (BF) search tree to the DB and sends it to $\sf IS$. The
{\em client} $\sf C$ sends search queries to $\sf IS$ and obtained encrypted
results, to each of which $\sf C$ obtains the decryption key secret-shared
between $\sf C$ and $\sf IS$. The decryptions keys are arranged in the offline
setup stage, using shuffling and homomorphic public-key
encryption.
%
BlindSeer provides the following privacy guarantees:
\begin{itemize}\setlength\itemsep{0em}
\item Query privacy: $\sf S$ is not involved with the search protocol at
  all, so the client's queries are hidden from $\sf S$. Moreover, $\sf IS$ holds
    the {\em shuffled, encrypted} database, so the client's queries remain
    private to a great degree against $\sf IS$. 

\item Data privacy: $\sf IS$ deals with only the encrypted database, so the
records are hidden from $\sf IS$. Moreover, $\sf IS$ returns only the records
that satisfy the client query, and therefore, $\sf C$ cannot arbitrarily access
all the records in the database.  
\end{itemize}

\subsubsection{Proposed Work}
When a database is outsourced, the database owner is often the client
itself, in which case we don't need to worry about whether the plaintext data
would be leaked to the client.  In this proposal, we will consider this setting
of the client as the data owner and significantly simplify the architecture of
the BlindSeer system. This will bring us a construction that has better
efficiency and a richer set of functionalities. In particular:

\begin{itemize}\setlength\itemsep{0em}
\item {\em Much simpler, efficient setup.}
In this new simpler threat model, the client can be regarded as playing as both
$\sf S$ and $\sf C$ in the original BlindSeer architecture. Therefore, the
client can just hold one symmetric key in contrast to the original BlindSeer
system where a slow public-key encryption scheme and a random shuffling
must be introduced in the setup stage, none of which we need in our new
setting. This also give us {\em better backward compatibility}.

\item {\em MPC is not necessary.} In the original BlindSeer system, in order to
hide from $\sf C$ the BF data stored in each node of the search tree, $\sf
C$ and $\sf IS$ have to execute costly MPC computation. In our setting, the
    client can simply ask for the necessary encrypted BF data and decrypt them,
    since we don't need to hide anything from the client.  


\item {\em Dynamic record insertion.} In the original BlindSeer, one cannot
insert new encrypted BF data to the search tree dynamically.  It is mainly
because the decryption key for each record must be arranged in the setup
stage.
%
Furthermore, in the original BlindSeer, BF data are encrypted with a very
simple mechanism of one-time pad so that MPC computation may be reasonably
efficient. Dynamic addition of BF data will lead to a more complicated
encryption structure, and the MPC computation will be much more costly.
In our setting, however, we don't need any key setup, nor MPC computation.
Therefore, we can add BF data directly to the search tree with much more
efficiency.  
\end{itemize}

\noindent We believe that it is important to provide an efficient solution for various
settings. For use cases where this new setting is appropriate, we believe this simplified
construction will be an order of magnitude faster than the current BlindSeer system.

\paragraph{Proximity of high-dimensional data.}
BlindSeer already provides conjunctions and range queries.  These queries can be combined to answer proximity queries on a Euclidean space.
In particular, a query that searches for the points close to $(x, y)$ can be defined as a 2D rectangle defined by the top-left point $(x_1, y_1)$ and the
bottom-right point $(y_1, y_2)$ can be described with range queries and
conjunctions as follows: $$ x_1 \le x \le x_2 ~~{\sf AND}~~~ y_1 \le y \le y_2.
$$ This technique can be easily extended to data with multiple dimensions. 

\paragraph{Graph structure and the shortest path.}
Using Bloom filters we can encode a graph structure as follows:
\begin{itemize}
\item For each edge $(a, b)$ from vertex $a$ to vertex $b$, we insert an
  encryption $c = \Enc((a,b) \| w(a,b) )$, where $w(a,b)$ is the weight of the edge
    $(a,b)$. Moreover, have the ciphertext $c$ indexed by a BF keyword `$\sf edge$:a*'. 
\end{itemize}

\noindent
We can find all the neighbors of vertex $a$ along with the weight of the
associated edge. Therefore, we can run variants of Dijkstra's algorithm and
compute compute the shortest path, given two vertices $a$ and $b$ and a limit
on the maximum number of arcs. The key to this approach is the ``native''
ability of Bloom filters to handle conjunction queries.  BlindSeer is based on
the Bloom filters so it can support proximity and graph structure by adding
relevant keyword index appropriately.
%
%Moreover, considering that BlindSeer allows disjunctions and conjunctions, the
%client can enjoy a much richer set of functionalities.  



\section{Analysis and Comparison}
\label{sec:analysis}
%!TEX root = dbproposal.tex

In this section of the proposal, we outline our efforts to analyze,
implement, and compare the most promising schemes from the previous two
sections. These efforts will be \emph{concurrent} and \emph{ongoing}
with the theoretical developments.�A main goal of our approach is to create cryptographic search systems that will be deployable in the future.  Thus, it is critical to understand
the practical implications of different approaches.

The main goals of this thrust of our work is to put our schemes in
context of existing approaches and provide clear comparisons for
practitioners and researchers alike.
The work of this section will also feature the most prominent engagement
with student researchers, particularly
undergraduate students at our respective institutions.

\subsection{Leakage analysis}

All of the schemes we have proposed make some compromise of leakage for
performance. Unfortunately, many prior works either focus primarily on security
proofs and (sometimes novel) security definitions, or make heuristic
arguments for security. It is rarely clear how to compare the leaked information.  This means it is not
straightforward, for example, to decide what is ``more secure'' between
different approaches.  This difficulty in comparing is not only a semantic gap. Sometimes leaked information is damaging for a particular use case and innocuous for another.  This ambiguity suggests the need for standard and comprehensive benchmarks.

One of PIs made progress towards fair comparisons in a recent
Systemization of Knowledge~\cite{SP:FVYSHG17}. We will continue in this vein and use existing
metrics whenever possible to place our work in a fair context within the
state of the art.  Two main components of this fair comparison are 1) the use of  consistent naming for leakage and 2) maintaining a partially-ordering of schemes' leakage.

For this analysis to be meaningful, cryptographic designers must be cognizant of
recent attacks on PPE and related schemes such as
\cite{CCS:CGPR15,CCS:KKNO16}. These attacks often depend crucially on
the datasets used and the assumed prior knowledge, and we will use the same (or equivalent) attacks
against our schemes to provide a meaningful comparison.

Much of the work of this proposal is in extending existing
non-interactive and interactive schemes to the multi-dimensional setting
via some support of distance queries. Some existing metrics can apply
directly in this setting, for example the notion of \emph{incomparable
pairs} introduced in \cite{CCS:RACY16}. Other notions related to
closeness, i.e., \emph{pairwise distance}, also make sense in a
Euclidean space. Rather than develop entirely new definitions, when
possible we will use existing metrics to quantify the security
improvements our schemes provide.

\subsection{Implementation and experimental analysis}

\emph{Throughout all phases of the project}, the most promising of our
schemes will be implemented and tested on realistic scenarios.
Indeed, we plan to include
experimental components in many of our published artifacts, with
links to the open source implementations developed.

Compatibility with existing database software is a key priority of
our approach.  We will seek to incorporate our schemes within existing
open-source projects such as the SSE library Clusion
(\url{https://github.com/encryptedsystems/Clusion}) and
the cloud-based graph database software
Apache Rya (\url{https://rya.apache.org/}).

In order to evaluate the real-world performance of our schemes, we will
be careful to use realistic and relevant datasets and queries. Tools
such as the automatic test-suite generator developed in the IARPA SPAR
project \cite{HH14,varia2015automated} will be used to generate
repeatable and realistic experiments.  As necessary, we will extend these tools to support proximity databases.  We expect to extend both data and query generation.  As possible, these changes will be fed back into the existing open-source projects.

Our experimental evaluations will also cover the \emph{security} of our
schemes. The approaches we propose all
entail some limited information leakage.
We will use empirical tools to highlight the practical
implications of this limited leakage, including leakage under known
attacks. For example, in \cite{CCS:RACY16}, two of the co-PIs not only
evaluated the number of incomparable ciphertext pairs in theory, but
also tested this leakage on a publicly-available salary database.  Working with the community, we will identify a small set of datasets that highlight the different leakage profiles in the literature.  We expect these datasets can be used to effectively communicate the tradeoffs between different schemes.


\section{Prior Accomplishments and NSF Support}

%\paragraph{Adam O'Neill:}
\noindent {\bf Adam O'Neill:}
In his Ph.D.~work, the PI  developed the notions of deterministic encryption~\cite{C:BelBolONe07,Amanatidis2007,C:BolFehONe08,C:BFOR08,TCC:FulNeiRey12} and order-preserving encryption~\cite{EC:BCLO09,C:BolCheONe11} to help enable search on encrypted data with processing time comparable to that for unencrypted data, while providing as-strong-as-possible security guarantees subject to this constraint.
The PI has also worked on  instantiating random oracles~\cite{C:KilOneSmi10,TCC:GoyONeRao11,EC:LewONeSmi13},   aggregate signatures~\cite{CCS:BGOY07,AC:GLOW12}, deniable encryption~\cite{C:OnePeiWat11},  chosen-ciphertext security~\cite{EC:KilMohOne10,PKC:DFMO14}, and  functional encryption~\cite{EPRINT:ONeill10b,C:DIJOPP13, CANS:BelONe13}.   %Furthermore, he collaborated with database faculty George Kollios on refinements to order-preserving encryption~\cite{}.
Since joining Georgetown, he has also been working on  applications of indistinguishability obfuscation~\cite{PKC:DGLOZ16} and on integrating cryptography with emerging applications, such as outsourced database systems using modular order-preserving encryption~\cite{mavroforakis2015modular} and  privacy preserving network provenance using structured encryption~\cite{zhang2017privacy}.


Prior support: ``EAGER: Guaranteed-Secure and Searchable Genomic Data Repositories.'' (PI). Proposal Number 1650419.  2016 - 2017. \$99,999.
``Program Obfuscation: From Theory to Practice." NSF Research Experiences for Undergraduates Supplement (PI).
Supplement to Award \#IIP-1362046,   2014 - 2019, \$8,000.

%\paragraph{Benjamin Fuller:}
\noindent {\bf Benjamin Fuller:}
In his Ph.D.~work, the co-PI worked on deterministic encryption with Dr. O'Neill \cite{TCC:FulNeiRey12,JC:FulONeRey15}.  His main focus was on cryptography with noise developing new fuzzy extractors~\cite{AC:FulMenRey13,AC:FulReySmi16,EC:CFPRS16}.  Fuzzy extractors can be thought of as a special case of distance preserving encryption where only a single point is comparable.  He then oversaw evaluation and implementation of encrypted search systems at MIT Lincoln Laboratory as part of the IARPA SPAR project~\cite{spar_baa} including BlindSeer codeveloped by Dr. Choi.  Since joining the University of Connecticut in 2016, his work has focused on driving cryptography to practice including authentication and fuzzy extractors~\cite{EPRINT:HFDD17,EPRINT:BKFY17,EPRINT:ABCFG16}, secure outsourced databases~\cite{SP:FVYSHG17}, and multi-party computation~\cite{EPRINT:CunFulYak16}.  

Prior NSF support: not applicable.

%\paragraph{Seung Geol Choi:}
\noindent {\bf Seung Geol Choi:}
The co-PI is mainly interested in achieving privacy in practice. As for the
works directly related to this proposal, he participated in the IARPA SPAR
project~\cite{spar_baa} as a member of the team of Columbia University and Bell
Labs to build the BlindSeer system~\cite{SP:PKVKMC14}, and he also recently
introduced a new system, called POPE, that supports range queries over
encrypted data~\cite{CCS:RACY16}.  

He has also worked on topics related to this proposal such as
ORAMs~\cite{SP:RocAviCho16,NDSS:ACMR17,CCS:RACM17}, secure multi-party
computation~\cite{AC:CEJMY07,TCC:CDMW09,AC:CEMY09,RSA:CHKMR12,TCC:CKKZ12,PKC:CKWZ13,C:CKMZ14,TCC:CKSYZ14},
and various encryption schemes~\cite{TCC:CDMW08,AC:CDMW09,AC:LCLPY13}. 

Prior NSF support: ``RUI: Achieving Practical Privacy for the Cloud.'' (co-PI) Award number 1618269, 2016-2019, \$355K. 

%\paragraph{Daniel S.\ Roche:}
\noindent {\bf Daniel S.\ Roche:}
This co-PI comes from an algorithms background, having worked
extensively in the area of computer algebra and publishing frequently in
the top venues of that area
\cite{Roc09,Roc11,GR10,GR11,GR11a,GRT10,GRT12,HR10,AGR14,AR14,AGR15}. Recently, his interests have turned to
developing improved algorithms and data structures for ensuring privacy
in remote storage, which is closely related to the topic of this
proposal
\cite{SP:RocAviCho16,CCS:RACY16,NDSS:ACMR17,CCS:RACM17}.

The co-PI has a proven track record of working with undergraduates and
graduate students at other institutions,
including multiple publications from such collaborations
\cite{AGR13,AGR14,AGR15,AR14,KRT15,GR16}.

Prior NSF support: ``AF: Small: RUI: Faster Arithmetic for Sparse
Polynomials and Integers.'' (PI) Award number 1319994, 2013-2016, \$123K.

Prior NSF support: ``RUI: Achieving Practical Privacy for the Cloud.''
(co-PI) Award number 1618269, 2016-2019, \$355K.

\begin{table}[t]
\footnotesize
\noindent \begin{tabular}{l | c | c | c | c | c | c | c | c | c | c | c | c}
Thrust & \multicolumn{4}{c}{Year 1} & \multicolumn{4}{c}{Year 2} & \multicolumn{4}{c}{Year 3} \\
& Q1 & Q2 & Q3 & Q4 & Q1 & Q2 & Q3 & Q4 & Q1 & Q2 & Q3 & Q4\\\hline
Distance-Preserving Encryption\\
\,\,\,\,Security of Ideal DPE (O'Neil, Fuller) & \multicolumn{4}{c}{\cellcolor{blue!25}}\\
\,\,\,\,DRE (Fuller, O'Neil) & \multicolumn{3}{c}{} &\multicolumn{4}{c}{\cellcolor{blue!25}} \\
\,\,\,\,Approximate DRE (O'Neill, Fuller, Roche) & \multicolumn{5}{c}{} &\multicolumn{5}{c}{\cellcolor{blue!25}} \\
Allowing Interaction \\
\,\,\,\,POPE: functionality (Roche, Choi) &\multicolumn{4}{c}{\cellcolor{blue!25}} \\
\,\,\,\,POPE: Enhanced Security (Roche, Choi, O'Neil) & \multicolumn{3}{c}{} & \multicolumn{3}{c}{\cellcolor{blue!25}}\\
\,\,\,\,Simplify Blind Seer (Choi, Fuller) & \multicolumn{4}{c}{} & \multicolumn{4}{c}{\cellcolor{blue!25}} \\
\,\,\,\,ORAM \& Blind Seer (Choi, Fuller, O'Neil) & \multicolumn{7}{c}{}& \multicolumn{3}{c}{\cellcolor{blue!25}}\\
Impl./Eval. (O'Neil, Fuller, Choi, Roche) & \multicolumn{5}{c}{} &  \multicolumn{7}{c}{\cellcolor{blue!25}}\\
\end{tabular}
\caption{Work schedule}
\label{tab:work}
\end{table}
\section{Schedule and Management Plan}
The work described in this proposal will be performed by a combination of the PIs, graduate students at Georgetown University and University of Connecticut, and undergraduate students at all three institutions.  A collaboration plan is attached as supplementary material.  In this section, we provide a brief overview of the timeline of the proposed research.  For each task, we have identified the lead PI for that task but we expect all tasks to involve collaboration between all PIs.  The work is summarized in Table~\ref{tab:work}.

\clearpage
\bibliographystyle{alpha}
\bibliography{cryptobib/abbrev3,cryptobib/crypto,Bencrypto,dan}





\end{document}

