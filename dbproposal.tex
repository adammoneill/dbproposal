\documentclass[11pt]{article}

\usepackage{fullpage,varwidth,url}
\usepackage{amssymb}
\usepackage{amsmath,amsthm}
\usepackage{relsize}
\usepackage{color}
%
%    \setlength{\evensidemargin}{-0.2in}
%    \setlength{\oddsidemargin}{-0.2in}
%    \setlength{\textwidth}{6.9in}
%    \setlength{\textheight}{9.5in}
%    \setlength{\topmargin}{-0.35in}
%    \setlength{\headheight}{0in}
%    \setlength{\headsep}{0in}
%    \setlength{\footskip}{0.5in}
\title{{{\Large{TWC: Small:  Next-Generation Secure Outsourced Databases}}}}
%\author{Sanjam Garg \\University of California, Berkeley \and 	Mohammad Mahmoody \\ University of Virginia\and Adam O'Neill \\Georgetown University}


%\usepackage{anysize} % to set up document margins.
%\marginsize{1.1in}{1.1in}{1.1in}{1.0in}
%\vspace{-1mm}
\usepackage{times}
\usepackage{etex}
\usepackage{framed}

\newcommand{\Att}{\textsc{Attack}}
\newcommand{\Hyb}{\textsc{Hybrid}}
\newcommand{\Idl}{\textsc{Ideal}}
\newcommand{\Lzy}{\textsc{Lazy}}


\newcommand{\OR}{\mathsf{OR}}
\newcommand{\AND}{\mathsf{AND}}


\newtheorem{thm}{Theorem}[section]
\newtheorem{lem}[thm]{Lemma}
\newtheorem{cor}[thm]{Corollary}
\newtheorem{propo}[thm]{Proposition}
\newtheorem{clm}[thm]{Claim}
\newtheorem{defn}[thm]{Definition}
\newtheorem{conj}[thm]{Conjecture}
\newtheorem{assm}[thm]{Assumption}
\newtheorem{rem}[thm]{Remark}
\newtheorem{obs}[thm]{Observation}
\newtheorem{egs}[thm]{Example}
\newtheorem{fct}[thm]{Fact}
\newtheorem{expr}{Experiment}
\newtheorem{cons}[thm]{Construction}
\newtheorem{nte}[thm]{Note}
\newtheorem{aim}[thm]{Aim}

% keys mathsf
%\newcommand{\sk}{\mathsf{sk}}
%\newcommand{\SK}{\mathsf{SK}}
%\newcommand{\pk}{\mathsf{pk}}
%\newcommand{\PK}{\mathsf{PK}}
%\newcommand{\dk}{\mathsf{dk}}
%\newcommand{\DK}{\mathsf{DK}}


%\newcommand{\att}{\cA}
%\newcommand{\prd}{\cP}


%%%%%%%%%%%%%%%%%%%%%%%%%%%%%%%%%%%%%%%%%%%%%%%%%%%55
\newcommand{\remove}[1]{}
\newcommand{\num}[1]{{\bf (#1)}}

\newcommand{\procfont}[1]{\mathsc{#1}}
\newcommand{\tablefont}[1]{\mathsf{#1}}

\newcommand{\mpk}{\pi}
\newcommand{\msk}{\varfont{msk}}
\newcommand{\usk}{\varfont{dk}}
\newcommand{\usknew}{\overline{\varfont{dk}}}
\newcommand{\ctxt}{\varfont{C}}
\newcommand{\ctxtnew}{\overline{\varfont{C}}}
\newcommand{\IBE}{\schemefont{IBE}}
\newcommand{\IBEnew}{\overline{\schemefont{IBE}}}
\newcommand{\mskspace}{{\cal S}}
%\newcommand{\kg}{{\cal K}}
\newcommand{\kg}{\mathsf{Kg}}
\newcommand{\kgnew}{\overline{{\cal K}}}
%\newcommand{\pg}{{\cal P}}
\newcommand{\pg}{\mathsf{Pg}}
\newcommand{\Enc}{\enc}
\newcommand{\Encnew}{\encnew}
\newcommand{\Dec}{\decc}
\newcommand{\Decnew}{\deccnew}

\def\vk{\mathit{vk}}
\def\pars{\pi}
\def\ek{\varfont{ek}}
\def\pk{\varfont{pk}}
\def\sk{\varfont{sk}}
\newcommand{\vecsk}{\mathbf{sk}}
\newcommand{\vecpk}{\mathbf{pk}}
\newcommand{\vecm}{\mathbf{m}}
\newcommand{\vecM}{\mathbf{M}}
\def\dk{\varfont{dk}}
\def\ak{\varfont{ak}}
\def\stateinfo{\varfont{st}}
\def\st{\varfont{state}}
\def\stt{\varfont{St}}
\newcommand{\RO}{H}
\newcommand{\keygen}{KG}
\newcommand{\pargen}{PG}
\newcommand{\trapdoor}{Td}
\newcommand{\test}{Test}
%\newcommand{\setup}{Setup}
\renewcommand{\AE}{\mathcal{AE}}
%\newcommand{\enc}{\mathcal{E}}
\newcommand{\enc}{\mathsf{Enc}}
%\newcommand{\decc}{\mathcal{D}}
\newcommand{\decc}{\mathsf{Dec}}
\newcommand{\eval}{\mathsf{Eval}}
\newcommand{\genc}{\algfont{\Enc}}
\newcommand{\dec}{\mathsf{Dec}}
\newcommand{\gdec}{\algfont{\Dec}}
%\newcommand{\com}{Com}
\newcommand{\ver}{Ver}
\newcommand{\open}{\ver}
\def\Hash{H}

\newcommand{\Funcs}{\mathsf{Funcs}}

\newcommand{\ol}{\overline}
\newcommand{\wt}[1]{\widetilde{#1}}
\newcommand{\wh}[1]{\widehat{#1}}
\newcommand{\nin}{\not \in}

\newcommand{\es}{\varnothing} % empty set
\newcommand{\se}{\subseteq}
\newcommand{\sm}{\setminus}
\newcommand{\nse}{\not \se}
\newcommand{\nf}[2]{\nicefrac{#1}{#2}}
\newcommand{\rv}[1]{\mathbf{#1}} % Random Variable
\newcommand{\mal}[1]{\widehat{#1}} % Malicious version of the alg

\newcommand{\schemefont}[1]{{\mathsf{#1}}}
\newcommand{\primfont}[1]{{\mathsf{#1}}}
\newcommand{\algfont}[1]{{\mathsf{#1}}}
\newcommand{\advfont}[1]{{\mathsf{#1}}}
\newcommand{\orfont}[1]{\mathsc{#1}}
\newcommand{\varfont}[1]{\mathit{#1}}
\newcommand{\randvarfont}[1]{\mathbf{#1}}
\newcommand{\constfont}[1]{\mathtt{#1}}
\newcommand{\notionfont}[1]{\mathrm{#1}}
\newcommand{\eventfont}[1]{{\mathsc{#1}}}
\newcommand{\vectorfont}[1]{\mathslbf{#1}}
\newcommand{\transformfont}[1]{\mathsf{#1}}


\newcommand{\tuple}[1]{\langle{#1}\rangle}
\def\from{\mbox{from}\ }
\def\From{\mbox{From}\ }
\def\bits{\{0,1\}}
\def\cross{\times}
\newcommand{\xor}{{\oplus}}
\newcommand{\Colon}{{:\;\;}}
\newcommand{\mystrut}{\rule{0em}{15pt}}
\newcommand{\namestrut}{\rule{0em}{20pt}}
\def\poly{\mathop{\rm poly}\nolimits}
\def\emptystring{\varepsilon}
\def\emptyid{()}
\newcommand{\Dom}{\mathsf{Dom}}
\newcommand{\DomR}{\mathsf{DomR}}
\newcommand{\Rng}{\mathsf{Rng}}
\newcommand{\RngR}{\mathsf{RngR}}
\newcommand{\Keys}{\mathsf{Keys}}
\newcommand{\calA}{{\cal A}}
\newcommand{\calB}{{\cal B}}
\newcommand{\calC}{{\cal C}}
\newcommand{\calF}{{\cal F}}
\newcommand{\calL}{{\cal L}}
\newcommand{\calM}{{\cal M}}
\newcommand{\calO}{{\cal O}}
\newcommand{\calR}{{\cal R}}
\newcommand{\calH}{{\cal H}}
\newcommand{\calG}{{\cal G}}
\newcommand{\calD}{{\cal D}}
\newcommand{\calE}{{\cal E}}
\newcommand{\calP}{{\cal P}}
\newcommand{\calS}{{\cal S}}
\newcommand{\calU}{{\cal U}}
\newcommand{\calX}{{\cal X}}
\newcommand{\calY}{{\cal Y}}
\newcommand{\calEE}{{\cal EE}}
\newcommand{\N}{{{\mathbb N}}}
\newcommand{\Z}{{{\mathbb Z}}}
\newcommand{\R}{{{\mathbb R}}}
\newcommand{\goesto}{{\rightarrow}}
\newcommand{\eqdef}{\;\stackrel{\rm def}{=}\;}
\newcommand{\then}{{\;;\;\;}}   % for [ ; ; ; : ] notation
\newcommand{\andthen}{{\;:\;\;}}
\def\union{\cup}
\def\bigunion{\bigcup}
\def\intersection{\cap}
\def\bigintersection{\bigcap}
\def\suchthatt{\: :\:}
\newcommand{\suchthat}{{\mbox{s.t.\ }}}
\def\next{\:;\:}
\def\nextt{\:;\:}
\newcommand{\sett}[1]{\{#1\}}
\newcommand{\set}[2]{\{\:#1 \suchthatt #2\:\}}
\newcommand{\setsize}[1]{\left|{#1}\right|}
\def\leqq{\;\leq\;}
\def\eqq{\;=\;}
\def\geqq{\;\geq\;}
\def\equivv{\;\equiv\;}
\def\prn#1{\left(#1\right)}
\newcommand{\getsr}{{\:{\leftarrow{\hspace*{-3pt}\raisebox{.75pt}{$\scriptscriptstyle\$$}}}\:}}
% \def\getsr{\stackrel{{\scriptscriptstyle\$}}{\leftarrow}}
% \renewcommand{\choose}[2]{{{#1}\atopwithdelims(){#2}}}

\newcommand{\Var}{{\mbox{\bf Var}}}
\newcommand{\E}{\mathbf{E}}
\newcommand{\EE}[1]{{\E\left[{#1}\right]}}
\newcommand{\EEE}[2]{{\E_{#1}\left[{#2}\right]}}
\newcommand{\Prob}[1]{{\Pr\left[\,{#1}\,\right]}}
\newcommand{\probb}[2]{{\Pr}_{#1}\left[\,{#2}\,\right]}
\newcommand{\probbs}[3]{{\Pr}^{#1}_{#2}\left[\,{#3}\,\right]}
\newcommand{\probbp}[2]{{\Pr}_{#1}'\left[\,{#2}\,\right]}
\newcommand{\Probb}[2]{{{\Pr}_{#1}\left[\,{#2}\,\right]}}
\newcommand{\Probbp}[2]{{{\Pr}_{#1}'\left[\,{#2}\,\right]}}
\newcommand{\condProb}[2]{{\Pr}\left[\,#1\,|\,#2\,\right]}
\newcommand{\CondProb}[2]{{\Pr}\left[\: #1\:\left|\right.\:#2\:\right]}
\newcommand{\CondProbb}[3]{{\Pr}_{#1}\left[\: #2\:\left|\right.\:#3\:\right]}
\newcommand{\CondProbbp}[3]{%
{\Pr}_{#1}^{'}\left[\: #2\:\left|\right.\:#3\:\right]}


% ===================================================================
\newcommand{\kwfont}[1]{{\ensuremath{\mathrm{#1}}}}
\newcommand{\kwfunction}{{\kwfont{function}\ }}
\newcommand{\kwlabel}{{\kwfont{label}\ }}
\newcommand{\kwfor}{{\kwfont{for}\ }}
\newcommand{\kwand}{{\kwfont{and}\ }}
\newcommand{\kwor}{{\kwfont{or}\ }}
\newcommand{\kwnot}{{\kwfont{not}}}
\newcommand{\kwdo}{{\kwfont{do}\ }}
\newcommand{\kwreturn}{{\kwfont{return}\ }}
\newcommand{\kwReturn}{{\kwfont{Return}\ }}
\newcommand{\kwalgorithm}{{\ensuremath{\mathbf{Algorithm}\ }}}
\newcommand{\kwprotocol}{{\kwfont{Protocol}\ }}
\newcommand{\kwexperiment}{{\kwfont{Experiment}\ }}
\newcommand{\kwadversary}{{\kwfont{Adversary}\ }}
\newcommand{\kworacle}{{\kwfont{Oracle}\ }}
\newcommand{\kwuntil}{{\kwfont{until}\ }}
\newcommand{\kwrepeat}{{\kwfont{repeat}\ }}
\newcommand{\kwif}{{\kwfont{If}\ }}
\newcommand{\kwthen}{{\kwfont{Then}\ }}
\newcommand{\kwelse}{{\kwfont{Else}\ }}
\newcommand{\kwabort}{{\kwfont{abort}\ }}
\newcommand{\kwgoto}{{\kwfont{goto}\ }}
\newcommand{\kwwhile}{{\kwfont{while}\ }}
\newcommand{\kwparse}{{\kwfont{parse}\ }}
\newcommand{\kwas}{{\kwfont{as}\ }}
\newcommand{\kwstatic}{{\kwfont{static}\ }}
\newcommand{\kwrun}{{\kwfont{run}\ }}
\newcommand{\kwbegin}{{\kwfont{begin}\ }}
\newcommand{\kwend}{{\kwfont{end}\ }}
\newcommand{\kwstart}{{\kwfont{start}\ }}
\newcommand{\kwcontinue}{{\kwfont{continue}\ }}
\newcommand{\kwdefine}{{\kwfont{define}\ }}
\newcommand{\kwflip}{{\kwfont{flip}\ }}
\newcommand{\kwlet}{{\kwfont{let}\ }}
\newcommand{\kwof}{{\kwfont{of}\ }}
\newcommand{\kwcase}{{\kwfont{case}\ }}
\newcommand{\kwswitch}{{\kwfont{switch}\ }}
\newcommand{\kwpick}{{\kwfont{pick}\ }}
\newcommand{\kwset}{{\kwfont{set}\ }}
\newcommand{\kwcompute}{{\kwfont{compute}\ }}
\newcommand{\comment}[1]{\hspace{15pt}{\small /$\!\!$/\ #1}}
\newcommand{\Comment}[1]{\hspace{5pt}{/$\!\!$/\ #1}}
\newcommand{\sComment}[1]{\hspace{2pt}{/$\!\!$/#1}}

\newcommand{\Coins}{\mathsf{Coins}}

\newcommand{\ProbExp}[2]{{\Pr}\left[\: #1\:\suchthatt\:#2\:\right]}

\newcommand{\PRG}{\schemefont{PRG}}
\newcommand{\PRGpg}{\schalg{PRG}{Ev}}
\newcommand{\PRGkg}{\schalg{PRG}{Kg}}

\newcommand{\FSE}{\schemefont{FSE}}

\newcommand{\PE}{\schemefont{PE}}
\newcommand{\FE}{\schemefont{FE}}
\newcommand{\setup}{\mathsf{Setup}}
\newcommand{\keyder}{\mathsf{KeyDer}}

\newcommand{\OT}{\schemefont{OT}}
\newcommand{\Send}{\procfont{Sender}}
\newcommand{\Rec}{\procfont{Receiver}}
\newcommand{\PKE}{\schemefont{PKE}}
\newcommand{\FPKE}{\schemefont{FPKE}}
\newcommand{\DPKE}{\schemefont{DPKE}}
\newcommand{\DE}{\schemefont{DE}}
\newcommand{\PKEpg}{\schalg{PKE}{Pg}}
\newcommand{\oPKEpg}{\overline{{\cal P}}}
\newcommand{\PKEEnc}{\schalg{PKE}{Enc}}
\newcommand{\oPKEEnc}{\schalg{\overline{PKE}}{Enc}}
\newcommand{\PKEDec}{\schalg{PKE}{Dec}}
\newcommand{\oPKEDec}{\overline{{\cal D}}}
\newcommand{\PKEkg}{\schalg{PKE}{Kg}}
\newcommand{\oPKEkg}{\overline{{\cal K}}}
\newcommand{\PKEpk}{\mathit{ek}}
\newcommand{\PKEPK}{\schalg{PKE}{PK}}
\newcommand{\PKESK}{\schalg{PKE}{SK}}
\newcommand{\oPKEpk}{\overline{\mathit{ek}}}
\newcommand{\PKEsk}{\mathit{dk}}
\newcommand{\oPKEsk}{\overline{\mathit{dk}}}
\newcommand{\PKEpars}{\pi}
\newcommand{\oPKEpars}{\overline{\pi}}

\newcommand{\SE}{\schemefont{SE}}
\newcommand{\SEpg}{{\cal P}}
\newcommand{\SEEnc}{{\cal E}}
\newcommand{\SEDec}{{\cal D}}
\newcommand{\SEkg}{{\cal K}}
\newcommand{\SEkey}{K}
\newcommand{\SEpars}{\pi}

%%%  English %%%%%%%%%%%%%%%%%%%%%%%%%%%%%%%%%%%%%%%%%%%%%%%%%%%%%%

\newcommand{\etal}{et~al.\ }
\newcommand{\aka}{also known as,\ }
\newcommand{\resp}{resp.,\ }
\newcommand{\ie}{i.e.,\ }
\newcommand{\wolog}{w.l.o.g.\ }
\newcommand{\Wolog}{W.l.o.g.\ }
\newcommand{\eg}{e.g.,\ }
\newcommand{\Eg}{E.g.,\ }
\newcommand{\wrt} {with respect to\ }
\newcommand{\cf}{{cf.,\ }}

%%%  math %%%%%%%%%%%%%%%%%%%%%%%%%%%%%%%%%%%%%%%%%%%%%%%%%%%%%%

\newcommand{\round}[1]{\lfloor #1 \rceil}
\newcommand{\ceil}[1]{\lceil #1 \rceil}
\newcommand{\floor}[1]{\lfloor #1 \rfloor}
\newcommand{\angles}[1]{\langle #1 \rangle}
\newcommand{\parens}[1]{( #1 )}
\newcommand{\bracks}[1]{[ #1 ]}
\newcommand{\bra}[1]{\langle#1\rvert}
\newcommand{\ket}[1]{\lvert#1\rangle}


\newcommand{\adjRound}[1]{\left\lfloor #1 \right\rceil} % Adjusted Round
\newcommand{\adjCeil}[1]{\left\lceil #1 \right\rceil}
\newcommand{\adjFloor}[1]{\left\lfloor #1 \right\rfloor}
\newcommand{\adjAngles}[1]{\left\langle #1 \right\rangle}
\newcommand{\adjParens}[1]{\left( #1 \right)}
\newcommand{\adjBracks}[1]{\left[ #1 \right]}
\newcommand{\adjBra}[1]{\left\langle#1\right\rvert}
\newcommand{\adjKet}[1]{\left\lvert#1\right\rangle}
\newcommand{\adjSet}[1]{\left\{ #1 \right\}}
\newcommand{\half}{\tfrac{1}{2}}
\newcommand{\third}{\tfrac{1}{3}}
\newcommand{\quarter}{\tfrac{1}{4}}
%\newcommand{\eqdef}{:=}
%\newcommand{\zo}{\{0,1\}}




\newcommand{\cA}{{\mathcal A}}
\newcommand{\cB}{{\mathcal B}}
\newcommand{\cC}{{\mathcal C}}
\newcommand{\cD}{{\mathcal D}}
\newcommand{\cE}{{\mathcal E}}
\newcommand{\cF}{{\mathcal F}}
\newcommand{\cG}{{\mathcal G}}
\newcommand{\cH}{{\mathcal H}}
\newcommand{\cI}{{\mathcal I}}
\newcommand{\cJ}{{\mathcal J}}
\newcommand{\cK}{{\mathcal K}}
\newcommand{\cL}{{\mathcal L}}
\newcommand{\cM}{{\mathcal M}}
\newcommand{\cN}{{\mathcal N}}
\newcommand{\cO}{{\mathcal O}}
\newcommand{\cP}{{\mathcal P}}
\newcommand{\cQ}{{\mathcal Q}}
\newcommand{\cR}{{\mathcal R}}
\newcommand{\cS}{{\mathcal S}}
\newcommand{\cT}{{\mathcal T}}
\newcommand{\cU}{{\mathcal U}}
\newcommand{\cV}{{\mathcal V}}
\newcommand{\cW}{{\mathcal W}}
\newcommand{\cX}{{\mathcal X}}
\newcommand{\cY}{{\mathcal Y}}
\newcommand{\cZ}{{\mathcal Z}}

\newcommand{\bfA}{\mathbf{A}}
\newcommand{\bfB}{\mathbf{B}}
\newcommand{\bfC}{\mathbf{C}}
\newcommand{\bfD}{\mathbf{D}}
\newcommand{\bfE}{\mathbf{E}}
\newcommand{\bfF}{\mathbf{F}}
\newcommand{\bfG}{\mathbf{G}}
\newcommand{\bfH}{\mathbf{H}}
\newcommand{\bfI}{\mathbf{I}}
\newcommand{\bfJ}{\mathbf{J}}
\newcommand{\bfK}{\mathbf{K}}
\newcommand{\bfL}{\mathbf{L}}
\newcommand{\bfM}{\mathbf{M}}
\newcommand{\bfN}{\mathbf{N}}
\newcommand{\bfO}{\mathbf{O}}
\newcommand{\bfP}{\mathbf{P}}
\newcommand{\bfQ}{\mathbf{Q}}
\newcommand{\bfR}{\mathbf{R}}
\newcommand{\bfS}{\mathbf{S}}
\newcommand{\bfT}{\mathbf{T}}
\newcommand{\bfU}{\mathbf{U}}
\newcommand{\bfV}{\mathbf{V}}
\newcommand{\bfW}{\mathbf{W}}
\newcommand{\bfX}{\mathbf{X}}
\newcommand{\bfY}{\mathbf{Y}}
\newcommand{\bfZ}{\mathbf{Z}}


\newcommand{\bfa}{\mathbf{a}}
\newcommand{\bfb}{\mathbf{b}}
\newcommand{\bfc}{\mathbf{c}}
\newcommand{\bfd}{\mathbf{d}}
\newcommand{\bfe}{\mathbf{e}}
\newcommand{\bff}{\mathbf{f}}
\newcommand{\bfg}{\mathbf{g}}
\newcommand{\bfh}{\mathbf{h}}
\newcommand{\bfi}{\mathbf{i}}
\newcommand{\bfj}{\mathbf{j}}
\newcommand{\bfk}{\mathbf{k}}
\newcommand{\bfl}{\mathbf{l}}
\newcommand{\bfm}{\mathbf{m}}
\newcommand{\bfn}{\mathbf{n}}
\newcommand{\bfo}{\mathbf{o}}
\newcommand{\bfp}{\mathbf{p}}
\newcommand{\bfq}{\mathbf{q}}
\newcommand{\bfr}{\mathbf{r}}
\newcommand{\bfs}{\mathbf{s}}
\newcommand{\bft}{\mathbf{t}}
\newcommand{\bfu}{\mathbf{u}}
\newcommand{\bfv}{\mathbf{v}}
\newcommand{\bfw}{\mathbf{w}}
\newcommand{\bfx}{\mathbf{x}}
\newcommand{\bfy}{\mathbf{y}}
\newcommand{\bfz}{\mathbf{z}}



\newcommand{\sfA}{\mathsf{A}}
\newcommand{\sfB}{\mathsf{B}}
\newcommand{\sfC}{\mathsf{C}}
\newcommand{\sfD}{\mathsf{D}}
\newcommand{\sfE}{\mathsf{E}}
\newcommand{\sfF}{\mathsf{F}}
\newcommand{\sfG}{\mathsf{G}}
\newcommand{\sfH}{\mathsf{H}}
\newcommand{\sfI}{\mathsf{I}}
\newcommand{\sfJ}{\mathsf{J}}
\newcommand{\sfK}{\mathsf{K}}
\newcommand{\sfL}{\mathsf{L}}
\newcommand{\sfM}{\mathsf{M}}
\newcommand{\sfN}{\mathsf{N}}
\newcommand{\sfO}{\mathsf{O}}
\newcommand{\sfP}{\mathsf{P}}
\newcommand{\sfQ}{\mathsf{Q}}
\newcommand{\sfR}{\mathsf{R}}
\newcommand{\sfS}{\mathsf{S}}
\newcommand{\sfT}{\mathsf{T}}
\newcommand{\sfU}{\mathsf{U}}
\newcommand{\sfV}{\mathsf{V}}
\newcommand{\sfW}{\mathsf{W}}
\newcommand{\sfX}{\mathsf{X}}
\newcommand{\sfY}{\mathsf{Y}}
\newcommand{\sfZ}{\mathsf{Z}}

\newcommand{\sfa}{\mathsf{a}}
\newcommand{\sfb}{\mathsf{b}}
\newcommand{\sfc}{\mathsf{c}}
\newcommand{\sfd}{\mathsf{d}}
\newcommand{\sfe}{\mathsf{e}}
\newcommand{\sff}{\mathsf{f}}
\newcommand{\sfg}{\mathsf{g}}
\newcommand{\sfh}{\mathsf{h}}
\newcommand{\sfi}{\mathsf{i}}
\newcommand{\sfj}{\mathsf{j}}
\newcommand{\sfk}{\mathsf{k}}
\newcommand{\sfl}{\mathsf{l}}
\newcommand{\sfm}{\mathsf{m}}
\newcommand{\sfn}{\mathsf{n}}
\newcommand{\sfo}{\mathsf{o}}
\newcommand{\sfp}{\mathsf{p}}
\newcommand{\sfq}{\mathsf{q}}
\newcommand{\sfr}{\mathsf{r}}
\newcommand{\sfs}{\mathsf{s}}
\newcommand{\sft}{\mathsf{t}}
\newcommand{\sfu}{\mathsf{u}}
\newcommand{\sfv}{\mathsf{v}}
\newcommand{\sfw}{\mathsf{w}}
\newcommand{\sfx}{\mathsf{x}}
\newcommand{\sfy}{\mathsf{y}}
\newcommand{\sfz}{\mathsf{z}}

\newcommand{\eps}{\epsilon}
%\newcommand{\e}{\epsilon}
\newcommand{\veps}{\varepsilon}
%\newcommand{\vare}{\varepsilon}
\newcommand{\vphi}{\varphi}
\newcommand{\vsigma}{\varsigma}
\newcommand{\vrho}{\varrho}
\newcommand{\vpi}{\varpi}
\newcommand{\tO}{\widetilde{O}}
\newcommand{\tOmega}{\widetilde{\Omega}}
\newcommand{\tTheta}{\widetilde{\Theta}}
\newcommand{\field}{\ensuremath \Bbbk}
\newcommand{\reals}{\ensuremath \mathbb{R}}

\newcommand{\client}{\mathsl{client}}
\newcommand{\server}{\mathsl{server}}
\newcommand{\TTP}{\procfont{TTP}}
\newcommand{\rec}{\mathsf{rec}}
\newcommand{\doc}{\mathsf{doc}}
\newcommand{\pred}{\mathsf{pred}}
\newcommand{\scrub}{\mathsf{scrub}}

\newcommand{\ODS}{\schemefont{ODS}}
\newcommand{\commit}{\mathsf{Commit}}
\newcommand{\query}{\mathsf{Query}}
\newcommand{\DB}{\mathsf{DB}}


\newcommand{\sanjam}[1]{\textcolor{red}{Sanjam: #1}}
\newcommand{\moh}[1]{}%{\textcolor{red}{Moh: #1}}
\newcommand{\adam}[1]{\textcolor{red}{Adam: #1}}
\newcommand{\ben}[1]{\textcolor{red}{Ben: #1}}

%---
\newtheorem{definition}{Definition}[section]
\newtheorem{question}{Question}[section]
\newtheorem{fact}{Fact}[section]
\newtheorem{lemma}{Lemma}[section]
\newtheorem{corollary}{Corollary}[section]
\newtheorem{theorem}{Theorem}[section]
\newtheorem{claim}{Claim}[section]
\newtheorem{assumption}{Assumption}[section]
\newtheorem{proposition}{Proposition}[section]
\newtheorem{hypothesis}{Hypothesis}[section]
\newtheorem{observation}{Observation}[section]
\theoremstyle{remark}
\newtheorem{construction}{Construction}[section]
\newtheorem{remark}{Remark}[section]
\newtheorem{example}{Example}[section]

\newcommand{\pone}{\mbox{$P_1$}}
\newcommand{\ptwo}{\mbox{$P_2$}}

\renewcommand{\S}{{\cal S}}
\newcommand{\F}{{\cal F}}
\newcommand{\G}{{\mathbb{G}}}

\newcommand{\vecx}{{\mathbf{x}}}
\newcommand{\vecy}{{\mathbf{y}}}

\newcommand{\numcorr}{{\mathsf{ncorr}}}
\newcommand{\numencq}{{\mathsf{nencq}}}
\newcommand{\numkeyq}{{\mathsf{nkeyq}}}


\renewenvironment{proof}{\noindent{\bf Proof:~~}}{\qed}
\newcommand{\BPF}{\begin{proof}} \newcommand {\EPF}{\end{proof}}
\newenvironment{proofsketch}{\noindent{\bf Proof Sketch:~~}}{\qed}
\newcommand{\BPFS}{\begin{proofsketch}} \newcommand {\EPFS}{\end{proofsketch}}

%\def\qed{\quad\blackslug\lower 8.5pt\null\par}


\newcommand{\BPR}{\begin{myprotocol}}   \newcommand{\EPR}{\end{myprotocol}}
\newcommand{\ourfigg}[5]{
{\begin{figure}[#4]
\begin{center}
\framebox[\width][c]{
    \small
    \hbox{\quad
    \begin{varwidth}[c]{0.9\textwidth}
    %\begin{center}
    \begin{myfigure}
    [#1]
    \label{#2}
    \end{myfigure}
    %\end{center}
    \vspace{-3ex}
    #5
    \end{varwidth}
    \quad}
    }
    \begin{center} #3 \end{center}
    \vspace{-6ex}
\end{center}
\end{figure}
} }


% USAGE: \ourfig{TITLE}{LABEL}{CAPTION}{BODY}
\newcommand{\ourfig}[4]{\ourfigg{#1}{#2}{#3}{htb}{#4}}

\newcommand{\prott}[5]{
{\begin{figure}[#4]
\begin{center}
\framebox[\width][c]{
    \small
    \hbox{\quad
    \begin{varwidth}[c]{0.9\textwidth}
    %\begin{center}
    \begin{myprotocol}
    [#1]
    \label{#2}
    \end{myprotocol}
    %\end{center}
    \vspace{-3ex}
    #5
    \end{varwidth}
    \quad}
    }
    \begin{center} #3 \end{center}
    \vspace{-6ex}
\end{center}
\end{figure}
} }

% USAGE: \prot{TITLE}{LABEL}{CAPTION}{BODY}
\newcommand{\prot}[4]{\prott{#1}{#2}{#3}{htb}{#4}}

\newcommand{\view}{{\sf view}}
\newcommand{\trans}{{\sf trans}}
\newcommand{\com}{Z}
\newcommand{\cecom}{{\sf CECom}}
\newcommand{\mxcom}{{\sf MXCom}}
\newcommand{\mxzk}{{\sf MXZK}}
\newcommand{\nmmxcom}{{\sf NMMXCom}}
\newcommand{\gen}{{\sf Gen}}


\newcommand{\crsgen}{{\sf CRSGen}}
\newcommand{\crs}{{\sf CRS}}


\newcommand{\Commit}{{\sf Com}}


\newcommand{\scheme} {{\mathcal{S}}}

\newcommand{\siggen} {{\sf SigKeyGen}}
\newcommand{\sig} {{\sf Sig}}

%\newcommand{\size} {{\sf SIZE}}
\newcommand{\state} {\mathrm{state}}

\newcommand{\idx} {\mathrm{index}}

\newcommand{\kdm}{{\scriptscriptstyle\mathrm{KDM}}}
\newcommand{\skdm}{{\scriptscriptstyle\mathrm{SKDM}}}
\newcommand{\cpa}{{\scriptscriptstyle\mathrm{CPA}}}
\newcommand{\new}{{\scriptscriptstyle\mathrm{NEW}}}


\newcommand{\rsetup}[1]{\R({\sf setup}, #1 ) }
\newcommand{\rquery}[1]{\R({\sf query}, #1 ) }
\newcommand{\rchall}[1]{\R({\sf challenge}, #1 ) }
\newcommand{\rfinal}[1]{\R({\sf final}, #1 ) }
\newcommand{\iO}{i\mathcal{O}}

\newcommand{\PRF}{{\mathsf{F}}}
\newcommand{\PRFGen}{\mathsf{PRF{.}Gen}}
\newcommand{\PRFPunc}{\mathsf{PRF{.}Punc}}
\newcommand{\seed}{\ensuremath{{K}}}
\newcommand{\secpar}{\secparam}
\newcommand{\Adv}{\mathcal{A}}
\newcommand{\rsample}{\gets}

\newcommand{\tf}{\mathrm{tf}}
\newcommand{\idf}{\mathrm{idf}}
\newcommand{\df}{\mathrm{df}}
\newcommand{\p}{\mathrm{P}}



\date{}
\begin{document}

\maketitle
%\vspace{-20mm}
%\begin{abstract}
%ent results have resolved functional encryption for all circuits. Does this solve most problems for functional encryption. We think not. In this paper we present the future vision for functional encryption. Our vision for functional encryption is to realize encryption systems which (at least asymptotically) approach the efficiency levels approached in insecure solutions.
%\end{abstract}
\vspace{-20mm}

\section{Introduction}

\subsection{Background and Motivation} 

In modern society, data is often stored at rest for later use.  Databases are pervasive and the task of organizing and maintaining data is often off-loaded to an external provider. However, this means the external provider must be trusted with confidentiality of the data.  To mitigate this, an attractive solution is for a client to encrypt its data, under a key only it knows, before sending it to the external provider.  However, standard encryption hides all partial information about the data and does not allow efficient query processing.  Recent work in the cryptographic community has tried to address this problem by providing specialized encryption schemes and other secure database outsourcing protocols that both provide confidentiality to the extent possible and allow efficient query processing.  \textbf{However, this work has so far been mainly limited to very basic SQL functionality and key-value stores, which are outdated}.  The goal of this proposal is to bring us beyond this basic functionality to ``next-generation'' functionality that practitioners expect.  
\ben{This isn't next gen, isn't current but that doesn't sound nearly as good}

\subsection{This Proposal} \label{subsec-prop}

The goal of this proposal is create new secure outsourced database protocols in line with the real-world functionality practitioners expect.  Real-world databases often store spatial, time-series, or graph data~\cite{SP:FVYSHG17}.  We focus on extending two key approaches to secure outsourced databases to these cases: Protocols based on \emph{property-preserving encryption} like CryptDB, and protocols based on \emph{secure multiparty computation} like BlindSeer.  Finally, we plan to extend recent leakage-abuse attacks to our new protocols and propose novel mitigations.

\paragraph*{Proposed contributions in brief.}
In sum, the main proposed contributions are:
\begin{itemize}
\item Design advanced types of property-preserving and partially-property preserving encryption schemes for use on spacial, time series, and graph data.
\item Design extensions of BlindSeer to handle such data.
\item Extend recent attacks on databases that use property-preserving encryption and propose novel mitigations of these attacks. 
\end{itemize}

\paragraph*{Intellectual merit and research team.}  

 \paragraph*{Broader Impacts.}

\subsection{Related Work}

We do not address approaches based on searchable symmetric encryption or functional encryption.  

%!TEX root = dbproposal.tex

\section{Property-Preserving Encryption for Proximity}
\label{sec:ppe}

In spatial databases, nearest neighbor queries (\emph{e.g.}, finding the closest soldier in the field) and clustering queries are pervasive.  %Algorithms for executing these query types need to understand distance relationships between tuples o perform the fundamental operation of \emph{distance comparison} between points in the database, \emph{i.e.}, determining which of two candidates points are closer to a target point.  
In large scale biometric databases, the fundamental operation is
comparison of the distance between a target point and all stored points.
Often, biometric databases return the nearest match if the distance is
less than some threshold.
% these examples are already in the intro
% For example, the Aadhaar system in India
% links biometrics with a unique 12 digit number with over 1 billion
% numbers issued~\cite{daugman2014600}.  Lastly, many machine learning
% algorithms depend on computing distance between points. Linear
% regression finds the line that minimizes the sum of distances between
% the line and each data point.  Similarly, the first principal component
% minimizes the sum of distances between the selected line and the
% dataset~\cite{wold1987principal}.  The remaining principal components
% are defined similarly subject to being orthogonal to previously defined
% principal components.
In this component of the proposal, we will consider encryption
algorithms which support various types of proximity queries.

\paragraph{Core operations.}
Despite the varied use of proximity functionality in databases, we find two core operations.     The first core operation is computing the distance between two points $a$ and $b$ or $d(a,b)$ according to some metric $d$.  This distance can be granular, for example $d(a, b) =15$ or coarse, $d(a,b)\in\{0,1,2\}$ corresponding to $\{equal, close, far\}$.  Even with this small set of outputs, users expect the function $d$ to behave as a metric so we restrict our attention to metric functions.  We call property-preserving encryption where $d(\enc(a), \enc(b)) = d(a,b)$ \emph{distance preserving encryption} or DPE.  Using the same language as for order-revealing and order-preserving encryption, we define a variant called \emph{distance-revealing encryption} or DRE where distance is computable  via some metric on the output space.

The second core operation takes a triple of points $a,b$ and $c$ and outputs the bit whether $d(a,c)<d(b,c)$.  This type of computation is frequently used in clustering, nearest neighbor, and learning algorithms.   We call a property-preserving encryption that achieves this functionality \emph{distance-comparison revealing encryption} or DCRE. When the distance comparison on ciphertexts uses the \emph{same} metric as on the plaintext we call \emph{distance-comparison preserving encryption} or DCPE.  The goal for this technique is to \emph{not} allow computation of $d(a,b)$.
 A related notion is \emph{approximate} distance-preserving encryption, where  $\delta_1 \cdot d(a,b) \leq d(\enc(a), \enc(b)) \leq \delta_2  \cdot d(a,b)$ for parameters $\delta_1, \delta_2$.  This can be used much like distance-preserving encryption in algorithms to give approximate guarantees.

\paragraph{Goals.}
The security achievable by DPE and DCPE is unclear even for the ideal objects.
Our research on encryption for proximity  consists of three tasks:
\begin{itemize}\setlength{\itemsep}{0em}
\item Understanding security of DPE and constructing DPE and DRE
schemes. (We expect the ideal security to vary widely based on the underlying metric
properties.)
\item Understanding the security of DCPE, and constructing DCPE and DCRE
schemes.
\item Exploring the connection to approximate distance-revealing
encryption. (We believe that approximate distance-revealing encryption
may be equivalent to distance-comparison preserving encryption.)
\end{itemize}

\subsection{DPE and DRE}

\paragraph{Prior work.}
Generalizing the work of and Li
et al.~\cite{li2010fuzzy,wang2013efficient},
Boldyreva and Chenette~\cite{boldyreva2014efficient}
defined a
\emph{closeness-preserving tagging function} where $\enc(a)$ and
$\enc(b)$ have associated tag sets.  If the tag sets intersect, this
implies that $d(a,b) \in\{equal, close\}$ with high probability;
otherwise the $d(a,b)$ is deemed to be $far$.
%This work generalizes the work of who construct a tagging function for particular string functions.  
%
The construction creates a tag for every possible neighbor and thus {\em
does not scale} to metrics where many values are considered close.
Boldyreva and Chenette also present a negative result: there is a
closeness function which \emph{requires} storage
proportional to the number of close neighbors~\cite[Theorem
5.2]{boldyreva2014efficient}.  This result is not known to hold if the
closeness function is a (well-known) metric.  

\iffalse
The approach can be summarized as follows:
\begin{enumerate}\setlength\itemsep{0em}
\item The clients appends to $\enc(a)$ a deterministic encryption of all $b$ such that $d(a,b) \le 1$.
\item To search, the client deterministically encrypts their search term $b$.
\item The server returns all ciphertexts where some encryption matches.\footnote{Some  modifications are necessary to meet ideal security such as hiding the number of neighbors.  See the prior work of Boldyreva and Chenette~\cite{boldyreva2014efficient}.}  
\end{enumerate}
\fi

Alternatively, the plaintext $a$ can be encrypted (without tags) and the client can search for the disjunction of all neighbors of $b$ (rather than just $b$).  Woodage et al.~apply this approach in the context of password authentication with typos~\cite{C:WCDJR17}.  Both of these approaches {\em require time linear in the number of neighbors} and are not viable for high dimensional data.

Lastly, the co-PI~\cite{EPRINT:ABCFG16} constructed an object called a
pseudoentropic isometric which is variant of distance-preserving
encryption.  This definition allows that
$d(\enc(a), \enc(b)) <d(a,b)$. It is possible to modify the
construction of \cite{EPRINT:ABCFG16} so that $d(\enc(a), \enc(b)) =
d(a,b)$, but this construction {\em only works for the set-difference
metric over large alphabets and requires strong cryptographic assumptions}.

\subsubsection{Proposed Work}

\paragraph{Understanding and constructing distance-preserving encryption.}
There is a scattering of work on variants of distance-revealing
encryption, but there is not a {\em solid theoretic foundation} for the
object.  Previous work defines noisy searchable encryption schemes but
never explicitly defines distance-revealing encryption.   Given the history of order-preserving encryption it is crucial to understand the security guarantees of distance-preserving and distance-revealing encryption. % In particular, we believe that distance-preserving encryption does not offer meaningful security.

%Ongoing work by the PIs indicates the viability of distance-preserving encryptionthis structure is highly metric dependent.  
The PIs first propose to study distance-revealing encryption for the Hamming metric space.
Consider the Hamming metric over $\mathcal{M}^\ell$ defined as
the number of positions in which two length-$\ell$ strings differ.
%$d(a,b) = | \{ 1\le i\le \ell | a_i \neq b_i\}|$.
Ongoing work by the PIs indicates that there is no secure
distance-preserving encryption for this metric unless the size of the
character set $|\mathcal{M}|$ is
super-polynomial.

% Define the following: 
% \begin{itemize}\setlength\itemsep{0em}
% \item Let $f: \mathcal{M}^\ell \rightarrow \mathcal{M}^\ell$ permute its input coordinates, and let $g_i: \mathcal{M}\rightarrow \mathcal{M}$ be a permutation.
% %\item Let $x\in\mathcal{M}^\ell$ be a plaintext points.  
% \end{itemize}
All distance-preserving encryption schemes
% $\enc: \mathcal{M}^\ell\rightarrow \mathcal{M}^\ell$
are of the form $\enc(x) =
f(g_1(x_1),..., g_\ell(x_\ell))$, where $f$ is a permutation of the
input coordinates and each $g_i$ is a permutation of $\mathcal{M}$.
% The permutation $f$ rotates the dimensions and $g_i$ shuffles individual dimensions.
These are the only
operations that preserve distance across the entire metric space, which
limits the potential security unless $\mathcal{M}$ is very large.

Whenever $|\mathcal{M}|$ is polynomial in the security parameter, the adversary can completely learn $\enc(x)$ with a polynomial number of plaintext/ciphertext pairs (linear for binary strings).  Prior work of the PI only provided security when $\mathcal{M}$ was superpolynomial in size~\cite{EPRINT:ABCFG16}, which our ongoing work shows is necessary.  However, the PIs also plan to investigate security of distance-preserving encryption for the Hamming metric when the adversary does not know any plaintext-ciphertext pairs (i.e., a so-called ciphertext-only attack) or a very small number of them, as was previously done for order-preserving encryption~\cite{C:BolCheONe11}.
 
 The PIs propose to study other metrics, in particular Euclidean,
 cosine-similarity, and Jaccard distance.    The PIs also plan to
 address the case of \emph{course grained} distance, e.g., where
 distance is in the set $\{far, equal, close\}$.  This problem is
 connected to property-preserving encryption for \emph{graph data}.  A
 course metric can be represented by a graph where neighboring vertices
 are near according to the metric.  Intuitively, our goal is to encrypt
 a graph as another graph with a subgraph of the same structure,
 which could lead to interesting questions in graph theory.

\paragraph{Understanding ideal distance-revealing encryption.}
In the case of \emph{ideal} distance-revealing encryption, a foundational question is whether it can be constructed from multilinear or bilinear maps.  There also may be interesting ``intermediate'' leakage profiles that are not ideal but leak less information than distance preserving encryption.   The PI has recently used this approach to find positive results for order-revealing encryption~\cite{EPRINT:CLOZ16}.

\paragraph{Distance-revealing encryption based on fuzzy extractors.}
We will construct distance-revealing encryption based on fuzzy
extractors which are a well known primitive for deriving a stable key
from a noisy source~\cite{EC:DodReySmi04}.  They consist of a pair of
algorithms: $\gen(a)$, which produces a stable key $key$ and a public
value $p$, and $\rep(b, p)$, which outputs $key$ if $d(a,b)\in\{equal,close\}$.  In ongoing work, we are constructing noisy searchable encryption from fuzzy extractors and distance-preserving encryption.  Let $\{a_i | 1\le i \le n\}$ be the set of plaintexts to be stored in the database.  Let $\enc$ be a distance-preserving encryption and let $H$ be a hash function.  Rather than directly storing $\enc(a_i)$ the client does the following:

\begin{itemize}\setlength\itemsep{0em}
\item Compute $c_i \leftarrow \enc(a_i)$ and then $k_i, p_i \leftarrow \gen(c_i)$. Send $p_i, H(k_i)$ to server.
\end{itemize}

To search for points close to $b$, the client encrypts $\enc(b)$ and presents this to the server which can rerun the fuzzy extractor for all stored terms.  The server can then return these terms to the client.  This approach has considerably less leakage than distance-preserving encryption as the server can only ``compare'' ciphertexts that are used in search.  The stored ciphertexts are not ``useful.''  We will extend this concept to remove the asymmetry between query and stored ciphertexts, transforming the construction into distance-revealing encryption.
We expect the output of this component to be 1) evidence on which metrics are suitable for DPE  and 2) constructions and analysis of DRE including constructions that use DPE.  

\subsection{Distance Comparison Revealing Encryption}
We call encryption that supports distance comparison \emph{distance-comparison revealing} encryption (DCRE).  As described above, many learning algorithms do not require direct computation of $d(a,b)$.  Rather it suffices to indicate which of two points is closer to a target point.  That is, it is sufficient to compute $d(a,c)\overset{?}<d(b,c)$.  Following the literature on order-revealing vs.~order-preserving encryption, we call the special case where ciphertexts themselves are spatial points \emph{distance-comparison preserving encryption} (DCPE).  The hope is that the weaker functionality of DCRE and DCPE can be secure for more metrics than distance-revealing and distance-preserving encryption respectively.
This leads to the following questions:

\begin{enumerate}\setlength{\itemsep}{0ex}
\item Can we design efficient DCPE?  What security can be achieved by such schemes?
\item Can we design efficient DCRE with better security?
\end{enumerate}

To answer the first question, in ongoing work we have found that
distance comparison preserving functions do not seem to have a nice
``recursive'' property as in the case of order-preserving functions,
which was crucially exploited by~\cite{EC:BCLO09}.  However, based on
computer experiments, we conjecture that \emph{distance-comparison
preserving functions are approximately distance-preserving}.    So far,
we have proven this conjecture in one dimension.  The intuition is that
as the number of points in the metric spaces increases, the number of degrees of freedom decreases.  The intuition is as follows:
\begin{enumerate}\setlength\itemsep{0em}
\item Suppose that $d(b,c) = k$ is known by the attacker, 
\item Learning that $d(a,c) < d(b,c)$ tells the attacker that $d(a,c)<k$.  
\item Suppose the attacker also knows that $d(f, c) = k/2$ and that $d(a,c) > d(f,c)$.  
\item The attacker can determine that $k/2< d(a,c) <k$.  
\item As more of these constraints are added, $d(a,c)$ is limited to smaller ranges.
\end{enumerate}
That is, the encryption mechanism reveals more accurate estimations on  the distance between $d(a,c)$.
If this conjecture is true, then for the first question we could equivalently turn our attention to the design and analysis of  an \emph{approximately distance-preserving} encryption scheme explored next.  That is, we expect a major outcome of this component to be 1) an understanding of the relationship between distance-comparison preserving encryption and approximate distance-preserving encryption (which we believe are equivalent), and 2) distance-comparison \emph{revealing} schemes (at least based on bilinear maps, hopefully even PRFs).

\subsection{Approximate Distance Revealing Encryption}
\label{sec:adre}
\paragraph{Prior work.}
The recent GRECS work allows minimum distance between any pair of
points~\cite{CCS:MKNK15}.  Rather than storing or computing distance
between pairs of points $a,b$, the work uses a primitive called a
sketch-based oracle.\footnote{This sketch-based oracle is different from
the notion of a secure sketch used in fuzzy extractor constructions.}  A logarithmic number of reference points $r_1,..., r_k$ are selected and the distance between $r_i$ and every point in the graph is computed and encrypted.  This structure is transmitted to the server.  Then at query time the client encrypts their pair $a,b$.  The server finds all reference points which are connected to $a,b$ and returns $\min_{r_i, r_j} d(a,r_i) + d(r_i, r_j) + d(r_j, b)$.  Given a careful selection of the reference points this distance can be shown to approximate the minimum distance between $d(a,b)$.

A second line of work uses locality-sensitive hashes which are designed to allow more efficient computation of nearest neighbor in high dimensional spaces~\cite{datar2004locality,slaney2008locality}.  A locality sensitive hash is function that is more likely to have collisions when two inputs are ``close'' in the input space.
% \begin{definition}
%   Let $(R,d)$ be a metric space.  A family of functions $\mathcal{H}$ is a $(r_1, r_2, p_1, p_2)$ locality sensitive hash (where $r_1< r_2$ and $p_1 >p_2$) if for any $x, y$ the following hold:
% \begin{itemize}\setlength\itemsep{0em}
% \item If $d(x, y) \le r_1$ then $\Pr_{h\leftarrow \mathcal{H}}[h(x) = h(y)] \ge p_1$, and 
% \item If $d(x, y) \ge r_2$ then $\Pr_{h\leftarrow \mathcal{H}}[h(x) = h(y)] \le p_2$.
% \end{itemize}
% \end{definition}

Kuzu, Islam and Kantarcioglu~\cite{kuzu2012efficient} use locality sensitive hashing to create an approximate distance-preserving data structure.  To create the index, for plaintext $a$ the client samples multiple locality sensitive hashes $h_1(a), h_2(a),...,h_k(a)$ and associates each output as a keyword with $a$ using deterministic encryption.
Then to query for $b$ the client queries computes $h_1(b),...,h_k(b)$. They then ask the server for all results that match
a single hash.% (this requires the ability to perform disjunctive queries).%
\footnote{To achieve this, they build an inverted index for each
locality sensitive hash that allows them to retrieve the document
identifiers.}  The client locally retrieves results and then restricts
to those documents that have a high number of hash matches.  With good
probability, these will correspond to those records that were close to
the original query.  Bringer et
al.~\cite{bringer2011identification,bringer2009error} use similar
techniques but insert the output of the locality sensitive hash into a
Bloom filter.  Both of these works provide relatively weak security and
make no attempt to hide records that match a small number of hash
function outputs.  

\paragraph{Proposed work.}
Distance-preserving functions are easy to characterize geometrically, in terms of a scaling factor plus flips, rotations and reflections.  To approximately preserve distance, we can also  ``perturb'' each image point within a ball of given radius.  
We can show that independent random such perturbations yields a function that, while not strictly DCPE, is \emph{approximately} so, and that encrypting via an approximately DCPE function still guarantees accuracy of nearest neighbor  search within the approximation.  Moreover, as such perturbations can easily be derandomized, this gives an efficient \emph{approximate} DCPE scheme from PRFs.  Finally, we will conduct a separate analysis in the spirit of~\cite{C:BolCheONe11} to answer the question of what privacy such a scheme provides.  

 A related question we plan to investigate is the privacy achievable using locality-sensitive hashing to perform nearest neighbor algorithms.  To our knowledge, prior work has not explicitly defined privacy properties for locality-sensitive hashing.  Moreover, the scheme of Kuzu et al.~\cite{kuzu2012efficient}  does not approach ideal security as the server learns which records match each subfeature.  In ongoing work, we are combining locality-sensitive hashing with the recent \emph{sample-then-hash} fuzzy extractor construction of the co-PI~\cite{EC:CFPRS16}.  The idea is as follows:

\begin{enumerate}\setlength\itemsep{0em}
\item For input $a$, compute $a_1 = h_1(a), a_2 = h_2(a),..., a_k = h_k(a)$.
\item For $i=1,...., \ell$,
sample $i_1,..., i_\eta\overset{\$}\leftarrow [k]$, compute $\alpha_i =
H(a_{i_1} || ... || a_{i_\eta})$, and append $\alpha_i$ as keyword to $a$
(using deterministic encryption).
\end{enumerate}
\noindent
This approach appends multiple locality sensitive hashes that individually have a small probability of matching but overall there is a high probability of a single match.  The probability of finding any matches between $a,b$ that are not close is small (see Canetti et al.~\cite{EC:CFPRS16}).
%\textbf{Open problems discussion:} 
%\begin{itemize}
%\item Understand the set of LSH and how they can be used for practical noisy sources.
%\item Is there any power to interactive solutions?  Everything I've seen is single message each way.
%\item Integrate LSH and an m-out-of-n search scheme to mitigate the leakage associated with the Kuzu scheme.
%\item Design a stronger primitive than LSH that has some entropy preservation properties so we can talk about what cost there is in revealing when LSH collisions occur.  Adam's student did some work on distance preserving transforms.  Ben is thinking about this as well.  
%\item None of these schemes talk about updates.  Is there anything interesting to consider there?
%\end{itemize}
%
%\paragraph{Prior Work}
%
%\ben{These are my notes from last year, still need to condense}
%
%Work by Li et al.~\cite{li2010fuzzy,wang2013efficient} that was published at INFOCOM 2010.  
%
%\paragraph{Summary of Techniques}  Here we focus on how they deal with noise.  We ignore the searching index structure in the rest of their work.  This work is compatible with an arbitrary index structure.  The scheme is built on a system that can handle keyword equality.  The basic idea of this paper is to expand noise into a set of possible neighbors.  Three strategies are proposed:
%
%\begin{enumerate}
%\item When inserting a keyword, for example `Alice', also insert all of the neighbors into the index structure.  For example if considering Hamming distance of $1$, insert `Blice', `Clice', etc.  This technique results in an index structure size proportional to the number of neighbors.  Furthermore, the number of neighbors is leaked to the server as well documents that have neighboring keywords.  However, search proceeds the same way as in standard equality search.  The authors note that this approach is not tenable for any large amount of noise.
%\item The second technique is a hybrid technique where both the inserted keyword set and the searched queries are modified.  Instead of inserting `Alice', `Blice', etc. the following keywords are inserted `*lice', `A*lice', etc.  Then when performing the search the querier with the word `Alica' asks all documents containing one of `*lice', ..., `Alic*', etc.  Note that this approach requires as underlying scheme that can handle disjunctive queries.  As such there may be more leakage associated with this approach.
%\item The third approach is to insert word GRAMS.  The idea is to insert all possible nearby substrings of a particular length.  I didn't really follow their discussion well on why this was better than the wildcard based approach.  But this technique seems to be similar to what is being done in conjunctive search system out of IBM research.
%\end{enumerate}
%
%\paragraph{Security}
%This work never presents a formal definition of security.  They roughly say that the server is allowed to learn ``the outcome and the pattern of search queries.''  They claim to be using the security definition of Curtmola et al.~\cite{curtmola2011searchable}.  For their basic scheme they leak when two documents have the same neighbor which is the same as leaking if they are within twice the noise tolerance.  
%
%\paragraph{Major gaps}
%This paper presents a relatively weak efficiency guarantees.  All of the schemes involve exhaustively listing the neighbors in one form or another.  Furthermore, the number of neighbors and the neighborhood pattern of all documents is revealed.


%\subsection{Component A-III:  Property-Preserving Encryption for Graph Data} 
%
%For graph data, shortest path and disease propagation queries are common.  Previous work looks at supporting shortest path queries in the context of searchable symmetric encryption.  Accordingly, we propose to investigate \emph{shortest path revealing  and preserving encryption} (SPRE and SPPE) and \emph{random walk revealing and preserving encryption} (RWRE and RWPE).   We believe that the study of such encryption schemes will lead to interesting questions in graph theory.  We will particularly try to leverage work on differentially private graph sanitization. 
%
%Regarding shortest path revealing schemes: 
%
%\begin{question}
%Can we design efficient SPPE?  What security can be achieved by such schemes?
%\end{question}
%
%
%\begin{question}
%Can we design efficient SPRE with better security?
%\end{question}
%
%\begin{question}
%Can we design ``partial'' SPPE with better security?
%\end{question}
%
%And then regarding random walk preserving schemes:
%
%\begin{question}
%Can we design efficient RWPE?  What security can be achieved by such schemes?
%\end{question}
%
%
%\begin{question}
%Can we design efficient RWRE with better security?
%\end{question}
%
%\begin{question}
%Can we design ``partial'' RWPE with better security?
%\end{question}





\subsection{Component A-II:  Property-Preserving Encryption for Time Series Data} 

In time series data one is often interested in correlations and anomalies.  Accordingly, we propose to look at \emph{correlation revealing and preserving encryption}  (CRE and CPE) and \emph{anomaly revealing and preserving encryption} (ARE and APE).  In other words, in the ``preserving'' case we are interested in perturbing statistical data in a way that preserves statistics or the fact that a point is an anomaly.  

Regarding correlation revealing schemes
\begin{question}
Can we design efficient CPE for correlations of interest?  Which correlations should we target?  What security can be achieved by such schemes?
\end{question}


\begin{question}
Can we design efficient CRE with better security?
\end{question}

\begin{question}
Can we design ``partial'' CPE with better security?
\end{question}


Regarding anomaly  evealing schemes
\begin{question}
Can we design efficient APE?  How should anomaly thresholds be set?  What security can be achieved by such schemes?
\end{question}


\begin{question}
Can we design efficient ARE with better security?
\end{question}

\begin{question}
Can we design ``partial'' APE with better security?
\end{question}

\subsection{Component A-III:  Property-Preserving Encryption for Graph Data} 

For graph data, shortest path and disease propagation queries are common.  Previous work looks at supporting shortest path queries in the context of searchable symmetric encryption.  Accordingly, we propose to investigate \emph{shortest path revealing  and preserving encryption} (SPRE and SPPE) and \emph{random walk revealing and preserving encryption} (RWRE and RWPE).   We believe that the study of such encryption schemes will lead to interesting questions in graph theory.  We will particularly try to leverage work on differentially private graph sanitization. 

Regarding shortest path revealing schemes: 

\begin{question}
Can we design efficient SPPE?  What security can be achieved by such schemes?
\end{question}


\begin{question}
Can we design efficient SPRE with better security?
\end{question}

\begin{question}
Can we design ``partial'' SPPE with better security?
\end{question}

And then regarding random walk preserving schemes:

\begin{question}
Can we design efficient RWPE?  What security can be achieved by such schemes?
\end{question}


\begin{question}
Can we design efficient RWRE with better security?
\end{question}

\begin{question}
Can we design ``partial'' RWPE with better security?
\end{question}


 \section{Component B: Approaches Based on Secure Multiparty Computation}
 %We note that to the best of our knowledge, no prior protocols supporting search on encrypted data support (efficient) relevance ranking. 

\subsection{Component B-I:  }

\subsection{Component B-II:  }


\section{Component C: Leakage-Abuse Defenses}

\subsection{Component C-I: Extensions to Recent Attacks}

\subsection{Component C-II:  Attack Mitigations}

\paragraph*{Modularity.}  One mitigation of the attacks we propose to look at is \emph{modularity} in the sense of modular order-preserving encryption (M-OPE)~\cite{C:BolCheOne11}.  The idea of M-OPE is to apply a secret random offset modulo the largest possible message to a message before encrypting it (the secret random offset is chosen once and fixed in the secret key), so that everything gets ``shifted.''  Although similar attacks seem to apply to M-OPE, we propose to investigate more fine-grained modularity as a defense.  In the specific case of order-preserving encryption, we propose \emph{digit-modular} OPE (DM-OPE) where there is a secret modular offset applied to each digit.  The base in which the data is written could even itself be secret.  It becomes more complicated to make range queries with DM-OPE, as it requires an exponential number of queries in the number of ``wrap around'' digits in the query.  However, we propose to investigate approximating the queries efficiently (with some false positives that the client can filter out).




\section{Prior Accomplishments and NSF Support}

\textbf{Adam O'Neill:}
In his Ph.D.~work, the PI  developed the notions of deterministic encryption~\cite{C:BelBolOne07,ABO07,C:BolFehOne08,C:BFOR08,TCC:FulNeiRey12} and order-preserving encryption~\cite{EC:BCLO09,C:BolCheOne11} to help enable search on encrypted data with processing time comparable to that for unencrypted data, while providing as-strong-as-possible security guarantees subject to this constraint.
The PI has also worked on  instantiating random oracles~\cite{C:KilOneSmi10,TCC:GoyONeRao11,EC:LewONeSmi13},   aggregate signatures~\cite{CCS:BGOY07,AC:GLOW12}, deniable encryption~\cite{C:OnePeiWat11},  chosen-ciphertext security~\cite{EC:KilMohOne10,PKC:DFMO14}, and  functional encryption~\cite{EPRINT:ONeill10b,C:DIJOPP13, CANS:BelONe13}.   %Furthermore, he collaborated with database faculty George Kollios on refinements to order-preserving encryption~\cite{}.
Since joining Georgetown, he has also been working on  applications of indistinguishability obfuscation~\cite{DGLOZ14} and on integrating cryptography with emerging applications, such as outsourced database systems using modular order-preserving encryption~\cite{SIGMOD:MCOKC15} and  privacy preserving network provenance using structured encryption~\cite{LOSZZ14}.

Prior support: ``Program Obfuscation: From Theory to Practice." NSF Research Experiences for Undergraduates Supplement (PI).
Supplement to Award \#IIP-1362046,   2014 - 2019, \$8,000.

\textbf{Benjamin Fuller:}
In his Ph.D.~work, the co-PI worked on deterministic encryption with Dr. O'Neill \cite{TCC:FulNeiRey12,JC:FulONeRey15}.  His main focus was on cryptography with noise developing new fuzzy extractors~\cite{AC:FulMenRey13,AC:FulReySmi16,EC:CFPRS16}.  Fuzzy extractors can be thought of as a special case of distance preserving encryption where only a single point is comparable.  He then oversaw evaluation and implementation of encrypted search systems at MIT Lincoln Laboratory as part of the IARPA SPAR project~\cite{spar_baa} including BLIND SEER codeveloped by Dr. Choi.  Since joining the University of Connecticut in 2016, his work has focused on driving cryptography to practice including authentication and fuzzy extractors~\cite{EPRINT:HFDD17,EPRINT:BKFY17,EPRINT:ABCFG16}, database search~\cite{SP:FVYSHG17}, and multi-party computation~\cite{EPRINT:CunFulYak16}.  

Prior NSF support: not applicable.

\textbf{Seung Geol Choi:}
\section{Schedule and Management Plan}

\bibliographystyle{alpha}
\bibliography{Bencrypto,cryptobib/abbrev3,cryptobib/crypto,new}





\end{document}
