%!TEX root = dbproposal.tex

\section{Introduction}

The importance of collecting and storing data is universal, with use cases in governmental~\cite{Powers2014}, commercial~\cite{Linoff:2002:MWT:560274,insightdata}, and personal sectors~\cite{Mons2011}.  Tremendous value can be extracted from data.  Data enables better decisions, improved health and economic growth.

This data is stored in database systems.  Databases have a rich history in both academia and the commercial spaces.  However, data volume is growing exponentially.  This growth in size and complexity is outpacing organizations' ability to manage and
organize this data.  Organizations are turning to external cloud providers
to manage their data needs.  

This outsourced storage represents a natural attack target.  Attacks occur against both government \cite{CyberAttacksOPM} and commercial~\cite{CyberAttacks,gressin2017equifax} datasets.
One natural response to this risk is to encrypt data before outsourcing. 
%
However, employing encryption comes at the cost of disabling the cloud server
from {\em quickly processing the data and answering
complex queries from the client}. 
%
Ideally, we could use more sophisticated cryptographic techniques to {\em create databases
capable of efficiently answering a client's queries without revealing
information to the cloud server}.  


Secure outsourced databases use advanced cryptography to achieve this goal.  This field encompasses a variety of cryptographic techniques, including property-preserving encryption or PPE~\cite{EC:PanRou12}, searchable symmetric encryption or SSE~\cite{CCS:CGKO06}, private information retrieval by keyword~\cite{EPRINT:ChoGilNao98}, and public-key encryption with keyword search~\cite{EC:BDOP04}.  

This research has largely split into two research threads: PPE which
emphasizes background compatibility and use of legacy database
management systems, and SSE which emphases security.  PPE creates
symmetric encryption techniques
compatible with an unprotected database. Examples include deterministic
encryption~\cite{C:BelBolONe07}, which can answer equality queries, and
order-preserving encryption~\cite{C:BolCheONe11,EC:BCLO09},
which can answer range queries. Academic teams, start-up companies (including Bitglass, Ciphercloud, Crypteron, PreVeil, Skyhigh, ZeroDB) and Fortune
500 companies (including Microsoft's SQL Server 2016 and Azure and Google's Encrypted BigQuery)  offer variants of property-preserving encryption.
%Recent attacks against these schemes indicate even ideal property-preserving
%encryption can't be secure for many applications.
%
SSE schemes can be viewed as starting from secure multi-party
computation and
optimizing solutions for common database tasks (both \cite{SP:FVKKKM15} and \cite{RSA:IKLO16} explicitly use multi-party computation for sensitive subcomputations). This approach requires redesign
of database indexing mechanisms and achieves better security at the cost of decreased efficiency and backward compatibility. 

These systems have been implemented at moderate scale.  Both property-preserving solutions~\cite{CACM:PRZB12,EPRINT:PodBoePop16} and searchable encryption~\cite{SP:PKVKMC14,SP:FVKKKM15,C:CJJKRS13,CCS:JJKRS13,NDSS:CJJJKR14,ESORICS:FJKNRS15,RSA:IKLO16} solutions have been tested on datasets with billions of records.

%
%  Folks--we need to be careful here.  DBMS functions are much broader than search; there's transformation and presentation, access control, backup, recovery management, etc., and we're talking about none of that here.


\subsection{Use Cases}
Emerging databases include large scale graph databases, analytic databases, and
biometric databases. 
\begin{enumerate}
\item Many data sets are naturally interpreted as large sparse graphs.  Examples include social networks such as Facebook and Twitter and communities such as organizational communication, academic co-authorship, and the co-stardom network.  This type of network is also used to perform Internet scale analysis.  Common graph algorithms include computing triangles (sets of nodes $\{a,b,c\}$ where all pairs are close according to a metric), shortest path algorithms, network diameter, and degree distribution.
\item  Increasingly, databases are not asked to return subsets of data but rather derive statistics and analytics about the stored data.  Machine-learning-as-a-service has emerged as a business model for large and small companies~\cite{mlservice}.  Many machine learning algorithms depend on computing distance between points. Linear regression finds the line that minimizes the sum of distances between the line and data points.  Similarly, the first principal component minimizes the sum of distances between the selected line and the dataset~\cite{wold1987principal}.  The remaining principal components are defined similarly (they must also be orthogonal to previously defined components).
\item The FBI has long held a fingerprint database with hundreds of millions of records~\cite{brislawn1996fbi} for identifying criminals.  Increasingly, countries are using biometrics as an identifier for citizens, linking biometrics with unique identifiers.  The Aadhaar system in India links biometrics with a unique 12 digit number with over 1 billion numbers issued~\cite{daugman2014600}.  Increasingly, passports are biometric enabled~\cite{stanton2008icao}.  As these databases move to indexing all citizens, privacy concerns abound.  These databases compare the distance betweeen a target point $a$ and the set of stored points $b_i$. They return $b^*$ such that $b^* = \min_i d(a,b_i)$ if the distance $d(a,b^*)$ is less than some defined threshold.  
\end{enumerate}

\noindent
These databases share a fundamental operation: {\em
computing distance/proximity of tuples of points}. 

\subsection{Inadequacy of Prior Work}

In 2000, Song, Wagner, and Perrig provided the first scheme with communication proportional to the description of the query and the server performing (roughly) a linear scan of the encrypted database~\cite{SP:SonWagPer00}.  There has been tremendous work since 2000: both PPE and SSE approaches handle much of SQL and NoSQL queries and scale to datasets of billions of records.  However, neither approach is capable of handling geometric data that is prominent in the applications discussed above. (There is some work on computing shortest path in networks~\cite{CCS:MKNK15}.)  Furthermore, each approach also has a second weakness:
\begin{enumerate}
\item PPE has been subject to a number of leakage-abuse attacks that show that order-preserving encryption (and sometimes deterministic encryption) is not safe in most cases~\cite{CCS:NavKamWri15,CCS:CGPR15,CCS:KKNO16,CCS:PouWri16,CCS:GMNRS16,EPRINT:GSBNR16,EPRINT:ZhaKatPap16}.  Industry has primarily adopted this approach.
\item SSE offers more limited functionality and efficiency than PPE based solutions.  The fastest SSE based solutions report overhead of 300\%~\cite{C:CJJKRS13,CCS:JJKRS13,NDSS:CJJJKR14,ESORICS:FJKNRS15} while PPE based solutions report overhead of 30\%~\cite{CACM:PRZB12}.  We are only aware of a single SSE solution that can handle JOIN statements~\cite{EPRINT:KamMoa16}.  As of this writing, the information  leaked by this scheme is not clear.  These solutions replace the entire database software stack with a custom ``cryptographic'' database.  We posit that the lack of backward compatibility and administrative drawbacks hurt industry adoption.  
\end{enumerate}

The emergence of PPE systems indicates the approach has benefits beyond efficiency.  These benefits include backward compatibility, use of legacy software, and improved transparency.
The proposed research will
consider proximity queries as a case study to understand the following
question.  \begin{quote}\emph{Is it possible to securely outsource
modern databases while using a traditional database software stack?
What is the minimum amount that an unprotected database must be modified
to achieve security?}\end{quote}


\paragraph{Out of scope.} We do not address approaches based on fully-homomorphic encryption~\cite{STOC:Gentry09} or functional encryption~\cite{FOCS:GGHRSW13}.  These solutions are too slow to be used for the scale of current data sets.  In addition, we do not consider improvements to private information retrieval~\cite{FOCS:CGKS95} or oblivious RAM~\cite{STOC:Goldreich87,goldreich1996software} to be in scope for this proposal.  Our proposed work will make use of these objects but will not try to improve their efficiency.  The PIs are aware of (and have contributed to) the leakage inference attacks on secure databases.  These attacks will inform the approach in this proposal but developing new attacks is not a component in the proposed research.  This choice is being made to keep scope feasible.

Lastly, Intel SGX is a promising hardware approach that is being used to isolate programs and provide security~\cite{EPRINT:CosDev16}.  SGX can be used to simulate cryptographic primitives~\cite{EPRINT:SasGorFle17,EPRINT:FVBG16}.  We believe that SGX can be used to upgrade the security of secure databases from honest-but-curious to malicious.  However, we leave combining our techniques with secure hardware to future work.


\subsection{Proposed work}
The goal of this proposal is to create secure outsourced databases that
will be adopted and used by industry.  Towards achieving this goal we
recognize two lessons from the past: (1) PPE must be carefully analyzed
to understanding security and (2) there are tremendous benefits of
maintaining backwards compatibility with existing database systems.
Together, these lessons tell us that some modification of database
systems may be necessary but these modifications should be judicious.
This will be our guiding principle thorough this project.  Our goal is to create a third approach to secure outsourced databases: 
\begin{quote}Cryptographic operations are restricted to only database operators (used to create index structures).  These cryptographic operators can call interactive protocols but do not require replacing the unprotected database. 
\end{quote}

\noindent
The proposed research will
consider proximity queries as a case study and give constructions supporting
these queries.  The research will be split into three components:
\begin{description}
\item[\Cref{sec:ppe}.] To show the promise of this approach, we
will first consider whether property-preserving encryption can be used
to answer proximity queries.  We expect that the strength and utility of
\emph{distance-preserving encryption} will depend on the definition of
security, in particular questions such as: (1) should distance be precisely preserved? 
(2) is distance-revealing encryption sufficient? and (3) what is the distance
metric? Building on the research of the PI on order-preserving
encryption~\cite{EC:BCLO09,C:BolCheONe11}, this component will consist
of the following tasks:
\begin{enumerate}
\setlength\itemsep{0em}
\item Analysis of security provided by distance-preserving and distance-comparison preserving encryption for common metrics
\item Design of distance-revealing encryption
\item Design of approximate distance-revealing encryption
\end{enumerate}

\item[\Cref{sec:interactive}.] When the security provided by
distance-revealing encryption appears inadequate for an application, we
will change the model to allow operators to call interactive protocols.
We will show what functionality can be built using \emph{interactive
operators}, drawing on recent work of the
co-PIs~\cite{SP:PKVKMC14,CCS:RACY16}.  This component will consist of
the following tasks:
\begin{enumerate}
\setlength\itemsep{0em}
\item Euclidean Proximity using Partial Distance-Preserving Encryption
\item Edit Distance using Partial Suffix Tree Encryption
\item Improved efficiency of MPC approaches in the two-party setting
\item Improved dynamism for interactive approaches
%TODO Dan: I don't know what dynamism means in this context...
\end{enumerate}

\item[\Cref{sec:analysis}.] Lastly, we will strengthen
security of the interactive approach of \Cref{sec:interactive}
while working towards a goal of maximum
backwards compatibility as achieved by the techniques of \Cref{sec:ppe}.
This component includes considering forward
security and eliminating leakage from protocols, and will
consist of the following tasks:
\begin{enumerate}\setlength\itemsep{0em}
\item Modular partial order-preserving encryption
\item Forward security for POPE
\item Combining ORAM and interactive approaches to reduce leakage
\end{enumerate}
\end{description}
The most promising of these approaches will be implemented.
We will seek out and utilize benchmarks, data sets, and query loads from
the state of the art in database development and academic research in
testing our schemes.
We will use the UConn HPC Cluster to evaluate our
implementation on realistic-size databases (\url{https://hpc.uconn.edu}).
We expect this evaluation
to use and extend the database and query generator created as part of
the IARPA SPAR project~\cite{varia2015automated}.

\subsection{Intellectual Merits and Broader Impacts}
\paragraph{Intellectual merits.}  
Encrypted databases and searchable encryption have rich histories rooted in the
design of oblivious random access machines.  The field has been the focus of multiple large scale projects including IARPA's APP and SPAR~\cite{spar_baa}.  The field is quite diverse bringing together cryptographers, system researchers, and database experts.  Furthermore, there is clear demand in industry for solutions.  Data breaches are becoming nearly daily events.  However, recent
leakage inference attacks have taught us that just because something is called encrypted does not make it secure.  This problem requires careful design that balances functionality, efficiency, and security needs.  The rapid deployment of property-preserving
techniques has reinforced the importance of simplicity, efficiency and backward
compatibility.  These two developments inform the core of our approach: using
interactivity to strengthen property-preserving encryption.

\paragraph{Broader impacts.}
The confidentiality of data is a core societal tenant.  Deployed encrypted
databases provide little security and may even hurt by providing a false sense
of security.  There is a tremendous need for research in this area to
understand the tradeoffs between security, functionality, and efficiency.

The PIs are committed to broad dissemination of research material.  The PIs
have participated in large scale evaluation of searchable encryption, working
on both constructions and attacks. The PIs have a track record of
releasing the source code implementations of their work (including
previous work on searchable encryption) and contributing
to open-source projects.
The protocols and other products of this proposal will be released
publicly as open-source implementations.

Lastly, all PIs are
dedicated to engaging with undergraduates. Two co-PIs being at an
undergraduate only institution, the US Naval Academy, where co-PI Roche
was recently recognized with the institution-wide Apgar Award for
Teaching.  PI Fuller supervises the cyber-security club at the
University of Connecticut and supports their efforts to understand and
research computer security.  This club allows students with varying
educational preparation to engage outside of the classroom and learn the
impact of computer science and security.  Co-PI O'Neill has previously
been awarded an REU and used it to  work with a female undergraduate
student.



